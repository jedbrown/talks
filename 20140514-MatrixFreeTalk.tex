% \documentclass[handout]{beamer}
\documentclass{beamer}

\mode<presentation>
{
  \usetheme{ANLBlue}
  % \usefonttheme[onlymath]{serif}
  % \usetheme{Singapore}
  % \usetheme{Warsaw}
  % \usetheme{Malmoe}
  % \useinnertheme{circles}
  % \useoutertheme{infolines}
  % \useinnertheme{rounded}

  \setbeamercovered{transparent=20}
}

\usepackage[english]{babel}
\usepackage[latin1]{inputenc}
\usepackage{alltt,listings,multirow,ulem,siunitx}
\usepackage[absolute,overlay]{textpos}
\TPGrid{1}{1}
\usepackage{pdfpages}
\usepackage{ulem}
\usepackage{multimedia}
\usepackage{multicol}
\newcommand\hmmax{0}
\newcommand\bmmax{0}
\usepackage{bm}
\usepackage{comment}
\usepackage{subcaption}

% font definitions, try \usepackage{ae} instead of the following
% three lines if you don't like this look
\usepackage{mathptmx}
\usepackage[scaled=.90]{helvet}
% \usepackage{courier}
\usepackage[T1]{fontenc}
\usepackage{tikz}
\usetikzlibrary{decorations.pathreplacing}
\usetikzlibrary{shadows,arrows,shapes.misc,shapes.arrows,shapes.multipart,arrows,decorations.pathmorphing,backgrounds,positioning,fit,petri,calc,shadows,chains,matrix}

\newcommand\vvec{\bm v}
\newcommand\bvec{\bm b}
\newcommand\bxk{\bvec_0 \times \kappa_0 \cdot \nabla}
\newcommand\delp{\nabla_\perp}

% \usepackage{pgfpages}
% \pgfpagesuselayout{4 on 1}[a4paper,landscape,border shrink=5mm]

\usepackage{JedMacros}

\newcommand{\timeR}{t_{\mathrm{R}}}
\newcommand{\timeW}{t_{\mathrm{W}}}
\newcommand{\mglevel}{\ensuremath{\ell}}
\newcommand{\mglevelcp}{\ensuremath{\mglevel_{\mathrm{cp}}}}
\newcommand{\mglevelcoarse}{\ensuremath{\mglevel_{\mathrm{coarse}}}}
\newcommand{\mglevelfine}{\ensuremath{\mglevel_{\mathrm{fine}}}}

%solution and residual
\newcommand{\vx}{\ensuremath{x}}
\newcommand{\vc}{\ensuremath{\hat{x}}}
\newcommand{\vr}{\ensuremath{r}}
\newcommand{\vb}{\ensuremath{b}}

%operators
\newcommand{\vA}{\ensuremath{A}}
\newcommand{\vP}{\ensuremath{I_H^h}}
\newcommand{\vS}{\ensuremath{S}}
\newcommand{\vR}{\ensuremath{I_h^H}}
\newcommand{\vI}{\ensuremath{\hat I_h^H}}
\newcommand{\vV}{\ensuremath{\mathbf{V}}}
\newcommand{\vF}{\ensuremath{F}}
\newcommand{\vtau}{\ensuremath{\mathbf{\tau}}}


\title{High-performance matrix-free operator application and preconditioning}
\author{{\bf Jed Brown} \texttt{jedbrown@mcs.anl.gov} (ANL and CU Boulder) \\
  \quad Dave May (ETH Z\"urich), Matt Knepley (UChicago)
}

% - Use the \inst command only if there are several affiliations.
% - Keep it simple, no one is interested in your street address.
% \institute
% {
%   Mathematics and Computer Science Division \\ Argonne National Laboratory
% }

\date{Sandia, 2014-05-14}

% This is only inserted into the PDF information catalog. Can be left
% out.
\subject{Talks}


% If you have a file called "university-logo-filename.xxx", where xxx
% is a graphic format that can be processed by latex or pdflatex,
% resp., then you can add a logo as follows:

% \pgfdeclareimage[height=0.5cm]{university-logo}{university-logo-filename}
% \logo{\pgfuseimage{university-logo}}



% Delete this, if you do not want the table of contents to pop up at
% the beginning of each subsection:
% \AtBeginSubsection[]
% {
% \begin{frame}<beamer>
%   \frametitle{Outline}
%   \tableofcontents[currentsection,currentsubsection]
% \end{frame}
% }

% \AtBeginSection[]
% {
%   \begin{frame}<beamer>
%     \frametitle{Outline}
%     \tableofcontents[currentsection]
%   \end{frame}
% }

% If you wish to uncover everything in a step-wise fashion, uncomment
% the following command:

% \beamerdefaultoverlayspecification{<+->}

\begin{document}
\lstset{language=C}
\normalem

\begin{frame}
  \titlepage
\end{frame}

\begin{frame}{Don't assemble}
  \begin{itemize}
  \item The purpose of a solver is to solve equations
  \item Matrix assembly doesn't solve equations
  \item Matrix assembly is only needed if the solver algorithm requires it
  \item Hardware is changing
    \begin{itemize}
    \item Longer to assemble a matrix than to solve using unassembled method
    \item Longer to compute residual with pre-assembled matrix than to solve
    \end{itemize}
  \item Characterize performance using arithmetic intensity
    \begin{itemize}
    \item depends on cache reuse model
    \end{itemize}
  \item I will look at finite-element methods in 3D
    \begin{itemize}
    \item mostly with $Q_2$ elements: $3\times 3\times 3$ points for each element
    \item tensor contraction kernels across these elements
    \item isoparametrically mapped elements
    \item other discretizations have similar structure, different constants
    \end{itemize}
  \end{itemize}
\end{frame}

\begin{frame}{Hardware Arithmetic Intensity}
  \begin{tabular}{lc}
    \toprule
    Operation                         & Arithmetic Intensity (flops/B) \\
    \midrule
    Sparse matrix-vector product      & 1/6                  \\
    Dense matrix-vector product       & 1/4                  \\
    Unassembled matrix-vector product & $\approx 8$          \\
    High-order residual evaluation    & $> 5$                \\
    \bottomrule
  \end{tabular}

  \bigskip

  \begin{tabular}{lrrr}
    \toprule
    Processor & BW (GB/s) & Peak (GF/s) & Balanced AI (F/B) \\
    \midrule
    E5-2670 8-core      & 35   & 166  & 4.7 \\
    Magny Cours 16-core & 49   & 281  & 5.7 \\
    Blue Gene/Q node    & 43   & 205  & 4.8 \\
    Tesla M2090         & 120  & 665  & 5.5 \\
    Kepler K20Xm        & 160 & 1310 & 8.2 \\ % http://www.elekslabs.com/2012/11/nvidia-tesla-k20-benchmark-facts.html
    Xeon Phi            & 150 & 1248 & 8.3 \\
    \bottomrule
  \end{tabular}
\end{frame}


\begin{frame}[shrink=5]{Performance of assembled versus unassembled}
  \vspace{1ex}
  \includegraphics[width=\textwidth]{figures/TensorVsAssembly} \\
  \begin{itemize}
  \item Arithmetic intensity for $\Qk p$ elements
    \begin{itemize}
    \item $< \frac 1 4$ (assembled), $\approx 10$ (unassembled), $\approx 4$ to $8$ (hardware)
    \end{itemize}
  \item store Jacobian information at Quass quadrature points, can use AD
  \end{itemize}
\end{frame}

\begin{frame}{But what are the constants?}
  \begin{itemize}
  \item Viscous operator application for heterogeneous Stokes $Q_2$-$P_1^{\text{disc}}$
  \item ``Tensor'': matrix-free implementation using tensor product structure on the reference element
  \item ``Tensor C'' absorbs metric term into stored tensor-valued coefficient
  \item Performance on 8 nodes of Edison (3686 GF/s peak)
  \end{itemize}
  \begin{tabular}{lrrrrrrr}
    \toprule
    Operator & flops & \multicolumn{2}{c}{Pessimal cache} & \multicolumn{2}{c}{Perfect cache} & Time & GF/s \\
    & & bytes & F/B & bytes & F/B & (ms) & \\
    \midrule
    Assembled & 9216 & --- & --- &  37248 & 0.247 & 42 & 113 \\
    Matrix-free & 53622 & 2376 & 22.5 & 1008 & 53 & 22 & 651 \\
    Tensor & 15228 & 2376 & 6.4 & 1008 & 15 & {\bf 4.2} & 1072 \\
    Tensor C & 14214 & 5832 & 2.4 & 4920 & 2.9 & --- & --- \\
    \bottomrule
  \end{tabular}
  % edison-8sinker-64-asm-00192.log:MGResid Level 2      247 1.0 1.0271e+01 1.1 6.50e+09 1.2 8.2e+05 8.3e+03 0.0e+00 10  8  3  7  0  13 14  4 11  0 112942
  % edison-8sinker-64-avx-00192.log:MGResid Level 2      247 1.0 1.0430e+00 1.0 6.01e+09 1.1 1.6e+06 5.5e+03 0.0e+00  3  7  5  9  0   6 14  6 12  0 1072558
  % edison-8sinker-64-ref-00192.log:MGResid Level 2      247 1.0 5.4271e+00 1.0 1.90e+10 1.1 1.6e+06 5.5e+03 0.0e+00  8 12  5  9  0  12 15  6 12  0 651324
\end{frame}

\begin{frame}{Cache versus vectorization}
  \begin{itemize}
  \item Fundamental trade-off
  \item Hardware gives us less cache per vector lane
  \item Intra-element vectorization is complicated and \"uber-custom
  \item Coordinate transformation is $27\cdot 9\cdot \texttt{sizeof(double)} = 1944$ bytes/element.
  \item Vectorize over 4 or 8 elements, perhaps hardware threads
  \item L1 cache is not this big: repeated spills in tensor contraction
  \item This is a \emph{very} simple problem
  \end{itemize}
\end{frame}

\begin{frame}{Matrix-free operator application}
  Discretize $-\nabla \Big(\kappa \nabla\cdot u\Big)$, yielding
  \begin{equation*}\label{eq:mf-scalar}
    A \mathbf u = \sum_{e \in \text{Elements}} \mathcal E_e^T \mathcal D_{\mathbf x}^T \Lambda(\omega \kappa) \mathcal D_{\mathbf x} \mathcal E_e \mathbf u
  \end{equation*}
  Physical gradient matrix
  $$\mathcal D_{\mathbf x} = \{\mathcal D_i | i \in \{x,y,z\} \} \in \mathbb R^{81\times 27} \quad \text{$Q_2$ elements in 3D; Gauss quadrature}$$
  \begin{itemize}
  \item Common method: precompute reference gradient matrix $\mathcal D_{\mathbf \xi}$, map to physical $\mathcal D_{\mathbf x} = \Lambda(\nabla_{\mathbf x}\mathbf\xi) \mathcal D_{\mathbf \xi}$
  \item Better: rearrange
  \begin{equation*}\label{eq:tensor}
    A \mathbf u = \sum_{e \in \text{Elements}} \mathcal E_e^T \mathcal D_{\mathbf \xi}^T \Lambda\Big((\nabla_{\mathbf x}\mathbf\xi)^T (\omega \kappa) (\nabla_{\mathbf x}\mathbf\xi) \Big) \mathcal D_{\mathbf \xi} \mathcal E_e \mathbf u .
  \end{equation*}
  where $\nabla_{\mathbf x}\mathbf\xi = (\nabla_{\mathbf\xi}\mathbf x)^{-1}$ is computed at quadrature points from the coordinate gradients.
  \begin{equation*}
    D_{\mathbf\xi} = \{\hat D \otimes \hat B \otimes \hat B, \hat B \otimes \hat D \otimes \hat B, \hat B \otimes \hat B \otimes \hat D \}
  \end{equation*}
  \end{itemize}
\end{frame}

\begin{frame}{Assembly-free preconditioning}
  \begin{itemize}
  \item ``Optimization'' is pessimization if it compromises convergence
  \item pTatin3D: long-term lithosphere/tectonics package
    \begin{itemize}
    \item Dave May (ETH Z\"urich) and Laetitia Le Pourhiet (UPMC Paris)
    \item visco-plastic rheology, post-failure deformation, thermodynamics, free-surface
    \item multi-material transport using particles; $10^{10}$ jumps in coefficients
    \end{itemize}
  \item Block preconditioning with MG solve in viscous block
  \item Matrix-free fine grid, start coarsening geometrically
  \item Switch to Galerkin, smoothed aggregation (GAMG or ML)
  \end{itemize}
  \begin{tabular}{l rrr rr r}
    \toprule
    Operator    & Cores & Grid   & El/core & \multicolumn{2}{c}{Solve/core} & Op/core           \\
    &       &        &         & El/s                    & GF/s & kEl/s \\
    \midrule
    Assembled   & 192 & $64^3$ & 1265 & 46  & 0.9 & 33         \\
    Matrix-free & 192 & $64^3$ & 1265 & 69  & 2.6 & 62         \\
    Tensor      & 192 & $64^3$ & 1265 & 128 & 2.4 & 323        \\
    
    Matrix-free & 1536 & $96^3$ & 576  & 47  & 2.2 & 60  \\
    Tensor      & 1536 & $96^3$ & 576  & 72  & 2.2 & 252 \\
    
    Tensor      & 12288 & $192^3$ & 576  & 26 & 1.1 & 166 \\
    \bottomrule
  \end{tabular}
\end{frame}

\begin{frame}{pTatin3D}
  \begin{columns}
    \begin{column}{0.5\textwidth}
      \begin{center}
        \includegraphics[width=\textwidth]{figures/LaetiRifting}
      \end{center}
    \end{column}
    \begin{column}{0.5\textwidth}
      \begin{center}
        \includegraphics[width=\textwidth]{figures/inclusionsNc75_tubes_spheres.pdf}
      \end{center}
    \end{column}
  \end{columns}
\end{frame}

\begin{frame}{HPGMG-FE: a finite-element FAS benchmark}
  \begin{itemize}
  \item \url{https://hpgmg.org}
  \item Unassembled $Q_2$ elements, FMG solve (3,1 cycle)
  \item Scales at 23\% of peak on Edison (Cray XC-30)
  \item 10x faster than assembled elements, less memory used
  \end{itemize}
  \centering
  \includegraphics[width=0.75\textwidth]{figures/MG/hpgmgfe-edison-vesta.png}
\end{frame}

\end{document}
