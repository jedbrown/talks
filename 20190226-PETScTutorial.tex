%\documentclass[handout]{beamer}
\documentclass{beamer}

\mode<presentation>
{
%\usetheme{Singapore}
%\usetheme{Warsaw}
\usetheme{Malmoe}
\useinnertheme{circles}
\useoutertheme{infolines}
% \useinnertheme{rounded}

\setbeamercovered{transparent}
}

\usepackage[english]{babel}
\usepackage[latin1]{inputenc}
\usepackage{bm,textpos,alltt,listings,multirow,ulem}
\usepackage[outputdir=out]{minted}
\usepackage{url}
\usepackage{JedMacros}

% font definitions, try \usepackage{ae} instead of the following
% three lines if you don't like this look
\usepackage{mathptmx}
\usepackage[scaled=.90]{helvet}
\usepackage{courier}
\usepackage[T1]{fontenc}
\usepackage{tikz}
\usetikzlibrary[shapes.arrows,arrows,shapes.misc,chains]


\makeatletter
\def\sectionintoc{}
\def\beamer@sectionintoc#1#2#3#4#5{%
\ifnum\c@tocdepth>0%
\ifnum#4=\beamer@showpartnumber%
{
  \beamer@saveanother%
  \gdef\beamer@todo{}%
  \beamer@slideinframe=#1\relax%
  \expandafter\only\beamer@tocsections{\gdef\beamer@todo{%
      \beamer@tempcount=#5\relax%
      \advance\beamer@tempcount by\beamer@sectionadjust%
      \edef\inserttocsectionnumber{\the\beamer@tempcount}%
      \def\inserttocsection{\hyperlink{Navigation#3}{#2}}%
      \beamer@tocifnothide{\ifnum\c@section=#1\beamer@toc@cs\else\beamer@toc@os\fi}%
      {
        \ifbeamer@pausesections\pause\fi%
        \ifx\beamer@toc@ooss\beamer@hidetext
          \vskip.2em
        \else
          \vskip.2em
        \fi
        {%
          \hbox{\vbox{%
              \def\beamer@breakhere{\\}%
              \beamer@tocact{\ifnum\c@section=#1\beamer@toc@cs\else\beamer@toc@os\fi}    {section in toc}}}%
         \par%
        }%
      }%
    }
  }%
  \beamer@restoreanother%
  }
  \beamer@todo%
  \fi\fi%
}
\makeatother


% \usepackage{pgfpages}
% \pgfpagesuselayout{4 on 1}[letterpaper,landscape,border shrink=5mm]

% Macros for the p-Bratu revision numbers
\def\Rbasic{0}
\def\Rrlap{1}
\def\Rrbratu{2}
\def\Rrpbratu{3}
\def\Rassemblebratu{4}
\def\Rassemblepicard{5}
\def\Rmyprealloc{6}
\def\Rnewtoncrash{7}
\def\Rnewtonbug{8}
\def\Rnewtonfix{9}

\title[https://jedbrown.org/files/20190226-PETScSolvers.pdf]{PETSc Solvers Tutorial}

\author{Jed Brown and Tobin Isaac and Karl Rupp}


% - Use the \inst command only if there are several affiliations.
% - Keep it simple, no one is interested in your street address.
\institute{CU Boulder, Georgia Tech, and TU Vienna}

\date{SIAM CSE, 2019-02-26, Spokane, WA}


% This is only inserted into the PDF information catalog. Can be left
% out.
\subject{Talks}


% If you have a file called "university-logo-filename.xxx", where xxx
% is a graphic format that can be processed by latex or pdflatex,
% resp., then you can add a logo as follows:

% \pgfdeclareimage[height=0.5cm]{university-logo}{university-logo-filename}
% \logo{\pgfuseimage{university-logo}}



% Delete this, if you do not want the table of contents to pop up at
% the beginning of each subsection:
\AtBeginSubsection[]
{
 \begin{frame}<beamer>
 \frametitle{Outline}
 \tableofcontents[currentsection,currentsubsection]
 \end{frame}
}

\AtBeginSection[]{
\begin{frame}<beamer>
  \frametitle{Outline}
  \tableofcontents[currentsection]
\end{frame}
}

% If you wish to uncover everything in a step-wise fashion, uncomment
% the following command:

%\beamerdefaultoverlayspecification{<+->}

\begin{document}
\lstset{language=C}

\begin{frame}
  \titlepage
\end{frame}

\begin{frame}
\frametitle{Outline}
\tableofcontents
% You might wish to add the option [pausesections]
\end{frame}
\begin{frame}{Follow Up; Getting Help}
  \begin{itemize}
    \item \url{http://www.mcs.anl.gov/petsc}
    \item Public questions: \url{petsc-users@mcs.anl.gov}, archived
    \item Private questions: \url{petsc-maint@mcs.anl.gov}, not archived
  \end{itemize}
\end{frame}



\section{Introduction}
\newcommand\ganttline[4]{% line, tag, start end
   \node at (0,#1/2+.1) [anchor=base east] {#2};
   \fill[blue] (#3/\xtick-1991/\xtick,#1/2-.1) rectangle (#4/\xtick-1991/\xtick,#1/2+.1);}
\newcommand\ganttlabel[6]{% year, label, color, yloc, anchor
  \node[#3] at (#1/\xtick+#6/\xtick-1991/\xtick,#4) [anchor=#5] {#2};
  \fill[#3] (#1/\xtick-1991/\xtick,1/2-.1) rectangle (#1/\xtick-1991/\xtick+0.04,12/2+.1);}

%\begin{frame}{Timeline}
\frame{
\begin{figure}[htbp]
\def\present{2011.7}
\def\xtick{2.2}
\begin{tikzpicture}[y=-1cm]
   %\draw[help lines] (0.5,5) grid (8,0.5);
   \ganttlabel{1991}{1991}{red}{6.2}{north}{0}
   \ganttlabel{1995}{1995}{red}{6.2}{north}{0}
   \ganttlabel{2000}{2000}{red}{6.2}{north}{0}
   \ganttlabel{2005}{2005}{red}{6.2}{north}{0}
   \ganttlabel{2010}{2010}{red}{6.2}{north}{0}
   \ganttlabel{1992}{PETSc-1}{green!70!black}{0}{center}{0}
   \ganttlabel{1994.4}{MPI-1}{magenta!70!black}{-.5}{center}{0}
   \ganttlabel{1997.6}{MPI-2}{magenta!70!black}{-.5}{center}{0}
   \ganttlabel{1995.5}{PETSc-2}{green!70!black}{0}{center}{0}
   \ganttlabel{2008.9}{PETSc-3}{green!70!black}{0}{center}{0}
   \ganttline{1}{Barry}{1991}{\present}
   \ganttline{2}{Bill}{1991}{1996}
   \ganttline{3}{Lois}{1993}{2001}
   \ganttline{4}{Satish}{1997}{\present}
   \ganttline{5}{Dinesh}{1998}{2005.5}
   \ganttline{6}{Hong}{2001}{\present}
   \ganttline{7}{Kris}{2001}{2006}
   \ganttline{8}{Matt}{2001.5}{\present}
   \ganttline{9}{Victor}{2003}{2006.9}
   \ganttline{9}{}{2007.3}{2007.5}
   \ganttline{9}{}{2008.5}{2008.7}
   \ganttline{10}{Dmitry}{2005.6}{\present}
   \ganttline{11}{Lisandro}{2006.9}{\present}
   \ganttline{12}{Jed}{2009}{\present}
   \ganttline{13}{Shri}{2009.8}{\present}
\end{tikzpicture}
\end{figure}
}
%\end{frame}

\input{slides/PETSc/About.tex}
\begin{frame}{Jupyter access}
  \begin{itemize}
  \item Visit \url{https://siam.petsc.org}
  \item Log in with your name
  \item Password: \texttt{siamcse19}
  \item Click on Terminal
  \end{itemize}
\end{frame}
%\section{Installation}
\begin{frame}{Downloading}
\begin{itemize}
  \item \url{http://mcs.anl.gov/petsc}, download tarball
  \item We will use Mecurial in this tutorial:
  \begin{itemize}
    \item \url{http://mercurial.selenic.com}
    \item Debian/Ubuntu: \shell{aptitude install mercurial}
    \item Fedora: \shell{yum install mercurial}
  \end{itemize}
  \item Get the PETSc release
  \begin{itemize}\footnotesize
    \item \shell{\small hg clone \mtab\bslash \\
     \scriptsize http://petsc.cs.iit.edu/petsc/releases/petsc-3.1}
    \item \shell{cd petsc-3.1}
    \item \shell{\scriptsize hg clone http://petsc.cs.iit.edu/petsc/releases/BuildSystem-3.1 \mtab\bslash \\
        \qquad config/BuildSystem}
    \item Get the latest bug fixes with \shell{hg pull -{}-update}
  \end{itemize}
\end{itemize}
\end{frame}

\begin{frame}{Configuration}
\begin{block}{Basic configuration}
\begin{itemize}\footnotesize
  \item \shell{export PETSC\_DIR=\$PWD PETSC\_ARCH=mpich-gcc-dbg}
  \item \shell{./configure -{}-with-shared \mtab \bslash \\
  	\qquad\qquad -{}-with-blas-lapack-dir=/usr \mtab\bslash \\
  	\qquad\qquad -{}-download-\{mpich,ml,hypre\}}
  \item \shell{make all test}
\end{itemize}
\end{block}
\begin{itemize}
\item Other common options
  \begin{itemize}\footnotesize
  \item \item \code{-{}-with-mpi-dir=/path/to/mpi}
  \item \code{-{}-with-scalar-type=$<$real or complex$>$}
  \item \code{-{}-with-precision=$<$single,double,longdouble$>$}
  \item \code{-{}-with-64-bit-indices}
  \item \code{-{}-download-\{umfpack,mumps,scalapack,blacs,parmetis\}}
  \end{itemize}
\item reconfigure at any time with \\
  {\footnotesize \shell{mpich-gcc-dbg/conf/reconfigure-mpich-gcc-dbg.py \mtab\bslash\\
      \qquad\qquad -{}-new-options}}
\end{itemize}
\end{frame}

\frame{
\frametitle{Automatic Downloads}

\begin{itemize}
  \item Most packages can be automatically
  \begin{itemize}
    \item Downloaded
    \item Configured and Built (in {\kb \$PETSC\_DIR/externalpackages})
    \item Installed with PETSc
  \end{itemize}

  \item Currently works for
  \begin{itemize}
    \item petsc4py
    \item PETSc documentation utilities (Sowing, lgrind, c2html)
    \item BLAS, LAPACK, BLACS, ScaLAPACK, PLAPACK
    \item MPICH, MPE, Open MPI
    \item ParMetis, Chaco, Jostle, Party, Scotch, Zoltan
    \item MUMPS, Spooles, SuperLU, SuperLU\_Dist, UMFPack, pARMS
    \item PaStiX, BLOPEX, FFTW, SPRNG
    \item Prometheus, HYPRE, ML, SPAI
    \item Sundials
    \item Triangle, TetGen, FIAT, FFC, Generator
    \item HDF5, Boost
  \end{itemize}
\end{itemize}
\emph{Can also use \code{-{}-with-xxx=/path/to/your/install}}
}


\begin{frame}{An optimized build}
  \begin{itemize}
  \item \shell{\small mpich-gcc-dbg/conf/reconfigure-mpich-gcc-dbg.py \\
      PETSC\_ARCH=mpich-gcc-opt \\
      -{}-with-debugging=0 \&\& make PETSC\_ARCH=mpich-gcc-opt}
  \item What does \code{-{}-with-debugging=1} (default) do?
    \begin{itemize}
    \item Keeps debugging symbols (of course)
    \item Maintains a stack so that errors produce a full stack trace (even SEGV)
    \item Does lots of integrity checking of user input
    \item Places sentinels around allocated memory to detect memory errors
    \item Allocates related memory chunks separately (to help find memory bugs)
    \item Keeps track of and reports unused options
    \item Keeps track of and reports allocated memory that is not freed \\
      \quad \code{-malloc\_dump}
    \end{itemize}
  \end{itemize}
\end{frame}


%\begin{frame}[shrink=20]{Interactions among composable linear, nonlinear, and timestepping solvers}
  \begin{tikzpicture}
    [obj/.style={rectangle,draw=blue!50!black,fill=blue!20,thick,minimum size=6mm},
    impl/.style={rounded rectangle,draw=green!50!black,fill=green!20,thick,font=\scriptsize},
    objbox/.style={rectangle,fill=none,draw=blue!50!black,thick}
    ]
    \begin{scope} [start chain,every node/.style={on chain,impl},node distance=1em]
      \node (arkimex) {ARK IMEX};
      \node (rosw)    {Rosenbrock-W};
      \node (ssp)     {SSP RK};
      \node (pseudo)  {Pseudo};
    \end{scope}
    \node[objbox,fit=(arkimex)(pseudo)] (tsbox) {};
    \node[obj,above={0cm of tsbox.north west},anchor=base] (ts) {TS};

    \begin{scope} [start chain,every node/.style={on chain,impl},node distance=1em]
      \node[below={3em of tsbox.south west},anchor=north west] (ls) {Newton line search};
      \node (viss) {VISS};
      \node (virs) {VIRS};
      \node (ms)  {Multi-stage};
      \node[below=of ls.south west,anchor=west] (ngmres) {NGMRES};
      \node (nrichardson)  {NRichardson};
      \node (fas)  {FAS};
      \node (qn)    {Quasi-Newton};
      \node (shell)  {Shell};
    \end{scope}
    \node[objbox,fit=(ls)(shell)] (snesbox) {};
    \node[obj,above={0cm of snesbox.north west},anchor=base] (snes) {SNES};

    \begin{scope} [start chain,every node/.style={on chain,impl},node distance=1em]
      \node[below={3em of ngmres},anchor=north west] (gmres) {GMRES};
      \node (fgmres) {FGMRES};
      \node (ibcgstab) {IBiCGStab};
      \node (cg) {CG};
      \node (preonly) {Pre only};
    \end{scope}
    \node[objbox,fit=(gmres)(preonly)] (kspbox) {};
    \node[obj,above={0cm of kspbox.north west},anchor=base] (ksp) {KSP};

    \begin{scope} [start chain,every node/.style={on chain,impl},node distance=1em]
      \node[below={3em of kspbox.south west},anchor=north east] (asm) {ASM};
      \node (fieldsplit) {FieldSplit};
      \node (mg) {MG};
      \node (pcksp) {KSP};
      \node (sor) {SOR};
      \node (ilu) {ILU};
      \node (lu) {LU};
      \node (pcshell) {Shell};
    \end{scope}
    \node[objbox,fit=(asm)(pcshell)] (pcbox) {};
    \node[obj,above={0cm of pcbox.north west},anchor=base] (pc) {PC};

    \begin{scope} [start chain,every node/.style={on chain,impl},node distance=1em]
      \node[below={3em of asm.south west},anchor=north east] (aij) {AIJ};
      \node (baij) {BAIJ};
      \node (sbaij) {SBAIJ};
      \node (nest) {Nest};
      %\node (is) {IS};
      \node (cusp) {CUSP};
    \end{scope}
    \node[objbox,fit=(aij)(cusp)] (matbox) {};
    \node[obj,above={0cm of matbox.north west},anchor=base] (mat) {Mat};

    \begin{scope} [start chain,every node/.style={on chain,impl},node distance=1em]
      \node[right={2em of cusp.east},anchor=west] (vecmpi) {MPI};
      \node (vecghost) {Ghost};
      \node (veccuda) {CUDA};
    \end{scope}
    \node[objbox,fit=(vecmpi)(veccuda)] (vecbox) {};
    \node[obj,above={0cm of vecbox.north west},anchor=base] (vec) {Vec};

    \begin{scope}
      [thick,draw=black!50,>=stealth,
      uses/.style={red!50!black,fill=red!20,font=\scriptsize}]
      \draw[>->] (arkimex) -- (snes) node[midway,uses] {$g(x,z+\alpha x,t) = 0$};
      \draw[>->] (rosw) to[out=-120,in=10] node[midway,uses] {$J_\alpha^{-1}$} (snes);
      \draw[>->] (pseudo) to[out=-150,in=0] node[midway,uses] {$J_\alpha^{-1}$} (snes);

      \draw[>->] (ngmres) to[out=180,in=-100] node[midway,uses] {npc} (snes.south west);
      \draw[>->] (nrichardson.south west) to[out=-150,in=-120] node[midway,uses] {npc} (snes.south west);

      \draw[>->] (ls.south west) to[out=-140,in=180] node[near end,uses] {$J^{-1}$} (ksp);
      \draw[>->] (virs.south west) to[out=-110,in=20] node[midway,uses] {$J_{\text{reduced}}^{-1}$} (ksp);
      \draw[>->] (qn) to[out=-150,in=0] node[near start,uses] {$H_0^{-1}$} (ksp);

      \draw[>->] (asm.west) to[out=170,in=-180] node[midway,uses] {sub} (ksp);
      \draw[>->] (fieldsplit.north west) to[out=150,in=-120] node[midway,uses] {split} (ksp.south west);
      \draw[>->] (mg.north west) to[out=150,in=-90] node[midway,uses] {levels} (ksp);
      \draw[>->] (pcksp) to[out=110,in=-90] node[midway,uses] {inner} (ksp);

      \draw[>->] (asm) to node[midway,uses] {overlap} (mat);
      \draw[>->] (sor) to[out=-165,in=10] node[near start,uses] {relax} (mat);
      \draw[>->] (lu) to[out=-160,in=0] (mat);
      \draw[>->] (ilu) to[out=-160,in=0] node[midway,uses] {factor} (mat);
      \draw[>->] (fieldsplit) to[out=-120,in=25] node[midway,uses] {sub} (mat);
    \end{scope}
  \end{tikzpicture}
\end{frame}


\section{Objects - Building Blocks of the Code}


\begin{frame}{MPI communicators}
  \begin{itemize}
  \item Opaque object, defines process group and synchronization channel
  \item PETSc objects need an \code{MPI\_Comm} in their constructor
    \begin{itemize}
    \item \code{PETSC\_COMM\_SELF} for serial objects
    \item \code{PETSC\_COMM\_WORLD} common, but \emph{not} required
    \end{itemize}
  \item Can split communicators, spawn processes on new communicators, etc
  \item Operations are one of
    \begin{itemize}
    \item Not Collective: \code{VecGetLocalSize(), MatSetValues()}
    \item Logically Collective: \code{KSPSetType(), PCMGSetCycleType()}
      \begin{itemize}
      \item checked when running in debug mode
      \end{itemize}
    \item Neighbor-wise Collective: \code{VecScatterBegin(), MatMult()}
      \begin{itemize}
      \item Point-to-point communication between two processes
      \item Neighbor collectives in upcoming MPI-3
      \end{itemize}
    \item Collective: \code{VecNorm(), MatAssemblyBegin(), KSPCreate()}
      \begin{itemize}
      \item Global communication, synchronous
      \item Non-blocking collectives in upcoming MPI-3
      \end{itemize}
    \end{itemize}
  \item Deadlock if some process doesn't participate (\eg wrong order)
  \end{itemize}
\end{frame}



\subsection{Options Database}
\begin{frame}[fragile]{Objects}
  % \begin{lstlisting}
  %   Mat A;
  %   PetscInt m,n,M,N;
  %   MatCreate(comm,&A);
  %   MatSetSizes(A,m,n,M,N);      /* or PETSC_DECIDE */ 
  %   MatSetOptionsPrefix(A,"foo_");
  %   MatSetFromOptions(A);
  %   /* Use A */
  %   MatView(A,PETSC_VIEWER_DRAW_WORLD);
  %   MatDestroy(A);
  % \end{lstlisting}
  \begin{minted}{c}
    Mat A;
    PetscInt m,n,M,N;
    MatCreate(comm,&A);
    MatSetSizes(A,m,n,M,N);      /* or PETSC_DECIDE */ 
    MatSetOptionsPrefix(A,"foo_");
    MatSetFromOptions(A);
    /* Use A */
    MatView(A,PETSC_VIEWER_DRAW_WORLD);
    MatDestroy(A);
  \end{minted}
  \begin{itemize}
  \item \code{Mat} is an opaque object (pointer to incomplete type)
    \oneitem{Assignment, comparison, etc, are cheap}
  \item What's up with this ``Options'' stuff?
    \begin{itemize}
    \item Allows the type to be determined at runtime: \code{-foo\_mat\_type sbaij}
    \item Inversion of Control similar to ``service locator'', \\
      related to ``dependency injection''
    \item Other options (performance and semantics) can be changed at
      runtime under \code{-foo\_mat\_}
    \end{itemize}
  \end{itemize}
\end{frame}

\begin{frame}{Basic {\kb PetscObject} Usage}

\vbox{Every object in PETSc supports a basic interface}

\begin{tabular}{|r|l|}
\hline
Function & Operation \\
\hline
{\kb Create()}               & create the object \\
{\kb Get/SetName()}          & name the object \\
{\kb Get/SetType()}          & set the implementation type \\
{\kb Get/SetOptionsPrefix()} & set the prefix for all options \\
{\kb SetFromOptions()}       & customize object from the command line \\
{\kb SetUp()}                & preform other initialization \\
{\kb View()}                 & view the object \\
{\kb Destroy()}              & cleanup object allocation \\
\hline
\end{tabular}

\vbox{Also, all objects support the {\kb -help} option.}

\end{frame}


\begin{frame}{Ways to set options}
  \begin{itemize}
  \item Command line
  \item Filename in the third argument of \code{PetscInitialize()}
  \item \code{$\sim$/.petscrc}
  \item \code{\$PWD/.petscrc}
  \item \code{\$PWD/petscrc}
  \item \code{PetscOptionsInsertFile()}
  \item \code{PetscOptionsInsertString()}
  \item \code{PETSC\_OPTIONS} environment variable
  \item command line option \code{-options\_file [file]}
  \end{itemize}
\end{frame}

\begin{frame}{Try it out}
  \shell{\small cd \$PETSC\_DIR/src/snes/examples/tutorials \&\& make ex5} \\
  \begin{itemize}
  \item \shell{./ex5 -da\_grid\_x 10 -da\_grid\_y 10 -par 6.7 \\
      -snes\_monitor -\{ksp,snes\}\_converged\_reason \\
      -snes\_view}
  \item \shell{./ex5 -da\_grid\_x 10 -da\_grid\_y 10 -par 6.7 \\
      -snes\_monitor -\{ksp,snes\}\_converged\_reason \\
      -snes\_view -mat\_view\_draw -draw\_pause 0.5}
  \item \shell{./ex5 -da\_grid\_x 10 -da\_grid\_y 10 -par 6.7 \\
      -snes\_monitor -\{ksp,snes\}\_converged\_reason \\
      -snes\_view -mat\_view\_draw -draw\_pause 0.5 \\
      -pc\_type lu -pc\_factor\_mat\_ordering\_type natural}
  \item Use \code{-help} to find other ordering types
\end{itemize}
\end{frame}

\begin{frame}{In parallel}
  \begin{itemize}
  \item \shell{mpiexec -n 4 \\
      ./ex5 -da\_grid\_x 10 -da\_grid\_y 10 -par 6.7 \\
      -snes\_monitor -\{ksp,snes\}\_converged\_reason \\
      -snes\_view -sub\_pc\_type lu}
  \item How does the performance change as you
    \begin{itemize}
    \item vary the number of processes (up to 32 or 64)?
    \item increase the problem size?
    \item use an inexact subdomain solve?
    \item try an overlapping method: \code{-pc\_type asm -pc\_asm\_overlap 2}
    \item simulate a big machine: \code{-pc\_asm\_blocks 512}
    \item change the Krylov method: \code{-ksp\_type ibcgs}
    \item use algebraic multigrid: \code{-pc\_type hypre}
    \item use smoothed aggregation multigrid: \code{-pc\_type ml}
    \end{itemize}
  \end{itemize}
\end{frame}


\section{Why Parallel?}
\begin{frame}{Why Parallel?}
  \begin{itemize}
  \item Solve a fixed problem faster
  \item Obtain a more accurate solution in the same amount of time
  \item Solve a more complicated problem in the same amount of time
  \item Use more memory than available on one machine
  \end{itemize}
\end{frame}

\begin{frame}{Strong Scaling}
  \begin{center}
    \includegraphics[width=.8\textwidth]{figures/olenz/olenz-time-np}
  \end{center}
  \begin{itemize}
  \item Good: shows absolute time
  \item Bad: log-log plot makes it difficult to discern efficiency
    \begin{itemize}
    \item Stunt 3: \url{http://blogs.fau.de/hager/archives/5835}
    \end{itemize}
  \end{itemize}
\end{frame}

\begin{frame}{Efficiency versus Number of Processes}
  \begin{center}
    \includegraphics[width=.8\textwidth]{figures/olenz/olenz-efficiency-np}
  \end{center}
  \begin{itemize}
  \item Good: shows efficiency at scale
  \item Bad: no absolute time
  \end{itemize}
\end{frame}

\begin{frame}{Efficiency versus Time}
  \begin{center}
    \includegraphics[width=.8\textwidth]{figures/olenz/olenz-efficiency-time}
  \end{center}
  \begin{itemize}
  \item Good: absolute time
  \item Good: efficiency (preferably with units, like DOF/s/process)
  \item Bad: harder to see machine size (but less important)
  \end{itemize}
\end{frame}

\begin{frame}{Scaling Challenges}
  \begin{quote} \centering
    The easiest way to make software scalable \\
    is to make it sequentially inefficient. \\
    (Gropp 1999)
  \end{quote}
  \begin{itemize}
  \item Solver iteration count may increase from
    \begin{itemize}
    \item increased resolution
    \item model parameters (e.g., coefficient contrast/structure)
    \item more realistic models (e.g., plasticity)
    \item model coupling
    \end{itemize}
  \item Algorithm may have suboptimal complexity (e.g., direct solver)
  \item Increasing spatial resolution requires more time steps (usually)
  \item Implementation/data structures may not scale
  \item Architectural effects -- cache, memory
  \end{itemize}
\end{frame}

\begin{frame}{Accuracy-time tradeoffs: \emph{de rigueur} in ODE community}
  \begin{center}
    \includegraphics[width=0.8\textwidth]{figures/HairerWanner-WorkPrecision.png}\\
    {\scriptsize [Hairer and Wanner (1999)]}
  \end{center}
  \begin{itemize}
  \item Tests discretization, adaptivity, algebraic solvers, implementation
  \item No reference to number of time steps, number of grid points, etc.
  \end{itemize}
\end{frame}


\section[Core I]{Core PETSc Components and Algorithms Primer}

\subsection{Nonlinear solvers: SNES}
\begin{frame}{Newton iteration: workhorse of SNES}
  \begin{textblock}{3}(11,0)
    \includegraphics[width=\textwidth]{figures/Newton}
  \end{textblock}
  \begin{itemize}
  \item Standard form of a nonlinear system
    \[ F(u) = 0 \]
  \item Iteration
    \begin{align*}
      \text{Solve:} & \qquad J(u) w = -F(u) \\
      \text{Update:} & \qquad u^+ \gets u + w
    \end{align*}
    \item Quadratically convergent near a root: $\abs{u^{n+1}-u^*} \in \bigO\Big(\abs{u^n-u^*}^2\Big)$
    \item Picard is the same operation with a different $J(u)$
  \end{itemize}
  \begin{example}[Nonlinear Poisson]
    \begin{align*}
      F(u)=0 \quad &\sim\quad -\div\big[ (1+u^2) \nabla u \big] - f = 0 \\
      J(u)w \quad &\sim\quad  -\div\big[(1+u^2)\nabla w + 2uw\nabla u \Big]
    \end{align*}
  \end{example}
  % \begin{example}[$\pp$-Bratu]
  %   Suppose $F$ is a discretization of
  %   \[ -\nabla \cdot \big( \eta \nabla u \big) - \lambda e^u - f = 0 \]
  %   \[\eta(\gamma) = (\epsilon^2+\gamma)^{\frac{\pfrak-2}{2}}, \qquad\quad \gamma = \half \abs{\nabla u}^2. \]
  %   Then $J(u)w$ is a discretization of
  %   \[ -\nabla \cdot \big( \eta \nabla w + \eta' (\nabla u \cdot \nabla w)\nabla u \big) - \lambda e^{u} w . \]
  % \end{example}
\end{frame}


\begin{frame}
\frametitle{SNES Paradigm}

The SNES interface is based upon callback functions
\begin{itemize}
  \item \code{FormFunction()}, set by \code{SNESSetFunction()}

  \medskip

  \item \code{FormJacobian()}, set by \code{SNESSetJacobian()}
\end{itemize}

\bigskip

  When PETSc needs to evaluate the nonlinear residual $F(x)$,
\begin{itemize}
  \item Solver calls the {\bf user's} function

  \medskip

  \item User function gets application state through the {\kb ctx} variable
  \begin{itemize}
    \item PETSc \emph{never} sees application data
  \end{itemize}
\end{itemize}
\end{frame}

\begin{frame}{SNES Function}

The user provided function which calculates the nonlinear residual has signature
\begin{center}
  {\small \mint{c}|PetscErrorCode (*func)(SNES snes,Vec x,Vec r,void *ctx)|}
\end{center}
\begin{itemize}
  \item[{\kb x}:] The current solution
  \item[{\kb r}:] The residual
  \item[{\kb ctx}:] The user context passed to {\kb SNESSetFunction()}
  \begin{itemize}
    \item Use this to pass application information, e.g. physical constants
  \end{itemize}
\end{itemize}

\end{frame}

\begin{frame}[fragile]{SNES Jacobian}
The user provided function which calculates the Jacobian has signature
\begin{minted}{c}
PetscErrorCode (*func)(SNES snes,Vec x,Mat *J,Mat *M,
                       MatStructure *flag,void *ctx)
\end{minted}

\begin{itemize}
  \item[{\kb x}:] The current solution
  \item[{\kb J}:] The Jacobian
  \item[{\kb M}:] The Jacobian preconditioning matrix (possibly J itself)
  \item[{\kb ctx}:] The user context passed to {\kb SNESSetFunction()}
  \begin{itemize}
    \item Use this to pass application information, e.g. physical constants
  \end{itemize}

  \item Possible {\kb MatStructure} values are:
  \begin{itemize}
    \item SAME\_NONZERO\_PATTERN
    \item DIFFERENT\_NONZERO\_PATTERN
  \end{itemize}
\end{itemize}

Alternatively, you can use
\begin{itemize}
  \item a builtin sparse finite difference approximation (``coloring'')
  \item automatic differentiation (ADIC/ADIFOR)
\end{itemize}

\end{frame}


\subsection{Linear Algebra background/theory}
\begin{frame}{Matrices}
  \begin{definition}<1->[Matrix]
    A \alert{matrix} is a linear transformation between finite dimensional vector spaces.
  \end{definition}
  \begin{definition}<2->[Forming a matrix]
    \alert{Forming} or \alert{assembling} a matrix means defining it's action in terms of entries (usually stored in a sparse format).
  \end{definition}
\end{frame}

\begin{frame}{Important matrices}
  \begin{enumerate}
  \item Sparse (e.g.~discretization of a PDE operator)
  \item \alert<2,4>{Inverse of \emph{anything} interesting $B = A^{-1}$}
  \item \alert<4>{Jacobian of a nonlinear function $J y = \lim_{\epsilon \to 0} \frac{F(x + \epsilon y) - F(x)}{\epsilon}$}
  \item \alert<2,4>{Fourier transform $\mathcal{F},\mathcal{F}^{-1}$}
  \item \alert<2,4>{Other fast transforms, e.g. Fast Multipole Method}
  \item \alert<2,4>{Low rank correction $B = A + u v^T$}
  \item \alert<2,4>{Schur complement $S = D - C A^{-1} B$}
  \item \alert<3,4>{Tensor product $A = \sum_e A_x^e \otimes A_y^e \otimes A_z^e$}
  \item \alert<3,4>{Linearization of a few steps of an explicit integrator}
  \end{enumerate}
  \begin{columns}\begin{column}{0.3\textwidth}\end{column}\begin{column}{0.7\textwidth}
  \begin{itemize}
  \item<only@2> These matrices are \alert<2>{dense}.  Never form them.
  \item<only@3>{Thes are \alert<3>{not very sparse}.}
    Don't form them.
  \item<only@4> {None of these matrices ``have entries''}
  \end{itemize}
\end{column}
\end{columns}
\end{frame}

\begin{frame}{What can we do with a matrix that doesn't have entries?}
  \begin{block}{Krylov solvers for $A x = b$}
    \begin{itemize}
    \item Krylov subspace: $\{b, Ab, A^2b, A^3b, \dotsc\}$
    \item Convergence rate depends on the spectral properties of the matrix
      \begin{itemize}
      \item Existance of small polynomials $p_n(A) < \epsilon$ where $p_n(0) = 1$.
      \item condition number $\kappa(A) = \norm{A} \norm{A^{-1}} = \sigma_{\text{max}}/\sigma_{\text{min}}$
      \item distribution of singular values, spectrum $\Lambda$, pseudospectrum $\Lambda_\epsilon$
%      \item $\epsilon$-pseudospectrum $\Lambda_\epsilon$, spectrum of $A + E$ where $\norm{E} < \epsilon$
      \end{itemize}
    \item For any popular Krylov method $\mathcal{K}$, there is a matrix
      of size $m$, such that $\mathcal{K}$ outperforms all other methods
      by a factor at least $\bigO(\sqrt{m})$~[Nachtigal et. al., 1992]%\cite{nachtigal1992fnm}
    \end{itemize}
  \end{block}
  \begin{block}{Typically...}
    \begin{itemize}
    \item The action $y \gets A x$ can be computed in $\bigO(m)$
    \item Aside from matrix multiply, the $n^{\text{th}}$ iteration requires at most $\bigO(mn)$
    \end{itemize}
  \end{block}
\end{frame}

\begin{frame}{GMRES}
  Brute force minimization of residual in $\{b,Ab,A^2b,\dotsc\}$
  \begin{enumerate}
  \item Use Arnoldi to orthogonalize the $n$th subspace, producing
    \[ A Q_n = Q_{n+1} H_n \]
  \item Minimize residual in this space by solving the overdetermined system
    \[ H_n y_n = e_1^{(n+1)} \]
    using $QR$-decomposition, updated cheaply at each iteration.
  \end{enumerate}
  Properties
  \begin{itemize}
  \item Converges in $n$ steps for all right hand sides if there exists a polynomial of degree $n$
    such that $\norm{p_n(A)} < \textit{tol}$ and $p_n(0)=1$.
  \item Residual is monotonically decreasing, robust in practice
  \item Restarted variants are used to bound memory requirements
  \end{itemize}
\end{frame}

\subsection{Preconditioning}
\begin{frame}{Preconditioning}
  \begin{block}{Idea: improve the conditioning of the Krylov operator}
    \begin{itemize}
    \item Left preconditioning
      \vspace{-1em}
      \begin{gather*}
        (P^{-1} A) x = P^{-1} b \\
        \{ P^{-1} b, (P^{-1}A) P^{-1} b, (P^{-1}A)^2 P^{-1} b, \dotsc \}
      \end{gather*}
    \item Right preconditioning
      \vspace{-1em}
      \begin{gather*}
        (A P^{-1}) P x = b \\
        \{ b, (A P^{-1}b, (A P^{-1})^2b, \dotsc \}
      \end{gather*}
    \item The product $P^{-1}A$ or $A P^{-1}$ is \emph{not} formed.
    \end{itemize}
  \end{block}
  \begin{definition}[Preconditioner]
      A \emph{preconditioner} $\PP$ is a method for constructing a
matrix (just a linear function, not assembled!)  $P^{-1} = \PP(A,A_p)$
using a matrix $A$ and extra information $A_p$, such that the spectrum
of $P^{-1}A$ (or $A P^{-1}$) is well-behaved.
    \end{definition}
\end{frame}

\begin{frame}{Preconditioning}
  \begin{definition}[Preconditioner]
      A \emph{preconditioner} $\PP$ is a method for constructing a matrix
      $P^{-1} = \PP(A,A_p)$ using a matrix $A$ and extra information $A_p$, such that
      the spectrum of $P^{-1}A$ (or $A P^{-1}$) is well-behaved.
    \end{definition}
    \begin{itemize}
    \item $P^{-1}$ is dense, $P$ is often not available and is not needed
    \item $A$ is rarely used by $\PP$, but $A_p = A$ is common
    \item $A_p$ is often a sparse matrix, the ``preconditioning matrix''
    \item Matrix-based: Jacobi, Gauss-Seidel, SOR, ILU(k), LU
    \item Parallel: Block-Jacobi, Schwarz, Multigrid, FETI-DP, BDDC
    \item Indefinite: Schur-complement, Domain Decomposition, Multigrid
    \end{itemize}
\end{frame}


\begin{frame}{Questions to ask when you see a matrix}
  \begin{enumerate}
  \item What do you want to do with it?
    \begin{itemize}
    \item Multiply with a vector
    \item Solve linear systems or eigen-problems
    \end{itemize}
  \item How is the conditioning/spectrum?
    \begin{itemize}
    \item distinct/clustered eigen/singular values?
    \item symmetric positive definite ($\sigma(A) \subset \R^+$)?
    \item nonsymmetric definite ($\sigma(A) \subset \{z \in \C : \Re [z] > 0 \}$)?
    \item indefinite?
    \end{itemize}
  \item How dense is it?
    \begin{itemize}
    \item block/banded diagonal?
    \item sparse unstructured?
    \item denser than we'd like?
    \end{itemize}
  \item Is there a better way to compute $Ax$?
  \item Is there a different matrix with similar spectrum, but nicer properties?
  \item \alert<2>{How can we precondition $A$?}
  \end{enumerate}
\end{frame}

\begin{frame}{Relaxation}
  Split into lower, diagonal, upper parts: \alert{$ A = L + D + U $}
  \begin{block}{Jacobi}
    Cheapest preconditioner: $P^{-1} = D^{-1}$
  \end{block}
  \begin{block}{Successive over-relaxation (SOR)}
    \begin{gather*}
      \left(L + \frac 1 \omega D\right) x_{n+1} = \left[\left(\frac
          1\omega-1\right)D - U\right] x_n + \omega b \\
      P^{-1} = \text{$k$ iterations starting with $x_0=0$}
    \end{gather*}
    \begin{itemize}
    \item Implemented as a sweep
    \item $\omega = 1$ corresponds to Gauss-Seidel
    \item Very effective at removing high-frequency components of residual
    \end{itemize}
  \end{block}
\end{frame}

\begin{frame}[shrink=5]{Factorization}
  Two phases
  \begin{itemize}
  \item symbolic factorization: find where fill occurs, only uses sparsity pattern
  \item numeric factorization: compute factors
  \end{itemize}
  \begin{block}{LU decomposition}
    \begin{itemize}
    \item Ultimate preconditioner
    \item Expensive, for $m\times m$ sparse matrix with bandwidth $b$, traditionally requires $\bigO(mb^2)$ time and $\bigO(mb)$ space.
      \begin{itemize}
      \item Bandwidth scales as $m^{\frac{d-1}{d}}$ in $d$-dimensions
      \item Optimal in 2D: $\bigO(m \log m)$ space, $\bigO(m^{3/2})$ time
      \item Optimal in 3D: $\bigO(m^{4/3})$ space, $\bigO(m^2)$ time
      \end{itemize}
    \item Symbolic factorization is problematic in parallel
    \end{itemize}
  \end{block}
  \begin{block}{Incomplete LU}
    \begin{itemize}
    \item Allow a limited number of levels of fill:
      ILU($k$)
    \item Only allow fill for entries that exceed threshold: ILUT
    \item Very poor scaling in parallel, don't bother beyond 8 PEs.
    \item No guarantees
    \end{itemize}
  \end{block}
\end{frame}

\begin{frame}{1-level Domain decomposition}
  Domain size $L$, subdomain size $H$, element size $h$
  \begin{block}{Overlapping/Schwarz}
    \begin{itemize}\item Solve Dirichlet problems on overlapping
      subdomains
    \item No overlap: $\textit{its} \in \bigO\big( \frac{L}{\sqrt{Hh}} \big)$
    \item Overlap $\delta$: $\textit{its} \in \big( \frac L {\sqrt{H\delta}} \big)$
    \end{itemize}
  \end{block}
  \begin{block}{Neumann-Neumann}
    \begin{itemize}
    \item Solve Neumann problems on non-overlapping subdomains
    \item $\textit{its} \in \bigO\big( \frac{L}{H}(1+\log\frac H h) \big)$
    \item Tricky null space issues (floating subdomains)
    \end{itemize}
  \end{block}
  \begin{itemize}
  \item Multilevel variants knock off the leading $\frac L H$
  \item Both overlapping and nonoverlapping with this bound
  \end{itemize}
  % \begin{block}{BDDC and FETI-DP}
  %   \begin{itemize}
  %   \item Neumann problems on subdomains with
  %     coarse grid correction
  %   \item $\textit{its} \in \bigO\big(1 + \log\frac H h \big)$
  %   \end{itemize}
  %   \includegraphics[width=0.7\textwidth]{bddc}
  % \end{block}
\end{frame}

\begin{frame}[shrink=5]{Multigrid}
  \begin{block}{Hierarchy: Interpolation and restriction operators}
    \begin{equation*}
    \II^\uparrow : X_{\text{coarse}} \to X_{\text{fine}} \qquad
    \II^\downarrow :  X_{\text{fine}} \to X_{\text{coarse}}
  \end{equation*}
  \end{block}
  \begin{itemize}
  \item Geometric: define problem on multiple levels, use grid to compute hierarchy
  \item Algebraic: define problem only on finest level, use matrix structure to build hierarchy
  \end{itemize}
  \begin{block}{Galerkin approximation}
    Assemble this matrix: $A_{\text{coarse}} = \II^\downarrow A_{\text{fine}} \II^\uparrow$
  \end{block}
  \begin{block}{Application of multigrid preconditioner ($V$-cycle)}
    \begin{itemize}
    \item Apply pre-smoother on fine level (any preconditioner)
    \item Restrict residual to coarse level with $\II^\downarrow$
    \item Solve on coarse level $A_{\text{coarse}} x = r$
    \item Interpolate result back to fine level with $\II^\uparrow$
    \item Apply post-smoother on fine level (any preconditioner)
    \end{itemize}
  \end{block}
\end{frame}

\begin{frame}{Multigrid convergence properties}
  \begin{itemize}
  \item Textbook: $P^{-1}A$ is spectrally equivalent to identity
  \item Most theory applies to SPD systems
  \item nonsymmetric (e.g. advection, shallow water, Euler) \\
    with low-order upwind discretization
  \item Good when coefficients in problem are smooth
    \begin{itemize}
    \item large jumps and anisotropy are harder
    \item build low-energy interpolants
    \item use stronger smoothers
    \end{itemize}
  \item Aggressive coarsening is critical, especially in parallel
  \item Most theory uses SOR smoothers, ILU often more robust
  \item Coarsest component usually solved semi-redundantly with direct solver
  \item Multilevel Schwarz is an extreme case of aggressive coarsening
    and strong smoothers.  Exotic interpolants for robustness.
  \end{itemize}
\end{frame}

\begin{frame}{Norms}
  \begin{itemize}
  \item Krylov subspace: $\{P^{-1}b,(P^{-1}A)P^{-1}b,(P^{-1}A)^2P^{-1}b,\dotsc\}$
  \item Subspace needs to contain the solution
    \begin{itemize}
    \item Diameter of preconditioned connectivity graph
    \end{itemize}
  \item Need to find the correct linear combination
    \begin{itemize}
    \item Optimize unpreconditioned residual norm (usually right preconditioning)
      \begin{equation*}
        \norm{A x - b}_2 = \norm{A (x - x^*)}_2 = \norm{x-x^*}_{A^T A}
      \end{equation*}
    \item Optimize preconditioned residual norm (usually left preconditioning)
      \begin{equation*}
        \norm{P^{-1} (A x - b)}_2 = \norm{P^{-1}A (x - x^*)}_2 = \norm{x-x^*}_{A^T P^{-T} P^{-1} A}
      \end{equation*}
    \item Natural norm (conjugate gradients) minimizes $\norm{x-x^*}_{P^{-1/2}AP^{-1/2}}$
    \end{itemize}
  \item Evaluating convergence
    \begin{itemize}
    \item Preconditioned, unpreconditioned, or natural norm
    \item Which one to trust?
    \item \code{-ksp\_monitor\_true\_residual}, \code{-ksp\_norm\_type}
    \end{itemize}
  \end{itemize}
\end{frame}


\subsection{Profiling}
\begin{frame}{Profiling}

\begin{itemize}
  \item Use {\kb -log\_summary} for a performance profile
  \begin{itemize}
    \item Event timing
    \item Event flops
    \item Memory usage
    \item MPI messages
  \end{itemize}

  \item Call {\kb PetscLogStagePush()} and {\kb PetscLogStagePop()}
  \begin{itemize}
    \item User can add new stages
  \end{itemize}

  \item Call {\kb PetscLogEventBegin()} and {\kb PetscLogEventEnd()}
  \begin{itemize}
    \item User can add new events
  \end{itemize}

  \item Call {\kb PetscLogFlops()} to include your flops
\end{itemize}

\end{frame}

\begin{frame}[fragile]{Reading \code{-log\_summary}}
\begin{itemize}
\item
{\scriptsize
\begin{verbatim}
                         Max       Max/Min        Avg      Total 
Time (sec):           1.548e+02      1.00122   1.547e+02
Objects:              1.028e+03      1.00000   1.028e+03
Flops:                1.519e+10      1.01953   1.505e+10  1.204e+11
Flops/sec:            9.814e+07      1.01829   9.727e+07  7.782e+08
MPI Messages:         8.854e+03      1.00556   8.819e+03  7.055e+04
MPI Message Lengths:  1.936e+08      1.00950   2.185e+04  1.541e+09
MPI Reductions:       2.799e+03      1.00000
\end{verbatim}}
\item Also a summary per stage
\item Memory usage per stage (based on when it was allocated)
\item Time, messages, reductions, balance, flops per event per stage
\item Always send \code{-log\_summary} when asking \\
  performance questions on mailing list
\end{itemize}
\end{frame}

\begin{frame}{Communication Costs}
  \begin{itemize}
  \item Reductions: usually part of Krylov method, latency limited
    \begin{itemize}
    \item \code{VecDot}
    \item \code{VecMDot}
    \item \code{VecNorm}
    \item \code{MatAssemblyBegin}
    \item Change algorithm (e.g. IBCGS)
    \end{itemize}
  \item Point-to-point (nearest neighbor), latency or bandwidth
    \begin{itemize}
    \item \code{VecScatter}
    \item \code{MatMult}
    \item \code{PCApply}
    \item \code{MatAssembly}
    \item \code{SNESFunctionEval}
    \item \code{SNESJacobianEval}
    \item Compute subdomain boundary fluxes redundantly
    \item Ghost exchange for all fields at once
    \item Better partition
    \end{itemize}
  \end{itemize}
\end{frame}

\begin{frame}{HPGMG-FE \quad \url{https://hpgmg.org}}
  \includegraphics[width=\textwidth]{figures/MG/titan-edison-supermuc-range.png}
\end{frame}


\subsection{Matrix Redux}
\begin{frame}{Matrices, redux}
What are PETSc matrices?
\begin{itemize}
\item Linear operators on finite dimensional vector spaces. (snarky)
  \item<2> Fundamental objects for storing stiffness matrices and Jacobians
  \item<2> Each process locally owns a contiguous set of rows
  \item<2> Supports many data types
  \begin{itemize}
    \item AIJ, Block AIJ, Symmetric AIJ, Block Diagonal, etc.
  \end{itemize}
  \item<2> Supports structures for many packages
  \begin{itemize}
    \item MUMPS, Spooles, SuperLU, UMFPack, Hypre
  \end{itemize}
\end{itemize}
\end{frame}

\begin{frame}{How do I create matrices?}

\begin{itemize}
  \item {\kb MatCreate(MPI\_Comm, Mat *)}
  \item {\kb MatSetSizes(Mat, int m, int n, int M, int N)}
  \item {\kb MatSetType(Mat, MatType typeName)}
  \item {\kb MatSetFromOptions(Mat)}
  \begin{itemize}
    \item Can set the type at runtime
  \end{itemize}
  \item {\kb MatMPIBAIJSetPreallocation(Mat,...)} %(Mat,int bs,int d_nz,const int
                                %d_nnz[],int o_nz,const int o_nnz[])}
    \oneitem{important for assembly performance, more tomorrow}
  \item {\kb MatSetBlockSize(Mat, int bs)}
    \oneitem{for vector problems}
  \item {\kb MatSetValues(Mat,...)}
  \begin{itemize}
    \item {\bf MUST} be used, but does automatic communication
    \item \cfunc|MatSetValuesLocal|, \cfunc|MatSetValuesStencil|
    \item \cfunc|MatSetValuesBlocked|
  \end{itemize}
\end{itemize}
\end{frame}

\begin{frame}{Matrix Polymorphism}

The PETSc {\kb Mat} has a single user interface,
\begin{itemize}
  \item Matrix assembly
  \begin{itemize}
    \item {\kb MatSetValues()}
  \end{itemize}

  \item Matrix-vector multiplication
  \begin{itemize}
    \item {\kb MatMult()}
  \end{itemize}

  \item Matrix viewing
  \begin{itemize}
    \item {\kb MatView()}
  \end{itemize}
\end{itemize}
but multiple underlying implementations.
\begin{itemize}
  \item AIJ, Block AIJ, Symmetric Block AIJ,
  \item Dense
  \item Matrix-Free
  \item etc.
\end{itemize}
A matrix is defined by its {\red{interface}}, not by its {\blue{data structure}}.

\end{frame}

\begin{frame}{Matrix Assembly}

\begin{itemize}
  \item A three step process
  \begin{itemize}
    \item Each process sets or adds values
    \item Begin communication to send values to the correct process
    \item Complete the communication
  \end{itemize}
  \item {\kb MatSetValues(Mat A, m, rows[], n, cols[], values[], mode)}
  \begin{itemize}
    \item {\kb mode} is either INSERT\_VALUES or ADD\_VALUES
    \item Logically dense block of values
  \end{itemize}
% TODO picture of MatSetValues
  \item Two phase assembly allows overlap of communication and computation
  \begin{itemize}
    \item {\kb MatAssemblyBegin(Mat m, type)}
    \item {\kb MatAssemblyEnd(Mat m, type)}
    \item {\kb type} is either MAT\_FLUSH\_ASSEMBLY or MAT\_FINAL\_ASSEMBLY
  \end{itemize}
  \item<2-> For vector problems\\
    {\kb MatSetValuesBlocked(Mat A, m, rows[], \\
      \qquad\qquad n, cols[], values[], mode)}
  \item<2-> The same assembly code can build matrices of different format
    \begin{itemize}
    \item choose format at run-time.
    \end{itemize}
\end{itemize}

\end{frame}

\input{slides/PETSc/Integration/EfficientMatrixAssembly.tex}
\begin{frame}{Why Are PETSc Matrices That Way?}

\begin{itemize}
  \item No one data structure is appropriate for all problems
  \begin{itemize}
    \item Blocked and diagonal formats provide significant performance benefits
    \item PETSc has many formats and makes it easy to add new data structures
  \end{itemize}

  \item Assembly is difficult enough without worrying about partitioning
  \begin{itemize}
    \item PETSc provides parallel assembly routines
    \item Achieving high performance still requires making most operations local
    \item However, programs can be incrementally developed.
    \item {\kb MatPartitioning} and {\kb MatOrdering} can help
  \end{itemize}

  \item Matrix decomposition in contiguous chunks is simple
  \begin{itemize}
    \item Makes interoperation with other codes easier
    \item For other ordering, PETSc provides ``Application Orderings'' (AO)
  \end{itemize}
\end{itemize}

\end{frame}

\begin{frame}{Approximating condition numbers}

\begin{itemize}
\item Small matrices: \\
  {\kb -pc\_type svd -pc\_svd\_monitor}
\item Large matrices (avoid restarts!): \\
  {\kb -pc\_type none -ksp\_type gmres -ksp\_monitor\_singular\_value \bslash\\
    \qquad-ksp\_gmres\_restart 1000}
\item Condition of preconditioned operator: \\
  {\kb -pc\_type some\_pc -ksp\_type gmres \bslash \\
    \qquad -ksp\_monitor\_singular\_value \bslash \\
    \qquad -ksp\_gmres\_restart 1000}
\end{itemize}

Try these:

\begin{itemize}
\item \shell{cd \$PETSC\_DIR/src/ksp/ksp/examples/tutorials \bslash \\ \&\& make ex2}
\item \shell{./ex2 -m 20 -n 20 <other\_options>}
\end{itemize}

\end{frame}


\section*{Preliminary Conclusions}
\begin{frame}{Preliminary Conclusions}
  \begin{block}{PETSc can help you}
    \begin{itemize}
    \item solve algebraic and DAE problems in your application area
    \item rapidly develop efficient parallel code, can start from examples
    \item develop new solution methods and data structures
    \item debug and analyze performance
    \item advice on software design, solution algorithms, and performance
      \begin{itemize}
        \item Public questions: \url{petsc-users@mcs.anl.gov}, archived
        \item Private questions: \url{petsc-maint@mcs.anl.gov}, not archived
      \end{itemize}
    \end{itemize}
  \end{block}
  \begin{block}{You can help PETSc}
    \begin{itemize}
    \item report bugs and inconsistencies, or if you think there is a better way
    \item tell us if the documentation is inconsistent or unclear
    \item consider developing new algebraic methods as plugins, contribute if your idea works
    \end{itemize}
  \end{block}
\end{frame}


\part{Integration and Efficiency}

\section{Application Integration}
\begin{frame}{Application Integration}

\begin{itemize}
  \item Be willing to experiment with algorithms
  \begin{itemize}
    \item No optimality without interplay between physics and algorithmics
  \end{itemize}

  \item Adopt flexible, extensible programming
  \begin{itemize}
    \item Algorithms and data structures not hardwired
  \end{itemize}

  \item Be willing to play with the real code
  \begin{itemize}
    \item Toy models have limited usefulness
    \item But make test cases that run quickly
  \end{itemize}

  \item If possible, profile before integration
  \begin{itemize}
    \item Automatic in PETSc
  \end{itemize}
\end{itemize}

\end{frame}

\begin{frame}{Incorporating PETSc into existing codes}
  \begin{itemize}
  \item PETSc does not seize \code{main()}, does not control output
  \item Propogates errors from underlying packages, flexible error handling
  \item Nothing special about \code{MPI\_COMM\_WORLD}
  \item Can wrap existing data structures/algorithms
    \begin{itemize}
    \item \code{MatShell}, \code{PCShell}, full implementations
    \item \code{VecCreateMPIWithArray()}
    \item \code{MatCreateSeqAIJWithArrays()}
    \item Use an existing semi-implicit solver as a preconditioner
    \item Usually worthwhile to use native PETSc data structures \\
      unless you have a good reason not to
    \end{itemize}
  \item Uniform interfaces across languages: C, C++, Fortran 77/90, Python
  \item Do not have to use high level interfaces
    \begin{itemize}
    \item but PETSc can offer more if you do, like MFFD and SNES Test
    \end{itemize}
  \end{itemize}
\end{frame}

\begin{frame}{Integration Stages}

\begin{itemize}
  % I like the simile of VC as a will. It cheap, easy, someone will help you,
  % and if you die you are an idiot for not using one
  \item \red{Version Control}
  \begin{itemize}
    \item It is impossible to overemphasize
  \end{itemize}

  \item Initialization
  \begin{itemize}
    \item Linking to PETSc
  \end{itemize}

  \item Profiling
  \begin{itemize}
    \item Profile \red{before} changing
    \item Also incorporate command line processing
  \end{itemize}

  \item Linear Algebra
  \begin{itemize}
    \item First PETSc data structures
  \end{itemize}

  \item Solvers
  \begin{itemize}
    \item Very easy after linear algebra is integrated
  \end{itemize}
\end{itemize}

\end{frame}

\begin{frame}{Initialization}

\begin{itemize}
  \item Call {\kb PetscInitialize()}
  \begin{itemize}
    \item Setup static data and services
    \item Setup MPI if it is not already
    \item Can set \code{PETSC\_COMM\_WORLD} to use your communicator \\
      (can always use subcommunicators for each object)
  \end{itemize}

  \item Call {\kb PetscFinalize()}
  \begin{itemize}
    \item Calculates logging summary
    \item Can check for leaks/unused options
    \item Shutdown and release resources
  \end{itemize}

  \item Can only initialize PETSc once
\end{itemize}

\end{frame}

\begin{frame}{Matrix Memory Preallocation}
\begin{itemize}
  \item PETSc sparse matrices are dynamic data structures
  \begin{itemize}
    \item can add additional nonzeros freely
  \end{itemize}

  \item Dynamically adding many nonzeros
  \begin{itemize}
    \item requires additional memory allocations
    \item requires copies
    \item can kill performance
  \end{itemize}

  \item Memory preallocation provides
  \begin{itemize}
    \item the freedom of dynamic data structures
    \item good performance
  \end{itemize}

  \item Easiest solution is to replicate the assembly code
  \begin{itemize}
    \item Remove computation, but preserve the indexing code
    \item Store set of columns for each row
  \end{itemize}

  \item Call preallocation routines for all datatypes
  \begin{itemize}
    \item {\kb MatSeqAIJSetPreallocation()}
    \item {\kb MatMPIBAIJSetPreallocation()}
    \item Only the relevant data will be used.  Or {\kb MatXAIJSetPreallocation()}
  \end{itemize}
\end{itemize}
\end{frame}

\begin{frame}{Sequential Sparse Matrices}
{\kb MatSeqAIJSetPreallocation(Mat A, int nz, int nnz[])}
\hbox{\qquad
\vbox{
\begin{itemize}
  \item[nz:] expected number of nonzeros in any row
  \item[{nnz[i]}:] expected number of nonzeros in row i
\end{itemize}
}
}

\begin{center}
%\includegraphics[width=2in]{figures/Mat/serialSparseMatrix_bcsstk32}
\includegraphics[width=.5\textwidth]{figures/EllipRCMSquare}
\end{center}
\end{frame}

\begin{frame}{Parallel Sparse Matrix}
\begin{itemize}
  \item Each process locally owns a submatrix of contiguous global rows
  \item Each submatrix consists of diagonal and off-diagonal parts
\end{itemize}

\begin{center}
\includegraphics[width=3.5in]{figures/Mat/parallelSparseMatrix}
\end{center}

\begin{itemize}
  \item {\kb MatGetOwnershipRange(Mat A,int *start,int *end)}
  \begin{itemize}
    \item[{\kb start}:] first locally owned row of global matrix
    \item[{\kb end-1}:] last locally owned row of global matrix
  \end{itemize}
\end{itemize}
\end{frame}

\begin{frame}{Parallel Sparse Matrices}
\hbox{
\quad
\vbox{
{\kb MatMPIAIJSetPreallocation(Mat A, int dnz, int dnnz[], \\
  \qquad \qquad int onz, int onnz[])}
\begin{itemize}
  \item[dnz:] expected number of nonzeros in any row in the diagonal block
  \item[{dnnz[i]}:] expected number of nonzeros in row i in the diagonal block
  \item[onz:] expected number of nonzeros in any row in the offdiagonal portion
  \item[{onnz[i]}:] expected number of nonzeros in row i in the offdiagonal portion
\end{itemize}
}
}
\end{frame}

\begin{frame}{Verifying Preallocation}
\begin{itemize}
  \item Use runtime options \\
    {\kb -mat\_new\_nonzero\_location\_err} \\
    {\kb -mat\_new\_nonzero\_allocation\_err}
  \item Use {\kb -ksp\_view} or {\kb -snes\_view} and look for \\
    {\kb total number of mallocs used during MatSetValues calls =0}
  \item Use runtime option {\kb -info}
  \item Output: \\
{\kb
  $[$proc \#$]$ Matrix size: \%d X \%d; storage space: \%d unneeded, \%d used \\
  $[$proc \#$]$ Number of mallocs during MatSetValues( )  is \%d
}
\end{itemize}

\bigskip

\begin{center}
\includegraphics[width=5in]{figures/PETSc/logInfoOutput}
\end{center}
\end{frame}

\begin{frame}{Block and symmetric formats}
  \begin{itemize}
  \item BAIJ
    \begin{itemize}
    \item Like AIJ, but uses static block size
    \item Preallocation is like AIJ, but just one index per block
    \end{itemize}
  \item SBAIJ
    \begin{itemize}
    \item Only stores upper triangular part
    \item Preallocation needs number of nonzeros in upper triangular \\
      parts of on- and off-diagonal blocks
    \end{itemize}
  \item \code{MatSetValuesBlocked()}
    \begin{itemize}
    \item Better performance with blocked formats
    \item Also works with scalar formats, if \code{MatSetBlockSize()} was called
    \item Variants \code{MatSetValuesBlockedLocal()}, \code{MatSetValuesBlockedStencil()}
    \item Change matrix format at runtime, don't need to touch assembly code
    \end{itemize}
  \end{itemize}
\end{frame}

\begin{frame}{Linear Solvers}{Krylov Methods}

\begin{itemize}
  \item Using PETSc linear algebra, just add:
  \begin{itemize}
    \item {\kb KSPSetOperators(KSP ksp, Mat A, Mat M, MatStructure flag)}
    \item {\kb KSPSolve(KSP ksp, Vec b, Vec x)}
  \end{itemize}

  \item Can access subobjects
  \begin{itemize}
    \item {\kb KSPGetPC(KSP ksp, PC *pc)}
  \end{itemize}

  \item Preconditioners must obey PETSc interface
  \begin{itemize}
    \item Basically just the KSP interface
  \end{itemize}

  \item Can change solver dynamically from the command line, {\kb -ksp\_type}
\end{itemize}

\end{frame}

\begin{frame}{Nonlinear Solvers}{Newton and Picard Methods}

\begin{itemize}
  \item Using PETSc linear algebra, just add:
  \begin{itemize}
    \item {\kb SNESSetFunction(SNES snes, Vec r, residualFunc, void *ctx)}
    \item {\kb SNESSetJacobian(SNES snes, Mat A, Mat M, jacFunc, void *ctx)}
    \item {\kb SNESSolve(SNES snes, Vec b, Vec x)}
  \end{itemize}

  \item Can access subobjects
  \begin{itemize}
    \item {\kb SNESGetKSP(SNES snes, KSP *ksp)}
  \end{itemize}

  \item Can customize subobjects from the cmd line
  \begin{itemize}
    \item Set the subdomain preconditioner to ILU with {\kb -sub\_pc\_type ilu} 
  \end{itemize}
\end{itemize}

\end{frame}


\section{Representative examples and algorithms}
\subsection{Hydrostatic Ice}
\begin{frame}{Hydrostatic equations for ice sheet flow}
  \begin{itemize}
  \item Valid in the limit $w_x \ll u_z$, independent of basal friction
  \item Eliminate $p$ and $w$ by incompressibility:\\
    \quad 3D elliptic system for $\bm u = (u,v)$
    \begin{align*}
      - \nabla\cdot \left[ \eta
        \begin{pmatrix}
          4 u_x + 2 v_y & u_y + v_x & u_z \\
          u_y + v_x & 2 u_x + 4 v_y & v_z
        \end{pmatrix} \right] + \rho g \nabla s & = 0
    \end{align*}
    \begin{align*}
      \eta(\gamma) &= \frac B 2 (\epsilon^2 + \gamma)^{\frac{1-\mathfrak n}{2\mathfrak n}}, \qquad \mathfrak n \approx 3 \\
      \gamma &= u_x^2 + v_y^2 + u_xv_y + \frac 1 4 (u_y+v_x)^2 + \frac 1 4 u_z^2 + \frac 1 4 v_z^2
    \end{align*}
    and slip boundary $\sigma \cdot \bm n = \beta^2 \bm u$ where
    \begin{align*}
      \beta^2(\gamma_b) &= \beta_0^2 (\epsilon_b^2 + \gamma_b)^{\frac{\mathfrak m-1}{2}}, \qquad 0 < \mathfrak m \le 1 \\
      \gamma_b &= \frac 1 2 (u^2 + v^2)
    \end{align*}
  \item $Q_1$ FEM: \code{src/snes/examples/tutorials/ex48.c}
  \end{itemize}
\end{frame}

\input{slides/THIX5kmClip.tex}

\begin{frame}{Some Multigrid Options}
  \begin{itemize}
  \item \code{-dmmg\_grid\_sequencce}: [FALSE] \\
    Solve nonlinear problems on coarse grids to get initial guess
  \item \code{-pc\_mg\_galerkin}: [FALSE] \\
    Use Galerkin process to compute coarser operators
  \item \code{-pc\_mg\_type}: [FULL] \\
    (choose one of) MULTIPLICATIVE ADDITIVE FULL KASKADE
  \item \code{-mg\_coarse\_\{ksp,pc\}\_*} \\
    control the coarse-level solver
  \item \code{-mg\_levels\_\{ksp,pc\}\_*} \\
    control the smoothers on levels
  \item \code{-mg\_levels\_3\_\{ksp,pc\}\_*} \\
    control the smoother on specific level
  \item These also work with ML's algebraic multigrid.
  \end{itemize}
\end{frame}

\begin{frame}{What is this doing?}
\begin{itemize}
\item
\begin{alltt}\footnotesize
mpiexec -n 4 ./ex48
-M 16
-P 2
-da\_refine\_hierarchy\_x 1,8,8 \\
-da\_refine\_hierarchy\_y 2,1,1
-da\_refine\_hierarchy\_z 2,1,1 \\
-dmmg\_grid\_sequence 1
-dmmg\_view
-log\_summary \\
-ksp\_converged\_reason
-ksp\_gmres\_modifiedgramschmidt \\
-ksp\_monitor
-ksp\_rtol 1e-2 \\
-pc\_mg\_type multiplicative \\
-mg\_coarse\_pc\_type lu
-mg\_levels\_0\_pc\_type lu \\
-mg\_coarse\_pc\_factor\_mat\_solver\_package mumps \\
-mg\_levels\_0\_pc\_factor\_mat\_solver\_package mumps \\
-mg\_levels\_1\_sub\_pc\_type cholesky \\
-snes\_converged\_reason
-snes\_monitor
-snes\_stol 1e-12 \\
-thi\_L 80e3
-thi\_alpha 0.05
-thi\_friction\_m 0.3 \\
-thi\_hom x
-thi\_nlevels 4
\end{alltt}
\item What happens if you remove \code{-dmmg\_grid\_sequence}?
\item What about solving with block Jacobi, ASM, or algebraic multigrid?
\end{itemize}
\end{frame}

\subsection{Driven cavity}
\begin{frame}{SNES Example}
\framesubtitle{Driven Cavity}
\hbox{
\includegraphics[width=.4\textwidth]{figures/SNES/DrivenCavitySolution}
\vbox{
\begin{itemize}
  \item Velocity-vorticity formulation
  \item Flow driven by lid and/or bouyancy
  \item Logically regular grid
  \begin{itemize}
    \item Parallelized with {\kb DA}
  \end{itemize}
  \item Finite difference discretization
  \item Authored by David Keyes
\end{itemize}
}
}
\end{frame}

\begin{frame}[fragile]{SNES Example}
\framesubtitle{Driven Cavity Application Context}
\begin{lstlisting}
/* Collocated at each node */
typedef struct {
  PetscScalar u,v,omega,temp;
} Field;

typedef struct {
       /* physical parameters */
   PassiveReal lidvelocity,prandtl,grashof;
       /* color plots of the solution */
   PetscTruth  draw_contours;
} AppCtx;
\end{lstlisting}
\end{frame}

\begin{frame}[fragile]{SNES Example}
\framesubtitle{Driven Cavity Residual Evaluation}
\small
\begin{lstlisting}
DrivenCavityFunction(SNES snes, Vec X, Vec F, void *ptr) {
  AppCtx        *user = (AppCtx *) ptr;
  /* local starting and ending grid points */
  PetscInt       istart, iend, jstart, jend;
  PetscScalar    *f;             /* local vector data */
  PetscReal      grashof = user->grashof;  
  PetscReal      prandtl = user->prandtl;
  PetscErrorCode ierr;

  /* Code to communicate nonlocal ghost point data */
  VecGetArray(F, &f);
  /* Code to compute local function components */
  VecRestoreArray(F, &f);
  return 0;
}
\end{lstlisting}
\end{frame}

\begin{frame}[fragile]{SNES Example}
\framesubtitle{Better Driven Cavity Residual Evaluation}

\small
\begin{lstlisting}
PetscErrorCode DrivenCavityFuncLocal(DALocalInfo *info,
                    Field **x,Field **f,void *ctx) {
  /* Handle boundaries */
  /* Compute over the interior points */
  for(j = info->ys; j < info->ys+info->ym; j++) {
    for(i = info->xs; i < info->xs+info->xm; i++) {
      /* convective coefficients for upwinding */
      /* U velocity */
      u          = x[j][i].u;
      uxx        = (2.0*u - x[j][i-1].u - x[j][i+1].u)*hydhx;
      uyy        = (2.0*u - x[j-1][i].u - x[j+1][i].u)*hxdhy;
      upw        = 0.5*(x[j+1][i].omega-x[j-1][i].omega)*hx
      f[j][i].u  = uxx + uyy - upw;
      /* V velocity, Omega, Temperature */
}}}
\end{lstlisting}

\begin{center}\small
\$PETSC\_DIR/src/snes/examples/tutorials/ex19.c
\end{center}
\end{frame}


\begin{frame}[fragile]{Running the driven cavity}
\footnotesize
  \begin{itemize}
  \item \code{./ex19 -lidvelocity 100 -grashof \alert{1e2} -da\_grid\_x 16 -da\_grid\_y 16 -snes\_monitor -snes\_view -da\_refine 2}
\only<2>{{\scriptsize \color{green!30!black} \tt \\
lid velocity = 100, prandtl \# = 1, grashof \# = 1000 \\
  0 SNES Function norm 7.682893957872e+02 \\
  1 SNES Function norm 6.574700998832e+02 \\
  2 SNES Function norm 5.285205210713e+02 \\
  3 SNES Function norm 3.770968117421e+02 \\
  4 SNES Function norm 3.030010490879e+02 \\
  5 SNES Function norm 2.655764576535e+00 \\
  6 SNES Function norm 6.208275817215e-03 \\
  7 SNES Function norm 1.191107243692e-07 \\
Number of SNES iterations = 7
}}
  \item \code{./ex19 -lidvelocity 100 -grashof \alert{1e4} -da\_grid\_x 16 -da\_grid\_y 16 -snes\_monitor -snes\_view -da\_refine 2}
\only<3>{{\scriptsize \color{green!30!black} \tt \\
lid velocity = 100, prandtl \# = 1, grashof \# = 10000 \\
  0 SNES Function norm 7.854040793765e+02 \\
  1 SNES Function norm 6.630545177472e+02 \\
  2 SNES Function norm 5.195829874590e+02 \\
  3 SNES Function norm 3.608696664876e+02 \\
  4 SNES Function norm 2.458925075918e+02 \\
  5 SNES Function norm 1.811699413098e+00 \\
  6 SNES Function norm 4.688284580389e-03 \\
  7 SNES Function norm 4.417003604737e-08 \\
Number of SNES iterations = 7
}}
  \item \code{./ex19 -lidvelocity 100 -grashof \alert{1e5} -da\_grid\_x 16 -da\_grid\_y 16 -snes\_monitor -snes\_view -da\_refine 2 -pc\_type lu}
\only<4>{{\scriptsize \color{green!30!black} \tt \\
lid velocity = 100, prandtl \# = 1, grashof \# = 100000 \\
  0 SNES Function norm 1.809960438828e+03 \\
  1 SNES Function norm 1.678372489097e+03 \\
  2 SNES Function norm 1.643759853387e+03 \\
  3 SNES Function norm 1.559341161485e+03 \\
  4 SNES Function norm 1.557604282019e+03 \\
  5 SNES Function norm 1.510711246849e+03 \\
  6 SNES Function norm 1.500472491343e+03 \\
  7 SNES Function norm 1.498930951680e+03 \\
  8 SNES Function norm 1.498440256659e+03 \\
  ...
}}
  \item<5-> Uh oh, we have convergence problems
  \item<5-> Does \code{-snes\_grid\_sequence} help?
  \end{itemize}
\end{frame}

\begin{frame}{Why isn't SNES converging?}
  \begin{itemize}
  \item The Jacobian is wrong (maybe only in parallel)
    \oneitem{Check with \code{-snes\_type test} and \code{-snes\_mf\_operator -pc\_type lu}}
  \item The linear system is not solved accurately enough
    \begin{itemize}
    \item Check with \code{-pc\_type lu}
    \item Check \code{-ksp\_monitor\_true\_residual}, try right preconditioning
    \end{itemize}
  \item The Jacobian is singular with inconsistent right side
    \begin{itemize}
    \item Use \code{MatNullSpace} to inform the \code{KSP} of a known null space
    \item Use a different Krylov method or preconditioner
    \end{itemize}
  \item The nonlinearity is just really strong
    \begin{itemize}
    \item Run with \code{-snes\_linesearch\_monitor}
    \item Try using trust region instead of line search \code{-snes\_type newtontr}
    \item Try grid sequencing if possible
    \item Use a continuation
    \end{itemize}
  \end{itemize}
\end{frame}

\begin{frame}{Globalizing the lid-driven cavity}
  \begin{itemize}
  \item Pseudotransient continuation ($\Psi tc$)
    \begin{itemize}
    \item Do linearly implicit backward-Euler steps, driven by
      steady-state residual
    \item Residual-based adaptive controller retains quadratic
      convergence in terminal phase
    \end{itemize}
  \item Implemented in \code{src/ts/examples/tutorials/ex26.c}
  \item {\footnotesize \shell{./ex26 -ts\_type pseudo -lidvelocity 100 -grashof 1e5 -da\_grid\_x 16 -da\_grid\_y 16 -ts\_monitor}}
\only<2>{{\tiny \color{green!30!black} \tt \\
16x16 grid, lid velocity = 100, prandtl \# = 1, grashof \# = 100000 \\
0 TS dt 0.03125 time 0 \\
1 TS dt 0.034375 time 0.034375 \\
2 TS dt 0.0398544 time 0.0742294 \\
3 TS dt 0.0446815 time 0.118911 \\
4 TS dt 0.0501182 time 0.169029 \\
... \\
24 TS dt 3.30306 time 11.2182 \\
25 TS dt 8.24513 time 19.4634 \\
26 TS dt 28.1903 time 47.6537 \\
27 TS dt 371.986 time 419.64 \\
28 TS dt 379837 time 380257 \\
29 TS dt 3.01247e+10 time 3.01251e+10 \\
30 TS dt 6.80049e+14 time 6.80079e+14 \\
CONVERGED\_TIME at time 6.80079e+14 after 30 steps
}}
  \item<3> Make the method nonlinearly implicit: \code{-snes\_type ls -snes\_monitor}
    \oneitem{Compare required number of linear iterations}
  \item<3> Try error-based adaptivity: \code{-ts\_type rosw -ts\_adapt\_dt\_min 1e-4}
  \item<3> Try increasing \code{-lidvelocity}, \code{-grashof}, and problem size
  \item<3> Coffey, Kelley, and Keyes, \emph{Pseudotransient continuation and differential algebraic equations}, SIAM J. Sci. Comp, 2003.
  \end{itemize}
\end{frame}

\begin{frame}{Nonlinear multigrid (full approximation scheme)}
\footnotesize
  \begin{itemize}
  \item V-cycle structure, but use nonlinear relaxation and skip the matrices
  \item \code{./ex19 -da\_refine 4 -snes\_monitor -snes\_type \alert{nrichardson} -npc\_fas\_levels\_snes\_type gs -npc\_fas\_levels\_snes\_gs\_sweeps 3 -npc\_snes\_type fas -npc\_fas\_levels\_snes\_type gs -npc\_snes\_max\_it 1 -npc\_snes\_fas\_smoothup 6 -npc\_snes\_fas\_smoothdown 6 -lidvelocity 100 -grashof 4e4}
\only<2>{{\tiny \color{green!30!black} \tt \\
lid velocity = 100, prandtl \# = 1, grashof \# = 40000 \\
  0 SNES Function norm 1.065744184802e+03 \\
  1 SNES Function norm 5.213040454436e+02 \\
  2 SNES Function norm 6.416412722900e+01 \\
  3 SNES Function norm 1.052500804577e+01 \\
  4 SNES Function norm 2.520004680363e+00 \\
  5 SNES Function norm 1.183548447702e+00 \\
  6 SNES Function norm 2.074605179017e-01 \\
  7 SNES Function norm 6.782387771395e-02 \\
  8 SNES Function norm 1.421602038667e-02 \\
  9 SNES Function norm 9.849816743803e-03 \\
 10 SNES Function norm 4.168854365044e-03 \\
 11 SNES Function norm 4.392925390996e-04 \\
 12 SNES Function norm 1.433224993633e-04 \\
 13 SNES Function norm 1.074357347213e-04 \\
 14 SNES Function norm 6.107933844115e-05 \\
 15 SNES Function norm 1.509756087413e-05 \\
 16 SNES Function norm 3.478180386598e-06 \\
Number of SNES iterations = 16
}}
\item \code{./ex19 -da\_refine 4 -snes\_monitor -snes\_type \alert{ngmres} -npc\_fas\_levels\_snes\_type gs -npc\_fas\_levels\_snes\_gs\_sweeps 3 -npc\_snes\_type fas -npc\_fas\_levels\_snes\_type gs -npc\_snes\_max\_it 1 -npc\_snes\_fas\_smoothup 6 -npc\_snes\_fas\_smoothdown 6 -lidvelocity 100 -grashof 4e4}
\only<3>{{\tiny \color{green!30!black} \tt \\
lid velocity = 100, prandtl \# = 1, grashof \# = 40000 \\
  0 SNES Function norm 1.065744184802e+03 \\
  1 SNES Function norm 9.413549877567e+01 \\
  2 SNES Function norm 2.117533223215e+01 \\
  3 SNES Function norm 5.858983768704e+00 \\
  4 SNES Function norm 7.303010571089e-01 \\
  5 SNES Function norm 1.585498982242e-01 \\
  6 SNES Function norm 2.963278257962e-02 \\
  7 SNES Function norm 1.152790487670e-02 \\
  8 SNES Function norm 2.092161787185e-03 \\
  9 SNES Function norm 3.129419807458e-04 \\
 10 SNES Function norm 3.503421154426e-05 \\
 11 SNES Function norm 2.898344063176e-06 \\
Number of SNES iterations = 11}}
  \end{itemize}
\end{frame}


\section{Difficult and coupled problems}
\begin{frame}{Splitting for Multiphysics}
  \begin{equation*}
    \begin{bmatrix}
      A & B \\ C & D
    \end{bmatrix}
    \begin{bmatrix}
      x \\ y
    \end{bmatrix}
    =
    \begin{bmatrix}
      f \\ g
    \end{bmatrix}
  \end{equation*}
  \begin{itemize}\item Relaxation:
    \code{-pc\_fieldsplit\_type [additive,multiplicative,symmetric\_multiplicative]}
    \begin{equation*}
      \begin{bmatrix}
        A & \\  & D
      \end{bmatrix}^{-1} \qquad 
      \begin{bmatrix}
        A & \\ C & D
      \end{bmatrix}^{-1} \qquad
      \begin{bmatrix}
        A & \\  & \bm 1
      \end{bmatrix}^{-1}
      \left(
        \bm 1 -
        \begin{bmatrix}
          A & B \\ & \bm 1
        \end{bmatrix}
        \begin{bmatrix}
          A & \\ C & D
        \end{bmatrix}^{-1}
      \right)
    \end{equation*}
    \begin{itemize}
    \item Gauss-Seidel inspired, works when fields are loosely coupled
    \end{itemize}
  \item Factorization: \code{-pc\_fieldsplit\_type schur}
    \begin{align*}
      \begin{bmatrix}
        A & B \\ & S
      \end{bmatrix}^{-1}
      \begin{bmatrix}
        1 & \\ CA^{-1} & 1
      \end{bmatrix}^{-1}, \qquad
      S = D - C A^{-1} B
    \end{align*}
    \begin{itemize}
    \item robust (exact factorization), can often drop lower block
    \item how to precondition $S$ which is usually dense?
      \begin{itemize}
      \item interpret as differential operators, use approximate commutators
      \end{itemize}
    \end{itemize}
  \end{itemize}
\end{frame}

\begin{frame}{Coupled approach to multiphysics}
  \begin{itemize}
  \item Smooth all components together
    \begin{itemize}
    \item Block SOR is the most popular
    \item Vanka smoothers for indefinite problems
    \end{itemize}
  \item Interpolate in a compatible way
  \item Scaling between fields is critical
  \item First-order upwind for transport
  \item Coarse space and subdomain problems should be compatible with inf-sup condition
  \item Open research area
  \end{itemize}
\end{frame}

\begin{frame}{Anisotropy and Heterogeneity}
  \begin{itemize}
  \item Anisotropy
    \begin{itemize}
    \item Semi-coarsening
    \item Line smoothers
    \item Order unknowns so that incomplete factorization ``includes'' a
      line smoother
    \end{itemize}
  \item Heterogeneity
    \begin{itemize}
    \item Make coarse grids align
    \item Strong smoothers
    \item Energy-minimizing interpolants
    \end{itemize}
  \end{itemize}
\end{frame}

\begin{frame}{Algebraic Multigrid Tuning}
  \begin{itemize}
  \item Smoothed Aggregation (GAMG, ML)
    \begin{itemize}
    \item Graph/strength of connection -- MatSetBlockSize()
    \item Threshold (\code{-pc\_gamg\_threshold})
    \item Aggregate (MIS, HEM)
    \item Tentative prolongation -- MatSetNearNullSpace()
    \item Eigenvalue estimate
    \item Chebyshev smoothing bounds
    \end{itemize}
  \item BoomerAMG (Hypre)
    \begin{itemize}
    \item Strong threshold (\code{-pc\_hypre\_boomeramg\_strong\_threshold})
    \item Aggressive coarsening options
    \end{itemize}
  \end{itemize}
\end{frame}


\end{document}
