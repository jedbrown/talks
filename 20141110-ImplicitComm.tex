% \documentclass[handout]{beamer}
\documentclass{beamer}

\mode<presentation>
{
  \usetheme{ANLBlue}
  % \usefonttheme[onlymath]{serif}
  % \usetheme{Singapore}
  % \usetheme{Warsaw}
  % \usetheme{Malmoe}
  % \useinnertheme{circles}
  % \useoutertheme{infolines}
  % \useinnertheme{rounded}

  \setbeamercovered{transparent=20}
}

\usepackage[english]{babel}
\usepackage[latin1]{inputenc}
\usepackage{alltt,listings,multirow,ulem,siunitx}
\usepackage[absolute,overlay]{textpos}
\TPGrid{1}{1}
\usepackage{pdfpages}
\usepackage{ulem}
\usepackage{multimedia}
\usepackage{multicol}
\newcommand\hmmax{0}
\newcommand\bmmax{0}
\usepackage{bm}
\usepackage{comment}
\usepackage{subcaption}

% font definitions, try \usepackage{ae} instead of the following
% three lines if you don't like this look
\usepackage{mathptmx}
\usepackage[scaled=.90]{helvet}
% \usepackage{courier}
\usepackage[T1]{fontenc}
\usepackage{tikz}
\usetikzlibrary{decorations.pathreplacing}
\usetikzlibrary{shadows,arrows,shapes.misc,shapes.arrows,shapes.multipart,arrows,decorations.pathmorphing,backgrounds,positioning,fit,petri,calc,shadows,chains,matrix}

\newcommand\vvec{\bm v}
\newcommand\bvec{\bm b}
\newcommand\bxk{\bvec_0 \times \kappa_0 \cdot \nabla}
\newcommand\delp{\nabla_\perp}

% \usepackage{pgfpages}
% \pgfpagesuselayout{4 on 1}[a4paper,landscape,border shrink=5mm]

\usepackage{JedMacros}

\newcommand{\timeR}{t_{\mathrm{R}}}
\newcommand{\timeW}{t_{\mathrm{W}}}
\newcommand{\mglevel}{\ensuremath{\ell}}
\newcommand{\mglevelcp}{\ensuremath{\mglevel_{\mathrm{cp}}}}
\newcommand{\mglevelcoarse}{\ensuremath{\mglevel_{\mathrm{coarse}}}}
\newcommand{\mglevelfine}{\ensuremath{\mglevel_{\mathrm{fine}}}}

%solution and residual
\newcommand{\vx}{\ensuremath{x}}
\newcommand{\vc}{\ensuremath{\hat{x}}}
\newcommand{\vr}{\ensuremath{r}}
\newcommand{\vb}{\ensuremath{b}}

%operators
\newcommand{\vA}{\ensuremath{A}}
\newcommand{\vP}{\ensuremath{I_H^h}}
\newcommand{\vS}{\ensuremath{S}}
\newcommand{\vR}{\ensuremath{I_h^H}}
\newcommand{\vI}{\ensuremath{\hat I_h^H}}
\newcommand{\vV}{\ensuremath{\mathbf{V}}}
\newcommand{\vF}{\ensuremath{F}}
\newcommand{\vtau}{\ensuremath{\mathbf{\tau}}}


\title{Can implicit integrators have \emph{less} data motion than explicit?}
\author{{\bf Jed Brown} \texttt{jedbrown@mcs.anl.gov} (ANL and CU Boulder) \\
  Debojyoti Ghosh (ANL)}

% - Use the \inst command only if there are several affiliations.
% - Keep it simple, no one is interested in your street address.
% \institute
% {
%   Mathematics and Computer Science Division \\ Argonne National Laboratory
% }

\date{$[HPC]^3$, 2014-11-10 \\
This talk: \url{http://59A2.org/files/20141110-ImplicitComm.pdf}}

% This is only inserted into the PDF information catalog. Can be left
% out.
\subject{Talks}


% If you have a file called "university-logo-filename.xxx", where xxx
% is a graphic format that can be processed by latex or pdflatex,
% resp., then you can add a logo as follows:

% \pgfdeclareimage[height=0.5cm]{university-logo}{university-logo-filename}
% \logo{\pgfuseimage{university-logo}}



% Delete this, if you do not want the table of contents to pop up at
% the beginning of each subsection:
% \AtBeginSubsection[]
% {
% \begin{frame}<beamer>
%   \frametitle{Outline}
%   \tableofcontents[currentsection,currentsubsection]
% \end{frame}
% }

\AtBeginSection[]
{
  \begin{frame}<beamer>
    \frametitle{Outline}
    \tableofcontents[currentsection]
  \end{frame}
}

% If you wish to uncover everything in a step-wise fashion, uncomment
% the following command:

% \beamerdefaultoverlayspecification{<+->}

\begin{document}
\lstset{language=C}
\normalem

\begin{frame}
  \titlepage
\end{frame}

\begin{frame}{Why implicit is silly for waves}
  \begin{itemize}
  \item Implicit methods require an implicit solve in each stage.
  \item Time step size proportional to CFL for accuracy reasons.
  \item Methods higher than first order are not unconditionally strong stability preserving (SSP; Spijker 1983).
    \begin{itemize}
    \item Empirically, $c_{\text{eff}} \le 2$, Ketcheson, Macdonald, Gottlieb (2008) and others
    \item Downwind methods offer to bypass, but so far not practical
    \end{itemize}
  \item Time step size chosen for stability
    \begin{itemize}
    \item Increase order if more accuracy needed
    \item Large errors from spatial discretization, modest accuracy
    \end{itemize}
  \item My goal: need less memory motion \emph{per stage}
    \begin{itemize}
    \item Better accuracy, symplecticity nice bonus only
    \item Cannot sell method without efficiency
    \end{itemize}
  \end{itemize}
\end{frame}

\begin{frame}{Motivation}
  \begin{itemize}
  \item Hardware trends
    \begin{itemize}
    \item Memory bandwidth a precious commodity (8+ flops/byte)
    \item Vectorization necessary for floating point performance
    \item Conflicting demands of cache reuse and vectorization
    \item Can deliver bandwidth, but latency is hard
    \end{itemize}
  \item Assembled sparse linear algebra is doomed!
    \begin{itemize}
    \item Limited by memory bandwidth (1 flop/6 bytes)
    \item No vectorization without blocking
    \end{itemize}
  \item Spatial-domain vectorization is \emph{intrusive}
    \begin{itemize}
    \item Must be unassembled to avoid bandwidth bottleneck
    \item Whether it is ``hard'' depends on discretization
    \item Geometry, boundary conditions, and adaptivity
    \end{itemize}
  \end{itemize}
\end{frame}

\begin{frame}{Hardware Arithmetic Intensity}
  \begin{tabular}{lc}
    \toprule
    Operation                         & Arithmetic Intensity (flops/B) \\
    \midrule
    Sparse matrix-vector product      & 1/6                  \\
    Dense matrix-vector product       & 1/4                  \\
    Unassembled matrix-vector product & $\approx 8$          \\
    High-order residual evaluation    & $> 5$                \\
    \bottomrule
  \end{tabular}

  \bigskip

  \begin{tabular}{lrrr}
    \toprule
    Processor & BW (GB/s) & Peak (GF/s) & Balanced AI (F/B) \\
    \midrule
    E5-2670 8-core      & 35   & 166  & 4.7 \\
    Magny Cours 16-core & 49   & 281  & 5.7 \\
    Blue Gene/Q node    & 43   & 205  & 4.8 \\
    Tesla M2090         & 120  & 665  & 5.5 \\
    Kepler K20Xm        & 160 & 1310 & 8.2 \\ % http://www.elekslabs.com/2012/11/nvidia-tesla-k20-benchmark-facts.html
    Xeon Phi            & 150 & 1248 & 8.3 \\
    \bottomrule
  \end{tabular}
\end{frame}


\begin{frame}{Sparse linear algebra is dead (long live sparse \ldots)}
  \begin{itemize}
  \item Arithmetic intensity $< 1/4$
  \item Idea: multiple right hand sides
    \begin{equation*}
      \frac{(2 k \text{ flops})(\text{bandwidth})}{\texttt{sizeof(Scalar)} + \texttt{sizeof(Int)}}, \quad k \ll \text{avg. nz/row}
    \end{equation*}
  \item Problem: popular algorithms have nested data dependencies
    \begin{itemize}
    \item Time step \\
      \qquad Nonlinear solve \\
      \qquad \qquad Krylov solve \\
      \qquad \qquad \qquad Preconditioner/sparse matrix
    \end{itemize}
  \item Cannot parallelize/vectorize these nested loops
  \item<2> \alert{Can we create new algorithms to reorder/fuse loops?}
    \begin{itemize}
    \item Reduce latency-sensitivity for communication
    \item Reduce memory bandwidth (reuse matrix while in cache)
    \end{itemize}
  \end{itemize}
\end{frame}

\begin{frame}{Attempt: $s$-step methods in 3D}
  \includegraphics[width=1.1\textwidth]{figures/SStepEfficiency.pdf}
  \begin{itemize}
  \item Limited choice of preconditioners (none optimal, surface/volume)
  \item Amortizing message latency is most important for strong-scaling
  \item $s$-step methods have high overhead for small subdomains
  \end{itemize}
\end{frame}

\begin{frame}{Attempt: distribute in time (multilevel SDC/Parareal)}
  \includegraphics[width=0.9\textwidth]{figures/EmmettMinionPFASSTCost.png}
  \begin{itemize}
  \item PFASST algorithm (Emmett and Minion, 2012)
  \item Zero-latency messages (cf. performance model of $s$-step)
  \item Spectral Deferred Correction: iterative, converges to IRK (Gauss, Radau, \ldots)
  \item Stiff problems use implicit basic integrator (synchronizing on spatial communicator)
  \end{itemize}
\end{frame}

\begin{frame}{Problems with SDC and time-parallel}
  \includegraphics[width=\textwidth]{figures/TS/SDCScalingEmmett.png} \\
  c/o Matthew Emmett, parallel compared to sequential SDC
  \begin{itemize}
  \item Iteration count not uniform in $s$; efficiency starts low
  \item Low arithmetic intensity; tight error tolerance (cf. Crank-Nicolson)
  \item Parabolic space-time (Greenwald and Brandt; Horton and Vandewalle)
  \end{itemize}
\end{frame}

\begin{frame}{Runge-Kutta methods}
  \begin{gather*}
    \dot u = F(u) \\
    \underbrace{
    \begin{pmatrix}
      y_1 \\
      \vdots \\
      y_s
    \end{pmatrix}}_Y =
    u^{n} + h
    \underbrace{
    \begin{bmatrix}
      a_{11} & \dotsb & a_{1s} \\
      \vdots & \ddots & \vdots \\
      a_{s1} & \dotsb & a_{ss}
    \end{bmatrix}}_A
    F
    \begin{pmatrix}
      y_1 \\
      \vdots \\
      y_s
    \end{pmatrix} \\
    u^{n+1} = u^n + b^T Y
  \end{gather*}
  \begin{itemize}
  \item General framework for one-step methods
  \item Diagonally implicit: $A$ lower triangular, stage order 1 (or 2 with explicit first stage)
  \item Singly diagonally implicit: all $A_{ii}$ equal, reuse solver setup, stage order 1
  \item If $A$ is a general full matrix, all stages are coupled, ``implicit RK''
  \end{itemize}
\end{frame}

\begin{frame}{Implicit Runge-Kutta}
  \begin{center}
    \begin{tabular}{>{$}c<{$} | >{$}c<{$} >{$}c<{$} >{$}c<{$}}
      \frac 1 2 - \frac{\sqrt{15}}{10} & \frac{5}{36} & \frac 2 9 - \frac{\sqrt{15}}{15} & \frac{5}{36} - \frac{\sqrt{15}}{30} \\
      \frac 1 2 & \frac{5}{36} + \frac{\sqrt{15}}{24} & \frac 2 9 & \frac{5}{36} - \frac{\sqrt{15}}{24} \\
      \frac 1 2 - \frac{\sqrt{15}}{10} & \frac{5}{36} + \frac{\sqrt{15}}{30} & \frac 2 9 + \frac{\sqrt{15}}{15} & \frac{5}{36} \\[4pt]
      \hline
      \vspace{4pt}
      & \frac{5}{18} & \frac 4 9 & \frac{5}{18}
    \end{tabular}
  \end{center}
  \begin{itemize}
  \item Excellent accuracy and stability properties
  \item Gauss methods with $s$ stages
    \begin{itemize}
    \item order $2s$, $(s,s)$ Pad\'e approximation to the exponential
    \item $A$-stable, symplectic
    \end{itemize}
  \item Radau (IIA) methods with $s$ stages
    \begin{itemize}
    \item order $2s-1$, $A$-stable, $L$-stable
    \end{itemize}
  \item Lobatto (IIIC) methods with $s$ stages
    \begin{itemize}
    \item order $2s-2$, $A$-stable, $L$-stable, self-adjoint
    \end{itemize}
  \item Stage order $s$ or $s+1$    
  \end{itemize}
\end{frame}

\begin{frame}{Method of Butcher (1976) and Bickart (1977)}
  \begin{itemize}
  \item Newton linearize Runge-Kutta system at $u^*$
    \begin{align*}
      Y &= u^{n} + h A F(Y) & \big[ I_s \otimes I_n + h A \otimes J(u^*)\big ] \delta Y &= RHS
    \end{align*}
  \item Solve linear system with tensor product operator
    \begin{equation*}
      \hat G = S \otimes I_n + I_s \otimes J
    \end{equation*}
    where $S = (hA)^{-1}$ is $s\times s$ dense, $J = -\partial F(u)/\partial u$ sparse
  \item SDC (2000) is Gauss-Seidel with low-order corrector
  \item Butcher/Bickart method: diagonalize $S = X \Lambda X^{-1}$
    \begin{itemize}
    \item $\Lambda \otimes I_n + I_s \otimes J$
    \item $s$ decoupled solves
    \item Complex eigenvalues (overhead for real problem)
    \end{itemize}
  \item Problem: $X$ is exponentially ill-conditioned wrt. $s$
  \item We avoid diagonalization
    \begin{itemize}
    \item Permute $\hat G$ to reuse $J$: $G = I_n \otimes S + J \otimes I_s$
    \item Stages coupled via register transpose at spatial-point granularity
    \item Same convergence properties as Butcher/Bickart
    \end{itemize}
  \end{itemize}
\end{frame}

\begin{frame}{MatTAIJ: ``sparse'' tensor product matrices}
  \begin{gather*}
    G = I_n \otimes S + J \otimes T
  \end{gather*}
  \begin{itemize}
  \item $J$ is parallel and sparse, $S$ and $T$ are small and dense
  \item More general than multiple RHS (multivectors)
  \item Compare $J \otimes I_s$ to multiple right hand sides in row-major
  \item Runge-Kutta systems have $T = I_s$ (permuted from Butcher method)
  \item Stream $J$ through cache once, same efficiency as multiple RHS
  \item Unintrusive compared to spatial-domain vectorization or $s$-step
  \end{itemize}
\end{frame}

\begin{frame}{Convergence with point-block Jacobi preconditioning}
  \begin{itemize}
  \item 3D centered-difference diffusion problem
  \end{itemize}
  \begin{tabular}{lrrrrr}
    \toprule
    Method & order & nsteps & Krylov its. & (Average) \\
    \midrule
    Gauss 1 & 2 & 16 & 130 & (8.1) \\
    Gauss 2 & 4 & 8 & 122 & (15.2) \\
    Gauss 4 & 8 & 4 & 100 & (25) \\
    Gauss 8 & 16 & 2 & 78 & (39) \\
    \bottomrule
  \end{tabular}
\end{frame}

\begin{frame}{We really want multigrid}
  \begin{itemize}
  \item Prolongation: $P \otimes I_s$
  \item Coarse operator: $I_n \otimes S + (R J P) \otimes I_s$
  \item Larger time steps
  \item GMRES(2)/point-block Jacobi smoothing
  \item FGMRES outer
  \end{itemize}
  \begin{tabular}{lrrrrr}
    \toprule
    Method & order & nsteps & Krylov its. & (Average) \\
    \midrule
    Gauss 1 & 2 & 16 & 82 & (5.1) \\
    Gauss 2 & 4 & 8 & 64 & (8) \\
    Gauss 4 & 8 & 4 & 44 & (11) \\
    Gauss 8 & 16 & 2 & 42 & (21) \\
    \bottomrule
  \end{tabular}
\end{frame}

\begin{frame}{Toward a better AMG for IRK/tensor-product systems}
  \begin{columns}
    \begin{column}{0.3\textwidth}
      \includegraphics[width=\textwidth]{figures/TS/Gauss8-Eig.png}
    \end{column}
    \begin{column}{0.7\textwidth}
      \begin{itemize}
      \item Start with $\hat R = R \otimes I_s$, $\hat P = P \otimes I_s$
        \begin{gather*}
          G_{\text{coarse}} = \hat R (I_n \otimes S + J \otimes I_s) \hat P
        \end{gather*}
      \item Imaginary component slows convergence
      \item Idea: rotate eigenvalues on coarse levels \\
        Erlangga and Nabben \emph{On a multilevel Krylov method for the Helmholtz equation preconditioned by shifted Laplacian}
      \end{itemize}
    \end{column}
  \end{columns}
\end{frame}

\begin{frame}{Implicit Runge-Kutta for advection}
  \begin{table}
    \centering
    \caption{Total number of iterations (communications or accesses of $J$) to solve linear advection to $t=1$ on a $1024$-point grid using point-block Jacobi preconditioning of implicit Runge-Kutta matrix.
      The relative algebraic solver tolerance is $10^{-8}$.}\label{tab:irk-advection}
    \begin{tabular}{lrrr}
      \toprule
      Family & Stages & Order & Iterations \\
      \midrule
      Crank-Nicolson/Gauss & 1 & 2 & 3627 \\
      Gauss & 2 & 4 & 2560 \\
      Gauss & 4 & 8 & 1735 \\
      Gauss & 8 & 16 & 1442 \\
      \bottomrule
    \end{tabular}
  \end{table}
  \begin{itemize}
  \item Naive centered-difference discretization
  \item Leapfrog requires 1024 iterations at CFL=1
  \item This is $A$-stable (can handle dissipation)
  \end{itemize}
\end{frame}

\begin{frame}{Outlook on IRK}
  \begin{itemize}
  \item IRK \emph{unintrusively} offers bandwidth reuse and vectorization
  \item No need for complex arithmetic (cf. Butcher and Bickart)
  \item Need polynomial smoothers for IRK spectra
  \item Change number of stages on spatially-coarse grids ($p$-MG, or even increase)?
  \item Experiment with SOR-type smoothers
    \begin{itemize}
    \item Prefer point-block Jacobi in smoothers for parallelism
    \end{itemize}
  \item Study efficiency for nonlinear problems
  \item Is it possible to speed up advection?
  \item Possible IRK correction for IMEX (non-smooth explicit function)
  \item PETSc implementation (parallel example running, interface in-progress)
  \end{itemize}
\end{frame}
\end{document}
