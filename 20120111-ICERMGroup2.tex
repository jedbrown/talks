% \documentclass[handout]{beamer}
\documentclass{beamer}

\mode<presentation>
{
  \usetheme{default}
  \usefonttheme[onlymath]{serif}
  % \usetheme{Singapore}
  % \usetheme{Warsaw}
  % \usetheme{Malmoe}
  % \useinnertheme{circles}
  % \useoutertheme{infolines}
  % \useinnertheme{rounded}

  \setbeamercovered{transparent=5}
}

\usepackage[english]{babel}
\usepackage[latin1]{inputenc}
\usepackage{textpos,alltt,listings,multirow,ulem,siunitx}
\usepackage{pdfpages}
\newcommand\hmmax{0}
\newcommand\bmmax{0}
\usepackage{bm}

% font definitions, try \usepackage{ae} instead of the following
% three lines if you don't like this look
\usepackage{mathptmx}
\usepackage[scaled=.90]{helvet}
% \usepackage{courier}
\usepackage[T1]{fontenc}
\usepackage{tikz}
\usetikzlibrary[shapes,shapes.arrows,arrows,shapes.misc,fit,positioning]

% \usepackage{pgfpages}
% \pgfpagesuselayout{4 on 1}[a4paper,landscape,border shrink=5mm]

\usepackage{JedMacros}

\title{Library considerations}
\author{Jed Brown, Felipe Cruz, Matt Knepley, Richard Mills, John Owens, Jack Poulson, Ulrike Meier Yang}


% - Use the \inst command only if there are several affiliations.
% - Keep it simple, no one is interested in your street address.

\date{2012-01-11}

% This is only inserted into the PDF information catalog. Can be left
% out.
\subject{Talks}


% If you have a file called "university-logo-filename.xxx", where xxx
% is a graphic format that can be processed by latex or pdflatex,
% resp., then you can add a logo as follows:

% \pgfdeclareimage[height=0.5cm]{university-logo}{university-logo-filename}
% \logo{\pgfuseimage{university-logo}}



% Delete this, if you do not want the table of contents to pop up at
% the beginning of each subsection:
% \AtBeginSubsection[]
% {
% \begin{frame}<beamer>
%   \frametitle{Outline}
%   \tableofcontents[currentsection,currentsubsection]
% \end{frame}
% }

\AtBeginSection[]
{
  \begin{frame}<beamer>
    \frametitle{Outline}
    \tableofcontents[currentsection]
  \end{frame}
}

% If you wish to uncover everything in a step-wise fashion, uncomment
% the following command:

% \beamerdefaultoverlayspecification{<+->}

\begin{document}
\lstset{language=C}
\normalem

\begin{frame}
  \titlepage
\end{frame}

\begin{frame}{Desired library features}
  \begin{itemize}
  \item Distribution control
    \begin{itemize}
    \item dense vs sparse
    \item direct vs iterative solvers
    \item hierarchical memory
    \item different communicators for different physics/multigrid levels
    \end{itemize}
  \item Scheduling control
    \begin{itemize}
    \item when and where to do which computation?
    \item how does user give advice?
    \item how to degrade gracefully when resources are unavailable?
    \end{itemize}
  \item Asynchronous interfaces
    \begin{itemize}
    \item crossing module/library boundaries
    \item how to ensure progress for operations that need user involvement
    \item manage dependencies to prevent resource starvation causing long chains
    \end{itemize}
  \item Hierarchical interfaces/levels of abstraction
  \item Composability and extensibility
    \begin{itemize}
    \item experimentation requires decoupling algorithms from problem specification
    \item plugin-style extensibility often necessary for multiphysics coupling
    \end{itemize}
  \end{itemize}
\end{frame}

\begin{frame}{Concerns for library implementers}
  \begin{itemize}
  \item user/library interfaces
  \item is ``performance portable'' dead?
    \begin{itemize}
    \item what can be dynamic?
    \item specialization technology (c++ templates, light DSLs, Python codegen)
    \end{itemize}
  \item {\bf \large F}rameworks and runtimes
  \item type sizes: 32 and 64-bit indices; single, double, quad precision
  \item data structure conversion vs independence
  \item loop and communication fusion
    \begin{itemize}
    \item compile-time versus run-time
    \item BLAS-1 operations, $s$-step methods
    \item aggregating parallel communication
    \end{itemize}
  \item Tool support
    \begin{itemize}
    \item functionality needed for adoption
    \item adoption needed for hardening and funding
    \end{itemize}
  \item Derivatives: current solutions are quite intrusive
  \end{itemize}
\end{frame}

\begin{frame}{Common themes and challenges}
  \begin{itemize}
  \item Configuration problem is getting harder
  \item Run-time versus compile-time choices
  \item Interfaces for extensibility are necessary, but delicate
  \item Libraries need to be good citizens
    \begin{itemize}
    \item portable configuration systems
    \item namespacing, restrictions to communicators, error handlers
    \item algorithmic and performance diagnostics
    \end{itemize}
  \item Fine-grained parallelism, interoperability between threading models
  \end{itemize}
\end{frame}
\end{document}
