% \documentclass[handout]{beamer}
\documentclass{beamer}

\mode<presentation>
{
  \usetheme{ANLBlue}
  % \usefonttheme[onlymath]{serif}
  % \usetheme{Singapore}
  % \usetheme{Warsaw}
  % \usetheme{Malmoe}
  % \useinnertheme{circles}
  % \useoutertheme{infolines}
  % \useinnertheme{rounded}

  \setbeamercovered{transparent=20}
}

\usepackage[english]{babel}
\usepackage[latin1]{inputenc}
\usepackage{alltt,listings,multirow,ulem,siunitx}
\usepackage[absolute,overlay]{textpos}
\TPGrid{1}{1}
\usepackage{pdfpages}
\usepackage{ulem}
\usepackage{multimedia}
\usepackage{multicol}
\newcommand\hmmax{0}
\newcommand\bmmax{0}
\usepackage{bm}
\usepackage{comment}
\usepackage{subcaption}

% font definitions, try \usepackage{ae} instead of the following
% three lines if you don't like this look
\usepackage{mathptmx}
\usepackage[scaled=.90]{helvet}
% \usepackage{courier}
\usepackage[T1]{fontenc}
\usepackage{tikz}
\usetikzlibrary{decorations.pathreplacing}
\usetikzlibrary{shadows,arrows,shapes.misc,shapes.arrows,shapes.multipart,arrows,decorations.pathmorphing,backgrounds,positioning,fit,petri,calc,shadows,chains,matrix,mindmap}

\newcommand\vvec{\bm v}
\newcommand\bvec{\bm b}
\newcommand\bxk{\bvec_0 \times \kappa_0 \cdot \nabla}
\newcommand\delp{\nabla_\perp}

% \usepackage{pgfpages}
% \pgfpagesuselayout{4 on 1}[a4paper,landscape,border shrink=5mm]

\usepackage{JedMacros}

\newcommand{\timeR}{t_{\mathrm{R}}}
\newcommand{\timeW}{t_{\mathrm{W}}}
\newcommand{\mglevel}{\ensuremath{\ell}}
\newcommand{\mglevelcp}{\ensuremath{\mglevel_{\mathrm{cp}}}}
\newcommand{\mglevelcoarse}{\ensuremath{\mglevel_{\mathrm{coarse}}}}
\newcommand{\mglevelfine}{\ensuremath{\mglevel_{\mathrm{fine}}}}

%solution and residual
\newcommand{\vx}{\ensuremath{x}}
\newcommand{\vc}{\ensuremath{\hat{x}}}
\newcommand{\vr}{\ensuremath{r}}
\newcommand{\vb}{\ensuremath{b}}

%operators
\newcommand{\vA}{\ensuremath{A}}
\newcommand{\vP}{\ensuremath{I_H^h}}
\newcommand{\vS}{\ensuremath{S}}
\newcommand{\vR}{\ensuremath{I_h^H}}
\newcommand{\vI}{\ensuremath{\hat I_h^H}}
\newcommand{\vV}{\ensuremath{\mathbf{V}}}
\newcommand{\vF}{\ensuremath{F}}
\newcommand{\vtau}{\ensuremath{\mathbf{\tau}}}


\title{Practical Multigrid Methods \\ for Momentum Balance in Ice Sheets}
\subtitle{This talk: \url{http://59A2.org/files/20150202-LIWGMultigrid.pdf}}

\author{{\bf Jed Brown} \texttt{jed@jedbrown.org} (ANL and CU Boulder)}

% - Use the \inst command only if there are several affiliations.
% - Keep it simple, no one is interested in your street address.
% \institute
% {
%   Mathematics and Computer Science Division \\ Argonne National Laboratory
% }

\date{Land Ice Working Group, NCAR, 2015-02-02}

% This is only inserted into the PDF information catalog. Can be left
% out.
\subject{Talks}


% If you have a file called "university-logo-filename.xxx", where xxx
% is a graphic format that can be processed by latex or pdflatex,
% resp., then you can add a logo as follows:

% \pgfdeclareimage[height=0.5cm]{university-logo}{university-logo-filename}
% \logo{\pgfuseimage{university-logo}}



% Delete this, if you do not want the table of contents to pop up at
% the beginning of each subsection:
% \AtBeginSubsection[]
% {
% \begin{frame}<beamer>
%   \frametitle{Outline}
%   \tableofcontents[currentsection,currentsubsection]
% \end{frame}
% }

\AtBeginSection[]
{
  \begin{frame}<beamer>
    \frametitle{Outline}
    \tableofcontents[currentsection]
  \end{frame}
}

% If you wish to uncover everything in a step-wise fashion, uncomment
% the following command:

% \beamerdefaultoverlayspecification{<+->}

\begin{document}
\lstset{language=C}
\normalem

\begin{frame}
  \titlepage
\end{frame}

\begin{frame}{Why do we need scalable solvers?}
  \begin{itemize}
  \item Increasing resolution
    \begin{itemize}
    \item larger problem sizes
    \item more 3D effects visible
    \item more time steps $\implies$ smaller budget per time step
    \end{itemize}
  \item Sequence of simulations -- data assimilation, UQ
  \item All other costs typically linear in problem size
  \end{itemize}
\end{frame}

\begin{frame}
  \includegraphics[width=\textwidth]{figures/KeyesStrongWeak} \\
  \uncover<2>{\large \alert{The easiest way to make software scalable is to make it sequentially inefficient. -- Gropp 1999}}
\end{frame}

\begin{frame}{Is multigrid needed?}
  \begin{itemize}
  \item Long-range coupling is slow to converge using local methods
    \begin{itemize}
    \item Iteration count proportional to diameter of support of Green's functions
    \end{itemize}
  \item How local are the Green's functions?
    \begin{itemize}
    \item Columns of the inverse matrix
      \begin{equation*}
        u(x) = \int_{y \in \Omega} G(x,y) f(y)
      \end{equation*}
    \item Sticky, flat bad -- Green's functions local (and SIA is accurate)
    \item Slippery bed (ice shelf), steep topography at high resolution
    \item Pressure: surface is Dirichlet boundary condition, causes rapid decay
    \end{itemize}
  \end{itemize}
\end{frame}

\begin{frame}{Bathymetry and stickyness distribution}
  \begin{itemize}
  \item Bathymetry:
    \begin{itemize}
    \item Aspect ratio $\epsilon = [H]/[x] \ll 1$
    \item Need surface \emph{and} bed slopes to be small
    \end{itemize}
  \item Stickyness distribution:
    \begin{itemize}
    \item Limiting cases of plug flow versus vertical shear
    \item Stress ratio: $\lambda = [\tau_{xz}]/[\tau_{\text{membrane}}]$
    \item Discontinuous: frozen to slippery transition at ice stream margins
    \end{itemize}
  \item Standard approach in glaciology: \\
    Taylor expand in $\epsilon$ and sometimes $\lambda$, drop higher order terms.
    \begin{itemize}
    \item[$\lambda \gg 1$] Shallow Ice Approximation (SIA), no horizontal coupling
    \item[$\lambda \ll 1$] Shallow Shelf Approximation (SSA), 2D elliptic solve in map-plane
    \item Hydrostatic and various hybrids, 2D or 3D elliptic solves
    \end{itemize}
  \item<2> \alert{\large Bed slope is discontinuous and of order 1.}
    \begin{itemize}
    \item Taylor expansions no longer valid
    \item Numerics require high resolution, subgrid parametrization, short time steps
    \item Still get low quality results in the regions of most interest.
    \end{itemize}
  %\item<2> \alert{\LARGE Basal sliding parameters are discontinuous.}
  \end{itemize}
\end{frame}

\begin{frame}{Hydrostatic equations for ice sheet flow}
  \begin{itemize}
  \item Valid when $w_x \ll u_z$, independent of basal friction {\small (Schoof\&Hindmarsh 2010)}
  \item Eliminate $p$ and $w$ from Stokes by incompressibility:\\
    \quad 3D elliptic system for $\bm u = (u,v)$
    \begin{align*}
      - \nabla\cdot \left[ \eta
        \begin{pmatrix}
          4 u_x + 2 v_y & u_y + v_x & u_z \\
          u_y + v_x & 2 u_x + 4 v_y & v_z
        \end{pmatrix} \right] + \rho g \bar\nabla h & = 0
    \end{align*}
    \begin{align*}
      \eta(\theta,\gamma) &= \frac{B(\theta)}{2} (\gamma_0 + \gamma)^{\frac{1-\mathfrak n}{2\mathfrak n}}, \qquad \mathfrak n \approx 3 \\
      \gamma &= u_x^2 + v_y^2 + u_xv_y + \frac 1 4 (u_y+v_x)^2 + \frac 1 4 u_z^2 + \frac 1 4 v_z^2
    \end{align*}
    and slip boundary $\sigma \cdot \bm n = \beta^2 \bm u$ where
    \begin{align*}
      \beta^2(\gamma_b) &= \beta_0^2 (\epsilon_b^2 + \gamma_b)^{\frac{\mathfrak m-1}{2}}, \qquad 0 < \mathfrak m \le 1 \\
      \gamma_b &= \frac 1 2 (u^2 + v^2)
    \end{align*}
  \item $Q_1$ FEM with Newton-Krylov-Multigrid solver in PETSc: \code{src/snes/examples/tutorials/ex48.c}
  \end{itemize}
\end{frame}

\frame{
  \vspace{-1em}
  \includegraphics[width=\textwidth]{figures/THI/y-5km-m6p5l4-clip}
  \vspace{-3.5em}
  \begin{itemize}
  \item Bathymetry is essentially discontinuous on any grid
  \item Shallow ice approximation produces oscillatory solutions
  \item Nonlinear and linear solvers have major problems or fail
  \item Grid sequenced Newton-Krylov multigrid works \\
    as well as in the smooth case
  \end{itemize}
}

\begin{frame}
  \begin{figure}
    \includegraphics[width=\textwidth]{figures/THI/y-10km-m10p6l5-ew}
    \centering\caption{Grid sequenced Newton-Krylov convergence for test $Y$.
    The ``cliff'' has \SI{58}{\degree} angle in the red line ($12\times 125$ meter elements), \SI{73}{\degree} for the cyan line ($6\times 62$ meter elements).}\label{fig:testy}
  \end{figure}
\end{frame}
\begin{frame}{Strong scaling on Blue Gene/P (Shaheen)}
\begin{figure}
  \includegraphics[width=\textwidth]{figures/THI/shaheen-strong}
  \centering\caption{Strong scaling on Shaheen for different size coarse levels problems and different coarse level solvers.
    The straight lines on the strong scaling plot have slope $-1$ which is optimal.}\label{fig:shaheen-strong}
\end{figure}
\end{frame}

\begin{frame}{Weak scaling on Blue Gene/P (Shaheen)}
  \begin{figure}
  \includegraphics[width=\textwidth]{figures/THI/shaheen-weak}
  \centering\caption{Weak scaling on Shaheen with a breakdown of time spent in different phases of the solution process.
    Times are for the full grid-sequenced problem, not just the finest level solve.}\label{fig:shaheen-weak}
\end{figure}
\end{frame}

\begin{frame}
  \begin{center}
    \alert{\huge One high-accuracy solve \\[0.2em]
      costs 30 times as much \\[0.5em]
      as a residual evaluation}
  \end{center}
  \begin{center}
    about 15 to reach truncation error

    \bigskip

    \uncover<2>{\alert{\large 1000 times faster than some popular methods} \\
      e.g. Lemieux, Price, Evans, Knoll, Salinger, Holland, Payne 2011 \\ (J. Computational Physics) \\
      --- Actual speedup subject to Amdahl's Law}
    
    \bigskip

    {(Brown, Smith, Ahmadia 2013, SIAM J. Scientific Computing)}
  \end{center}
\end{frame}
\begin{frame}{Algebraic multigrid for Hydrostatic}
  \begin{itemize}
  \item Easy to use: assemble a matrix and throw it over the wall
  \item Higher setup costs, lower arithmetic intensity
  \item AMG uses heuristics to diagnose anisotropy; varies by discretization
  \item Need to represent rotational modes
    \begin{itemize}
    \item Smoothed aggregation takes a ``near null space'' (translation plus rotation)
    \end{itemize}
  \end{itemize}
\end{frame}
\begin{frame}{Eigen-analysis plugin for solver design}
  Hydrostatic ice flow (nonlinear rheology and slip conditions)
  \begin{align}\label{eq:momentum}
    - \nabla \left[ \eta
      \begin{pmatrix}
        4 u_x + 2 v_y & u_y + v_x & u_z \\
        u_y + v_x & 2 u_x + 4 v_y & v_z
      \end{pmatrix} \right] + \rho g \nabla s & = 0,
  \end{align}
  \begin{itemize}
  \item Many solvers converge easily with no-slip/frozen bed, more difficult for slippery bed (ISMIP HOM test C)
  \item Geometric MG is good: $\lambda \in [0.805, 1]$ (SISC 2013)
  \end{itemize}
  % GAMG: ./ex48 -M 10 -P 8 -da_refine 1 -thi_mat_type aij -thi_hom C -dll_append ~/petsc-eig/mpich-opt/lib/libpetsc-eig.so -ksp_plugin eig -eig_type preconditioned -eig_eps_nev 10 -eig_eps_smallest_real -eig_view_vectors_vtk -eig_st_ksp_type gmres -eig_st_ksp_rtol 1e-9 -eig_eps_monitor_lg_all -eig_eps_view -pc_type gamg
  % Eigenvalue  0 (error): 0.0268052+0i (2.34383e-09)
  % Eigenvalue  1 (error): 0.0408511+0i (9.28564e-10)
  % Eigenvalue  2 (error): 0.0431757+0i (7.35697e-10)
  % Eigenvalue  3 (error): 0.0447336+0i (6.78016e-09)
  % Eigenvalue  4 (error): 0.0490315+0i (8.74661e-09)
  % Eigenvalue  5 (error): 0.0539488+0i (9.67847e-10)
  % Eigenvalue  6 (error): 0.055815+0i (1.7793e-09)
  % Eigenvalue  7 (error): 0.0598606+0i (1.92014e-09)
  % Eigenvalue  8 (error): 0.06518+0i (3.2315e-09)
  % Eigenvalue  9 (error): 0.0669961+0i (2.8736e-09)
  \vspace{-1ex}
  \begin{figure}
    \centering
    \begin{subfigure}{0.4\textwidth}
      \centering
      \includegraphics[width=\textwidth]{figures/THI/EigenGAMG/visit0000.png}
      \caption{$\lambda_0 = 0.0268$}
    \end{subfigure}
    \begin{subfigure}{0.4\textwidth}
      \centering
      \includegraphics[width=\textwidth]{figures/THI/EigenGAMG/visit0001.png}
      \caption{$\lambda_1 = 0.0409$}
    \end{subfigure}
    % \caption{Smallest two eigenpairs for smoothed aggregation with only translational modes (but no rotational modes).}
  \end{figure}
\end{frame}


\begin{frame}{HPGMG-FE \quad \url{https://hpgmg.org}}
  \includegraphics[width=\textwidth]{figures/MG/titan-edison-supermuc-range.png}
\end{frame}


\newcommand\smallterm[1]{{\color{gray} #1}}
\begin{frame}{Conservative (non-Boussinesq) two-phase ice flow}
  Find momentum density $\rho\uu$, pressure $p$, and total energy density $E$:
  \begin{gather*}
    (\rho\uu)_t + \div (\smallterm{\rho\uu\otimes\uu} - \eta D\uu_i + p\bm 1) - \rho \bm g = 0 \\
    \rho_t + \div \rho\uu = 0 \\
    E_t + \div \big((E+p)\uu - k_T\nabla T - k_\omega\nabla\omega \big) - \eta D\uu_i\tcolon D\uu_i - \smallterm{\rho\uu\cdot\bm g} = 0
  \end{gather*}
\begin{itemize}
\item Solve for density $\rho$, ice velocity $\uu_i$, temperature $T$, and melt fraction $\omega$ using constitutive relations.
\item This and many other formulations lead to a Stokes problem
\end{itemize}
\end{frame}

\begin{frame}{The Great Solver Schism: Monolithic or Split?}
  \begin{columns}
    \begin{column}{0.5\textwidth}
      \begin{block}{Monolithic}
        \begin{itemize}
        \item Direct solvers
        \item Coupled Schwarz
        \item Coupled Neumann-Neumann \\
          (need unassembled matrices)
        \item Coupled multigrid
        \item[X] Need to understand local spectral and compatibility properties of the coupled system
        \end{itemize}
      \end{block}
    \end{column}
    \begin{column}{0.5\textwidth}
      \begin{block}{Split}
        \begin{itemize}
        \item Physics-split Schwarz \\
          (based on relaxation)
        \item Physics-split Schur \\
          (based on factorization)
          \begin{itemize}
          \item  approximate commutators \\
            SIMPLE, PCD, LSC
          \item segregated smoothers
          \item Augmented Lagrangian
          \item ``parabolization'' for stiff waves
          \end{itemize}
        \item[X] Need to understand global coupling strengths
        \end{itemize}
      \end{block}
    \end{column}
  \end{columns}
  \begin{itemize}
  \item Preferred data structures depend on which method is used.
  \item Interplay with geometric multigrid.
  \end{itemize}
\end{frame}

\begin{frame}[shrink=5]{Stokes}
  \begin{block}{Weak form of the Newton step}
    Find $(\uu,p)$ such that
    \begin{align*}
      & \int_\Omega
      \color{red}{(D\vv)^T \big[\eta\bm 1 + \eta' D\ww \otimes D\ww\big] D\uu} \\
      &\qquad\quad - {\color{green!55!black}p \nabla\cdot \vv} - {\color{blue} q \nabla\cdot \uu} = - v\cdot F(\ww) & \forall (\vv,q) \\
    \end{align*}
  \end{block}
  \vspace{-2em}
  \begin{block}{Matrix}
    \vspace{-1.5em}
    \[ \begin{bmatrix}\color{red}{A(\ww)} & {\color{green!55!black} B^T}
      \\ {\color{blue} B} & \end{bmatrix}
    \begin{pmatrix} u \\ p \end{pmatrix}
    = - \begin{pmatrix} F_u(\ww) \\ 0 \end{pmatrix} \]
  \end{block}
  \begin{block}{Block factorization}
    \vspace{-1em}
    \[\begin{bmatrix} A & B^T \\ B & \end{bmatrix} =
    \begin{bmatrix} 1 & \\ BA^{-1} & 1 \end{bmatrix}
    \begin{bmatrix} A & B^T \\ & S \end{bmatrix} =
    \begin{bmatrix} A & \\ B & S \end{bmatrix}
    \begin{bmatrix} 1 & A^{-1}B^T \\ & 1 \end{bmatrix}
    \]
    where the Schur complement is
    \[  S = -B A^{-1} B^T . \]
  \end{block}
\end{frame}


\begin{frame}{Hardware Arithmetic Intensity}
  \begin{tabular}{lc}
    \toprule
    Operation                         & Arithmetic Intensity (flops/B) \\
    \midrule
    Sparse matrix-vector product      & 1/6                  \\
    Dense matrix-vector product       & 1/4                  \\
    Unassembled matrix-vector product & $\approx 8$          \\
    High-order residual evaluation    & $> 5$                \\
    \bottomrule
  \end{tabular}

  \bigskip

  \begin{tabular}{lrrr}
    \toprule
    Processor & BW (GB/s) & Peak (GF/s) & Balanced AI (F/B) \\
    \midrule
    E5-2670 8-core      & 35   & 166  & 4.7 \\
    Magny Cours 16-core & 49   & 281  & 5.7 \\
    Blue Gene/Q node    & 43   & 205  & 4.8 \\
    Tesla M2090         & 120  & 665  & 5.5 \\
    Kepler K20Xm        & 160 & 1310 & 8.2 \\ % http://www.elekslabs.com/2012/11/nvidia-tesla-k20-benchmark-facts.html
    Xeon Phi            & 150 & 1248 & 8.3 \\
    \bottomrule
  \end{tabular}
\end{frame}


\begin{frame}{Outlook}
  \begin{itemize}
  \item Choose suitable technology
  \item Geometric multigrid is simple and has low setup cost
  \item Algebraic multigrid has higher setup, more finicky to discover anisotropy
  \item Stokes problems
    \begin{itemize}
    \item block factorization is easiest (all run-time options in PETSc)
    \item coupled MG is worth considering
    \end{itemize}
  \item Newton linearization of sliding
  \item Mind the external factors
  \end{itemize}
\end{frame}

\end{document}
