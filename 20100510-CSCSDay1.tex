%\documentclass[handout]{beamer}
\documentclass{beamer}

\mode<presentation>
{
%\usetheme{Singapore}
%\usetheme{Warsaw}
\usetheme{Malmoe}
\useinnertheme{circles}
\useoutertheme{infolines}
% \useinnertheme{rounded}

\setbeamercovered{transparent}
}

\usepackage[english]{babel}
\usepackage[latin1]{inputenc}
\usepackage{bm,textpos,alltt,listings,multirow,ulem}

% font definitions, try \usepackage{ae} instead of the following
% three lines if you don't like this look
\usepackage{mathptmx}
\usepackage[scaled=.90]{helvet}
\usepackage{courier}
\usepackage[T1]{fontenc}
\usepackage{tikz}
\usetikzlibrary[shapes.arrows,arrows,shapes.misc]

% \usepackage{pgfpages}
% \pgfpagesuselayout{4 on 1}[a4paper,landscape,border shrink=5mm]

\usepackage{bm,textpos,alltt}
\usepackage{listings}

\newcommand{\II}{\mathcal{I}}
\newcommand{\C}{\mathbb{C}}
\newcommand{\D}{\mathcal{D}}
\newcommand{\E}{\mathcal{E}}
\newcommand{\F}{\mathcal{F}}
\newcommand{\I}{\mathcal{I}}
\newcommand{\N}{\mathcal{N}}
\newcommand{\PP}{\mathcal{P}}
\newcommand{\bigO}{\mathcal{O}}
\newcommand{\R}{\mathbb{R}}
\newcommand{\kb}{\tt}
\newcommand{\blue}{\textcolor{blue}}
\newcommand{\green}{\textcolor{green!70!black}}
\newcommand{\red}{\textcolor{red}}
\newcommand{\brown}{\textcolor{brown}}
\newcommand{\cyan}{\textcolor{cyan}}
\newcommand{\magenta}{\textcolor{magenta}}
\newcommand{\yellow}{\textcolor{yellow}}
\newcommand{\mini}{\mathop{\rm minimize}}
\newcommand{\st}{\mbox{subject to }}
\newcommand{\lap}[1]{\Delta #1}
\newcommand{\grad}[1]{\nabla #1}
\renewcommand{\div}[1]{\nabla \cdot #1}
\def\code#1{{\tt #1}}
\def\shell#1{{\tt \$ #1}}
\newcommand\mtab{\hspace{\stretch{1}}}
\newcommand\ud{\,\mathrm{d}}
\newcommand\bslash{{$\backslash$}}
\newcommand\half{{\frac 1 2}}
\newcommand{\abs}[1]{\left\lvert #1 \right\rvert}
\newcommand{\bigabs}[1]{\big\lvert #1 \big\rvert}
\newcommand{\norm}[1]{\left\lVert #1 \right\rVert}
\newcommand\oneitem[1]{\begin{itemize} \item #1 \end{itemize}}
\newcommand\pp{{\mathfrak p}}

\def\Rbasic{1}
\def\Rsnesmf{2}
\def\Rsnesmflambda{3}
\def\Rsnesmfp{4}
\def\Rcolor{5}
\def\Rassemblebratu{6}
\def\Rassemblepicard{7}
\def\Rmyprealloc{8}
\def\Rnewtoncrash{9}
\def\Rnewtonbug{10}
\def\Rnewtonfix{11}

\title[PETSc day 1]{The Portable Extensible Toolkit for Scientific computing}
\subtitle{Day 1: Usage and Algorithms}

\author{Jed Brown}


% - Use the \inst command only if there are several affiliations.
% - Keep it simple, no one is interested in your street address.
\institute[ETH Z\"urich]
{

}

\date{CSCS 2010-05-10}


% This is only inserted into the PDF information catalog. Can be left
% out.
\subject{Talks}



% If you have a file called "university-logo-filename.xxx", where xxx
% is a graphic format that can be processed by latex or pdflatex,
% resp., then you can add a logo as follows:

% \pgfdeclareimage[height=0.5cm]{university-logo}{university-logo-filename}
% \logo{\pgfuseimage{university-logo}}



% Delete this, if you do not want the table of contents to pop up at
% the beginning of each subsection:
% \AtBeginSubsection[]
% {
% \begin{frame}<beamer>
% \frametitle{Outline}
% \tableofcontents[currentsection,currentsubsection]
% \end{frame}
% }

% If you wish to uncover everything in a step-wise fashion, uncomment
% the following command:

%\beamerdefaultoverlayspecification{<+->}

\begin{document}
\lstset{language=C}

\begin{frame}
\titlepage
\end{frame}

\begin{frame}
\frametitle{Outline}
\tableofcontents
% You might wish to add the option [pausesections]
\end{frame}

\begin{frame}{Requests}
  \begin{itemize}
  \item Tell me if you do not understand
  \item Tell me if an example does not work
  \item Suggest better wording, figures, organization
  \item Follow up:
    \begin{itemize}
    \item Configuration issues, private: \url{petsc-maint@mcs.anl.gov}
    \item Public questions: \url{petsc-users@mcs.anl.gov}
    \end{itemize}
  \end{itemize}
\end{frame}

\section{Introduction}
\newcommand\ganttline[4]{% line, tag, start end
   \node at (0,#1/2+.1) [anchor=base east] {#2};
   \fill[blue] (#3/\xtick-1991/\xtick,#1/2-.1) rectangle (#4/\xtick-1991/\xtick,#1/2+.1);}
\newcommand\ganttlabel[6]{% year, label, color, yloc, anchor
  \node[#3] at (#1/\xtick+#6/\xtick-1991/\xtick,#4) [anchor=#5] {#2};
  \fill[#3] (#1/\xtick-1991/\xtick,1/2-.1) rectangle (#1/\xtick-1991/\xtick+0.04,12/2+.1);}

%\begin{frame}{Timeline}
\frame{
\begin{figure}[htbp]
\def\present{2011.7}
\def\xtick{2.2}
\begin{tikzpicture}[y=-1cm]
   %\draw[help lines] (0.5,5) grid (8,0.5);
   \ganttlabel{1991}{1991}{red}{6.2}{north}{0}
   \ganttlabel{1995}{1995}{red}{6.2}{north}{0}
   \ganttlabel{2000}{2000}{red}{6.2}{north}{0}
   \ganttlabel{2005}{2005}{red}{6.2}{north}{0}
   \ganttlabel{2010}{2010}{red}{6.2}{north}{0}
   \ganttlabel{1992}{PETSc-1}{green!70!black}{0}{center}{0}
   \ganttlabel{1994.4}{MPI-1}{magenta!70!black}{-.5}{center}{0}
   \ganttlabel{1997.6}{MPI-2}{magenta!70!black}{-.5}{center}{0}
   \ganttlabel{1995.5}{PETSc-2}{green!70!black}{0}{center}{0}
   \ganttlabel{2008.9}{PETSc-3}{green!70!black}{0}{center}{0}
   \ganttline{1}{Barry}{1991}{\present}
   \ganttline{2}{Bill}{1991}{1996}
   \ganttline{3}{Lois}{1993}{2001}
   \ganttline{4}{Satish}{1997}{\present}
   \ganttline{5}{Dinesh}{1998}{2005.5}
   \ganttline{6}{Hong}{2001}{\present}
   \ganttline{7}{Kris}{2001}{2006}
   \ganttline{8}{Matt}{2001.5}{\present}
   \ganttline{9}{Victor}{2003}{2006.9}
   \ganttline{9}{}{2007.3}{2007.5}
   \ganttline{9}{}{2008.5}{2008.7}
   \ganttline{10}{Dmitry}{2005.6}{\present}
   \ganttline{11}{Lisandro}{2006.9}{\present}
   \ganttline{12}{Jed}{2009}{\present}
   \ganttline{13}{Shri}{2009.8}{\present}
\end{tikzpicture}
\end{figure}
}
%\end{frame}

\input{slides/PETSc/About.tex}
\frame{
\frametitle{The Role of PETSc}

\vspace*{\fill}
\begin{minipage}{\linewidth}
\begin{quote}
\Large Developing parallel, nontrivial PDE solvers that deliver high performance is still difficult and requires
months (or even years) of concentrated effort.

\medskip

PETSc is a toolkit that can ease these difficulties and reduce the development time, but it is not a black-box PDE
solver, nor a \blue{silver bullet}.
\end{quote}
\qquad --- Barry Smith
\end{minipage}
\vspace*{\fill}\vspace*{\fill}
}

\section{Installation}
\begin{frame}{Downloading}
\begin{itemize}
  \item \url{http://mcs.anl.gov/petsc}, download tarball
  \item We will use Mecurial in this tutorial:
  \begin{itemize}
    \item \url{http://mercurial.selenic.com}
    \item Debian/Ubuntu: \shell{aptitude install mercurial}
    \item Fedora: \shell{yum install mercurial}
  \end{itemize}
  \item Get the PETSc release
  \begin{itemize}\footnotesize
    \item \shell{\small hg clone \mtab\bslash \\
     \scriptsize http://petsc.cs.iit.edu/petsc/releases/petsc-3.1}
    \item \shell{cd petsc-3.1}
    \item \shell{\scriptsize hg clone http://petsc.cs.iit.edu/petsc/releases/BuildSystem-3.1 \mtab\bslash \\
        \qquad config/BuildSystem}
    \item Get the latest bug fixes with \shell{hg pull -{}-update}
  \end{itemize}
\end{itemize}
\end{frame}

\begin{frame}{Configuration}
\begin{block}{Basic configuration}
\begin{itemize}\footnotesize
  \item \shell{export PETSC\_DIR=\$PWD PETSC\_ARCH=mpich-gcc-dbg}
  \item \shell{./configure -{}-with-shared \mtab \bslash \\
  	\qquad\qquad -{}-with-blas-lapack-dir=/usr \mtab\bslash \\
  	\qquad\qquad -{}-download-\{mpich,ml,hypre\}}
  \item \shell{make all test}
\end{itemize}
\end{block}
\begin{itemize}
\item Other common options
  \begin{itemize}\footnotesize
  \item \item \code{-{}-with-mpi-dir=/path/to/mpi}
  \item \code{-{}-with-scalar-type=$<$real or complex$>$}
  \item \code{-{}-with-precision=$<$single,double,longdouble$>$}
  \item \code{-{}-with-64-bit-indices}
  \item \code{-{}-download-\{umfpack,mumps,scalapack,blacs,parmetis\}}
  \end{itemize}
\item reconfigure at any time with \\
  {\footnotesize \shell{mpich-gcc-dbg/conf/reconfigure-mpich-gcc-dbg.py \mtab\bslash\\
      \qquad\qquad -{}-new-options}}
\end{itemize}
\end{frame}

\frame{
\frametitle{Automatic Downloads}

\begin{itemize}
  \item Most packages can be automatically
  \begin{itemize}
    \item Downloaded
    \item Configured and Built (in {\kb \$PETSC\_DIR/externalpackages})
    \item Installed with PETSc
  \end{itemize}

  \item Currently works for
  \begin{itemize}
    \item petsc4py
    \item PETSc documentation utilities (Sowing, lgrind, c2html)
    \item BLAS, LAPACK, BLACS, ScaLAPACK, PLAPACK
    \item MPICH, MPE, Open MPI
    \item ParMetis, Chaco, Jostle, Party, Scotch, Zoltan
    \item MUMPS, Spooles, SuperLU, SuperLU\_Dist, UMFPack, pARMS
    \item PaStiX, BLOPEX, FFTW, SPRNG
    \item Prometheus, HYPRE, ML, SPAI
    \item Sundials
    \item Triangle, TetGen, FIAT, FFC, Generator
    \item HDF5, Boost
  \end{itemize}
\end{itemize}
\emph{Can also use \code{-{}-with-xxx=/path/to/your/install}}
}


\begin{frame}{An optimized build}
  \begin{itemize}
  \item \shell{\small mpich-gcc-dbg/conf/reconfigure-mpich-gcc-dbg.py \\
      PETSC\_ARCH=mpich-gcc-opt \\
      -{}-with-debugging=0 \&\& make PETSC\_ARCH=mpich-gcc-opt}
  \item What does \code{-{}-with-debugging=1} (default) do?
    \begin{itemize}
    \item Keeps debugging symbols (of course)
    \item Maintains a stack so that errors produce a full stack trace (even SEGV)
    \item Does lots of integrity checking of user input
    \item Places sentinels around allocated memory to detect memory errors
    \item Allocates related memory chunks separately (to help find memory bugs)
    \item Keeps track of and reports unused options
    \item Keeps track of and reports allocated memory that is not freed \\
      \quad \code{-malloc\_dump}
    \end{itemize}
  \end{itemize}
\end{frame}


\section{Programming model}
\begin{frame}{PETSc Structure}

\begin{center}
\includegraphics[width=5.0in]{figures/PETSc/PETScPyramid}
\end{center}

\end{frame}

\frame{
\frametitle{Flow Control for a PETSc Application}

\begin{center}
\includegraphics[width=4.0in]{figures/SNES/FlowControl}
\end{center}
}

\input{collective}
\subsection{Options Database}
\begin{frame}[fragile]{Objects}
  % \begin{lstlisting}
  %   Mat A;
  %   PetscInt m,n,M,N;
  %   MatCreate(comm,&A);
  %   MatSetSizes(A,m,n,M,N);      /* or PETSC_DECIDE */ 
  %   MatSetOptionsPrefix(A,"foo_");
  %   MatSetFromOptions(A);
  %   /* Use A */
  %   MatView(A,PETSC_VIEWER_DRAW_WORLD);
  %   MatDestroy(A);
  % \end{lstlisting}
  \begin{minted}{c}
    Mat A;
    PetscInt m,n,M,N;
    MatCreate(comm,&A);
    MatSetSizes(A,m,n,M,N);      /* or PETSC_DECIDE */ 
    MatSetOptionsPrefix(A,"foo_");
    MatSetFromOptions(A);
    /* Use A */
    MatView(A,PETSC_VIEWER_DRAW_WORLD);
    MatDestroy(A);
  \end{minted}
  \begin{itemize}
  \item \code{Mat} is an opaque object (pointer to incomplete type)
    \oneitem{Assignment, comparison, etc, are cheap}
  \item What's up with this ``Options'' stuff?
    \begin{itemize}
    \item Allows the type to be determined at runtime: \code{-foo\_mat\_type sbaij}
    \item Inversion of Control similar to ``service locator'', \\
      related to ``dependency injection''
    \item Other options (performance and semantics) can be changed at
      runtime under \code{-foo\_mat\_}
    \end{itemize}
  \end{itemize}
\end{frame}

\begin{frame}{Basic {\kb PetscObject} Usage}

\vbox{Every object in PETSc supports a basic interface}

\begin{tabular}{|r|l|}
\hline
Function & Operation \\
\hline
{\kb Create()}               & create the object \\
{\kb Get/SetName()}          & name the object \\
{\kb Get/SetType()}          & set the implementation type \\
{\kb Get/SetOptionsPrefix()} & set the prefix for all options \\
{\kb SetFromOptions()}       & customize object from the command line \\
{\kb SetUp()}                & preform other initialization \\
{\kb View()}                 & view the object \\
{\kb Destroy()}              & cleanup object allocation \\
\hline
\end{tabular}

\vbox{Also, all objects support the {\kb -help} option.}

\end{frame}


\begin{frame}{Ways to set options}
  \begin{itemize}
  \item Command line
  \item Filename in the third argument of \code{PetscInitialize()}
  \item \code{$\sim$/.petscrc}
  \item \code{\$PWD/.petscrc}
  \item \code{\$PWD/petscrc}
  \item \code{PetscOptionsInsertFile()}
  \item \code{PetscOptionsInsertString()}
  \item \code{PETSC\_OPTIONS} environment variable
  \item command line option \code{-options\_file [file]}
  \end{itemize}
\end{frame}

\begin{frame}{Try it out}
  \shell{\small cd \$PETSC\_DIR/src/snes/examples/tutorials \&\& make ex5} \\
  \begin{itemize}
  \item \shell{./ex5 -da\_grid\_x 10 -da\_grid\_y 10 -par 6.7 \\
      -snes\_monitor -\{ksp,snes\}\_converged\_reason \\
      -snes\_view}
  \item \shell{./ex5 -da\_grid\_x 10 -da\_grid\_y 10 -par 6.7 \\
      -snes\_monitor -\{ksp,snes\}\_converged\_reason \\
      -snes\_view -mat\_view\_draw -draw\_pause 0.5}
  \item \shell{./ex5 -da\_grid\_x 10 -da\_grid\_y 10 -par 6.7 \\
      -snes\_monitor -\{ksp,snes\}\_converged\_reason \\
      -snes\_view -mat\_view\_draw -draw\_pause 0.5 \\
      -pc\_type lu -pc\_factor\_mat\_ordering\_type natural}
  \item Use \code{-help} to find other ordering types
\end{itemize}
\end{frame}

\begin{frame}{In parallel}
  \begin{itemize}
  \item \shell{mpiexec -n 4 \\
      ./ex5 -da\_grid\_x 10 -da\_grid\_y 10 -par 6.7 \\
      -snes\_monitor -\{ksp,snes\}\_converged\_reason \\
      -snes\_view -sub\_pc\_type lu}
  \item How does the performance change as you
    \begin{itemize}
    \item vary the number of processes (up to 32 or 64)?
    \item increase the problem size?
    \item use an inexact subdomain solve?
    \item try an overlapping method: \code{-pc\_type asm -pc\_asm\_overlap 2}
    \item simulate a big machine: \code{-pc\_asm\_blocks 512}
    \item change the Krylov method: \code{-ksp\_type ibcgs}
    \item use algebraic multigrid: \code{-pc\_type hypre}
    \item use smoothed aggregation multigrid: \code{-pc\_type ml}
    \end{itemize}
  \end{itemize}
\end{frame}


\section{Algorithms}
\begin{frame}{Newton iteration: foundation of SNES}
  \begin{textblock}{3}(11,0)
    \includegraphics[width=\textwidth]{figures/Newton}
  \end{textblock}
  \begin{itemize}
  \item Standard form of a nonlinear system
    \[ F(u) = 0 \]
  \item Iteration
    \begin{align*}
      \text{Solve:} & \qquad J(u) w = -F(u) \\
      \text{Update:} & \qquad u^+ \gets u + w
    \end{align*}
    \item Quadratically convergent near a root: $\abs{u^{n+1}-u^*} \in \bigO\Big(\abs{u^n-u^*}^2\Big)$
    \item Picard is the same operation with a different $J(u)$
  \end{itemize}
  \begin{example}[Nonlinear Poisson]
    \begin{align*}
      F(u)=0 \quad &\sim\quad -\div\big[ (1+u^2) \grad u \big] - f = 0 \\
      J(u)w \quad &\sim\quad  -\div\big[(1+u^2)\grad w + 2uw\grad u \Big]
    \end{align*}
  \end{example}
  % \begin{example}[$\pp$-Bratu]
  %   Suppose $F$ is a discretization of
  %   \[ -\nabla \cdot \big( \eta \nabla u \big) - \lambda e^u - f = 0 \]
  %   \[\eta(\gamma) = (\epsilon^2+\gamma)^{\frac{\pfrak-2}{2}}, \qquad\quad \gamma = \half \abs{\nabla u}^2. \]
  %   Then $J(u)w$ is a discretization of
  %   \[ -\nabla \cdot \big( \eta \nabla w + \eta' (\nabla u \cdot \nabla w)\nabla u \big) - \lambda e^{u} w . \]
  % \end{example}
\end{frame}

\subsection{Linear Algebra background/theory}
\begin{frame}{Matrices}
  \begin{definition}<1->[Matrix]
    A \alert{matrix} is a linear transformation between finite dimensional vector spaces.
  \end{definition}
  \begin{definition}<2->[Forming a matrix]
    \alert{Forming} or \alert{assembling} a matrix means defining it's action in terms of entries (usually stored in a sparse format).
  \end{definition}
\end{frame}

\begin{frame}{Important matrices}
  \begin{enumerate}
  \item Sparse (e.g.~discretization of a PDE operator)
  \item \alert<2,4>{Inverse of \emph{anything} interesting $B = A^{-1}$}
  \item \alert<4>{Jacobian of a nonlinear function $J y = \lim_{\epsilon \to 0} \frac{F(x + \epsilon y) - F(x)}{\epsilon}$}
  \item \alert<2,4>{Fourier transform $\mathcal{F},\mathcal{F}^{-1}$}
  \item \alert<2,4>{Other fast transforms, e.g. Fast Multipole Method}
  \item \alert<2,4>{Low rank correction $B = A + u v^T$}
  \item \alert<2,4>{Schur complement $S = D - C A^{-1} B$}
  \item \alert<3,4>{Tensor product $A = \sum_e A_x^e \otimes A_y^e \otimes A_z^e$}
  \item \alert<3,4>{Linearization of a few steps of an explicit integrator}
  \end{enumerate}
  \begin{columns}\begin{column}{0.3\textwidth}\end{column}\begin{column}{0.7\textwidth}
  \begin{itemize}
  \item<only@2> These matrices are \alert<2>{dense}.  Never form them.
  \item<only@3>{Thes are \alert<3>{not very sparse}.}
    Don't form them.
  \item<only@4> {None of these matrices ``have entries''}
  \end{itemize}
\end{column}
\end{columns}
\end{frame}

\begin{frame}{What can we do with a matrix that doesn't have entries?}
  \begin{block}{Krylov solvers for $A x = b$}
    \begin{itemize}
    \item Krylov subspace: $\{b, Ab, A^2b, A^3b, \dotsc\}$
    \item Convergence rate depends on the spectral properties of the matrix
      \begin{itemize}
      \item Existance of small polynomials $p_n(A) < \epsilon$ where $p_n(0) = 1$.
      \item condition number $\kappa(A) = \norm{A} \norm{A^{-1}} = \sigma_{\text{max}}/\sigma_{\text{min}}$
      \item distribution of singular values, spectrum $\Lambda$, pseudospectrum $\Lambda_\epsilon$
%      \item $\epsilon$-pseudospectrum $\Lambda_\epsilon$, spectrum of $A + E$ where $\norm{E} < \epsilon$
      \end{itemize}
    \item For any popular Krylov method $\mathcal{K}$, there is a matrix
      of size $m$, such that $\mathcal{K}$ outperforms all other methods
      by a factor at least $\bigO(\sqrt{m})$~[Nachtigal et. al., 1992]%\cite{nachtigal1992fnm}
    \end{itemize}
  \end{block}
  \begin{block}{Typically...}
    \begin{itemize}
    \item The action $y \gets A x$ can be computed in $\bigO(m)$
    \item Aside from matrix multiply, the $n^{\text{th}}$ iteration requires at most $\bigO(mn)$
    \end{itemize}
  \end{block}
\end{frame}

\begin{frame}{GMRES}
  Brute force minimization of residual in $\{b,Ab,A^2b,\dotsc\}$
  \begin{enumerate}
  \item Use Arnoldi to orthogonalize the $n$th subspace, producing
    \[ A Q_n = Q_{n+1} H_n \]
  \item Minimize residual in this space by solving the overdetermined system
    \[ H_n y_n = e_1^{(n+1)} \]
    using $QR$-decomposition, updated cheaply at each iteration.
  \end{enumerate}
  Properties
  \begin{itemize}
  \item Converges in $n$ steps for all right hand sides if there exists a polynomial of degree $n$
    such that $\norm{p_n(A)} < \textit{tol}$ and $p_n(0)=1$.
  \item Residual is monotonically decreasing, robust in practice
  \item Restarted variants are used to bound memory requirements
  \end{itemize}
\end{frame}


% \section{$\pfrak$-Bratu}
\begin{frame}{The $\pfrak$-Bratu equation}
  \begin{itemize}
  \item 2-dimensional model problem
    \begin{equation*}
      -\div \big(\abs{\nabla u}^{\pfrak-2} \nabla u \big) - \lambda e^u - f = 0, \qquad 1 \le \pfrak \le \infty, \quad \lambda < \lambda_{\text{crit}}(\pfrak)
    \end{equation*}
    Singular or degenerate when $\nabla u = 0$, turning point at $\lambda_{\text{crit}}$.
  \item Regularized variant
    \begin{gather*}
      -\div (\eta \grad u) - \lambda e^u - f = 0 \\
      \eta(\gamma) = (\epsilon^2 + \gamma)^{\frac{\pfrak-2}{2}} \qquad \gamma(u) = \half \abs{\grad u}^2
    \end{gather*}
  \item Jacobian
    \begin{gather*}
      J(u) w \sim -\div \big[ (\eta \bm 1 + \eta' \nabla u \otimes \nabla u) \grad w \big] - \lambda e^u w \\
      \eta' = \frac{\pfrak-2}{2} \eta / (\epsilon^2 + \gamma)
    \end{gather*}
    Physical interpretation: conductivity tensor flattened in direction $\grad u$ %($\pfrak < 2$)
  \end{itemize}
\end{frame}

% \frame{
%   \begin{itemize}
%   \item Start with 2-Laplacian plus Bratu, define only residuals
%   \item Matrix-free Jacobians, no preconditioning \code{-snes\_mf}
%   \item \shell{hg update -r\Rsnesmf}
%   \item \shell{./pbratu -da\_grid\_x 10 -da\_grid\_y 10 -snes\_mf -snes\_monitor -ksp\_converged\_reason}
%   \item \shell{./pbratu -da\_grid\_x 20 -da\_grid\_y 20 -snes\_mf -snes\_monitor -ksp\_converged\_reason}
%   \item \shell{./pbratu -da\_grid\_x 40 -da\_grid\_y 40 -snes\_mf -snes\_monitor -ksp\_converged\_reason}
%   \end{itemize}
% }

\subsection{Nonlinear solvers: SNES}
\frame{
\frametitle{Flow Control for a PETSc Application}

\begin{center}
\includegraphics[width=4.0in]{figures/SNES/FlowControl}
\end{center}
}

\begin{frame}
\frametitle{SNES Paradigm}

The SNES interface is based upon callback functions
\begin{itemize}
  \item \code{FormFunction()}, set by \code{SNESSetFunction()}

  \medskip

  \item \code{FormJacobian()}, set by \code{SNESSetJacobian()}
\end{itemize}

\bigskip

  When PETSc needs to evaluate the nonlinear residual $F(x)$,
\begin{itemize}
  \item Solver calls the {\bf user's} function

  \medskip

  \item User function gets application state through the {\kb ctx} variable
  \begin{itemize}
    \item PETSc \emph{never} sees application data
  \end{itemize}
\end{itemize}
\end{frame}

\begin{frame}{SNES Function}

The user provided function which calculates the nonlinear residual has signature
\begin{center}
  {\small \mint{c}|PetscErrorCode (*func)(SNES snes,Vec x,Vec r,void *ctx)|}
\end{center}
\begin{itemize}
  \item[{\kb x}:] The current solution
  \item[{\kb r}:] The residual
  \item[{\kb ctx}:] The user context passed to {\kb SNESSetFunction()}
  \begin{itemize}
    \item Use this to pass application information, e.g. physical constants
  \end{itemize}
\end{itemize}

\end{frame}

\begin{frame}[fragile]{SNES Jacobian}
The user provided function which calculates the Jacobian has signature
\begin{minted}{c}
PetscErrorCode (*func)(SNES snes,Vec x,Mat *J,Mat *M,
                       MatStructure *flag,void *ctx)
\end{minted}

\begin{itemize}
  \item[{\kb x}:] The current solution
  \item[{\kb J}:] The Jacobian
  \item[{\kb M}:] The Jacobian preconditioning matrix (possibly J itself)
  \item[{\kb ctx}:] The user context passed to {\kb SNESSetFunction()}
  \begin{itemize}
    \item Use this to pass application information, e.g. physical constants
  \end{itemize}

  \item Possible {\kb MatStructure} values are:
  \begin{itemize}
    \item SAME\_NONZERO\_PATTERN
    \item DIFFERENT\_NONZERO\_PATTERN
  \end{itemize}
\end{itemize}

Alternatively, you can use
\begin{itemize}
  \item a builtin sparse finite difference approximation (``coloring'')
  \item automatic differentiation (ADIC/ADIFOR)
\end{itemize}

\end{frame}


\subsection{Structured grid distribution: DA}
\begin{frame}{Distributed Array}
  \begin{itemize}
  \item Interface for topologically structured grids
  \item Defines (topological part of) a finite-dimensional function space
    \oneitem{Get an element from this space: \code{DACreateGlobalVector()}}
  \item Provides parallel layout
  \item Refinement and coarsening
    \oneitem{\code{DARefineHierarchy()}}
  \item Ghost value coherence
    \oneitem{\code{DAGlobalToLocalBegin()}}
  \item Matrix preallocation: \oneitem{\code{DAGetMatrix()}}
  \end{itemize}
\end{frame}
\frame{
\frametitle{Ghost Values}

To evaluate a local function $f(x)$, each process requires
\begin{itemize}
  \item its local portion of the vector $x$
  \item its \cyan{ghost values}, bordering portions of $x$ owned by neighboring processes
\end{itemize}

\begin{center}
\includegraphics[width=4in]{figures/DA/GhostValues}
\end{center}
}

\begin{frame}{DA Global Numberings}

\begin{center}
\begin{tabular}{cc}
\begin{tabular}{c}
\begin{tabular}{|ccc|cc|}
\hline
\multicolumn{3}{|c|}{Proc 2} & \multicolumn{2}{c|}{Proc 3} \\
\hline
25 & 26 & 27 & 28 & 29 \\
20 & 21 & 22 & 23 & 24 \\
15 & 16 & 17 & 18 & 19 \\
\hline
10 & 11 & 12 & 13 & 14 \\
 5 &  6 &  7 &  8 &  9 \\
 0 &  1 &  2 &  3 &  4 \\
\hline
\multicolumn{3}{|c|}{Proc 0} & \multicolumn{2}{c|}{Proc 1} \\
\hline
\end{tabular} \\
Natural numbering
\end{tabular}
& 
\begin{tabular}{c}
\begin{tabular}{|ccc|cc|}
\hline
\multicolumn{3}{|c|}{Proc 2} & \multicolumn{2}{c|}{Proc 3} \\
\hline
21 & 22 & 23 & 28 & 29 \\
18 & 19 & 20 & 26 & 27 \\
15 & 16 & 17 & 24 & 25 \\
\hline
 6 &  7 &  8 & 13 & 14 \\
 3 &  4 &  5 & 11 & 12 \\
 0 &  1 &  2 &  9 & 10 \\
\hline
\multicolumn{3}{|c|}{Proc 0} & \multicolumn{2}{c|}{Proc 1} \\
\hline
\end{tabular}\\
PETSc numbering
\end{tabular}
\end{tabular}
\end{center}
\end{frame}

\begin{frame}{DA Global vs. Local Numbering}

\begin{itemize}
  \item {\bf Global}: Each vertex has a unique id belongs on a unique process

  \item {\bf Local}: Numbering includes vertices from neighboring processes
  \begin{itemize}
    \item These are called \cyan{ghost} vertices
  \end{itemize}
\end{itemize}

\begin{center}
\begin{tabular}{cc}
\begin{tabular}{c}
\begin{tabular}{|ccc|cc|}
\hline
\multicolumn{3}{|c|}{Proc 2} & \multicolumn{2}{c|}{Proc 3} \\
\hline
 X &  X &  X &  X &  X \\
 X &  X &  X &  X &  X \\
\cyan{12} & \cyan{13} & \cyan{14} & \cyan{15} &  X \\
\hline
 8 &  9 & 10 & \cyan{11} &  X \\
 4 &  5 &  6 &  \cyan{7} &  X \\
 0 &  1 &  2 &  \cyan{3} &  X \\
\hline
\multicolumn{3}{|c|}{Proc 0} & \multicolumn{2}{c|}{Proc 1} \\
\hline
\end{tabular} \\
Local numbering
\end{tabular}
& 
\begin{tabular}{c}
\begin{tabular}{|ccc|cc|}
\hline
\multicolumn{3}{|c|}{Proc 2} & \multicolumn{2}{c|}{Proc 3} \\
\hline
21 & 22 & 23 & 28 & 29 \\
18 & 19 & 20 & 26 & 27 \\
15 & 16 & 17 & 24 & 25 \\
\hline
 6 &  7 &  8 & 13 & 14 \\
 3 &  4 &  5 & 11 & 12 \\
 0 &  1 &  2 &  9 & 10 \\
\hline
\multicolumn{3}{|c|}{Proc 0} & \multicolumn{2}{c|}{Proc 1} \\
\hline
\end{tabular}\\
Global numbering
\end{tabular}
\end{tabular}
\end{center}
\end{frame}

\frame{
\frametitle{DA Vectors}

\begin{itemize}
  \item The DA object contains only layout (topology) information
  \begin{itemize}
    \item All field data is contained in PETSc {\kb Vec}s
  \end{itemize}

  \item Global vectors are parallel 
  \begin{itemize}
    \item Each process stores a unique local portion
    \item {\kb DACreateGlobalVector(DA da, Vec *gvec)}
  \end{itemize}

  \item Local vectors are sequential (and usually temporary)
  \begin{itemize}
    \item Each process stores its local portion plus ghost values
    \item {\kb DACreateLocalVector(DA da, Vec *lvec)}
    \item includes ghost values!
  \end{itemize}

  \item Coordinate vectors store the mesh geometry
    \begin{itemize}
    \item \code{DAGetCoordinates(DA,Vec *coords)}
    \item Can be manipulated with their own DA \\
      \code{DAGetCoordinateDA(DA,DA *cda)}
    \end{itemize}
\end{itemize}
}

\begin{frame}{Updating Ghosts}

Two-step process enables overlapping\\
computation and communication

\medskip

\begin{itemize}
  \item {\kb DAGlobalToLocalBegin(da, gvec, mode, lvec)}
  \begin{itemize}
    \item {\kb gvec} provides the data 
    \item {\kb mode} is either {\kb INSERT\_VALUES} or {\kb ADD\_VALUES}
    \item {\kb lvec} holds the local and ghost values
  \end{itemize}

  \item {\kb DAGlobalToLocalEnd(da, gvec, mode, lvec)}
  \begin{itemize}
    \item Finishes the communication
  \end{itemize}
\end{itemize}

\medskip

The process can be reversed with {\kb DALocalToGlobal()}.
\end{frame}

\frame{
\frametitle{DMDA Stencils}

Both the \blue{box} stencil and \blue{star} stencil are available.

\begin{center}
\includegraphics[width=4.5in]{figures/DA/Stencils}
\end{center}
}

\frame{
\frametitle{Creating a DMDA}

{\small \kb DMDACreate2d(comm, xbdy, ybdy, type, M, N, m, n, \\
\qquad\qquad\qquad  dof, s, lm[], ln[], DA *da)}
\begin{columns}\begin{column}{0.15\textwidth}\end{column}\begin{column}{0.85\textwidth}
  \begin{itemize}
  \item[{\kb xbdy,ybdy}:] Specifies periodicity or ghost cells
  \begin{itemize}
    \item {\kb DMDA\_BOUNDARY\_NONE}, {\kb DMDA\_BOUNDARY\_GHOSTED}, {\kb DMDA\_BOUNDARY\_MIRROR}, {\kb DMDA\_BOUNDARY\_PERIODIC}
  \end{itemize}
  \item[{\kb type}:] Specifies stencil
  \begin{itemize}
    \item {\kb DMDA\_STENCIL\_BOX} or {\kb DMDA\_STENCIL\_STAR}
  \end{itemize}
  \item[{\kb M,N}:] Number of grid points in x/y-direction
  \item[{\kb m,n}:] Number of processes in x/y-direction
  \item[{\kb dof}:] Degrees of freedom per node
  \item[{\kb s}:] The stencil width
  \item[{\kb lm,ln}:] Alternative array of local sizes
  \begin{itemize}
    \item Use {\kb PETSC\_NULL} for the default
  \end{itemize}
\end{itemize}
\end{column} \end{columns}
}

\begin{frame}{Working with the local form}
  Wouldn't it be nice if we could just write our code for the natural numbering?
  \begin{itemize}
  \item Yes, that's what \code{DMDAVecGetArray()} is for.
  \item Also, DMDA offers \emph{local} callback functions
    \begin{itemize}
    \item \code{FormFunctionLocal()}, set by \code{DMDASetLocalFunction()}
      
      \medskip
      
    \item \code{FormJacobianLocal()}, set by \code{DMDASetLocalJacobian()}
    \end{itemize}

    \bigskip

  \item When PETSc needs to evaluate the nonlinear residual $F(x)$,
    \begin{itemize}
    \item Each process evaluates the local residual

      \medskip

    \item PETSc assembles the global residual automatically
      \begin{itemize}
      \item Uses \code{DMLocalToGlobal()} method
      \end{itemize}
    \end{itemize}
  \end{itemize}
\end{frame}

\frame{
\frametitle{DA Local Function}

The user provided function which calculates the nonlinear residual in 2D has signature \\
{\kb PetscErrorCode (*lfunc)(DMDALocalInfo *info, \\
  \qquad \qquad \qquad Field **x, Field **r, void *ctx)}
\begin{columns}\begin{column}{0.15\textwidth}\end{column}\begin{column}{0.85\textwidth}
\begin{itemize}
  \item[{\kb info}:] All layout and numbering information
  \item[{\kb x}:] The current solution
  \begin{itemize}
    \item Notice that it is a multidimensional array
  \end{itemize}
  \item[ {\kb r}:] The residual
  \item[ {\kb ctx}:] The user context passed to {\kb DMSetApplicationContext()} or to SNES
\end{itemize}
\end{column}\end{columns}

\bigskip

The local DMDA function is activated by calling
\begin{itemize}
\item {\kb SNESSetDM(snes,dm);}
\item {\kb DMDASNESSetFunctionLocal(dm, InsertMode imode, lfunc, ctx);} \\
  where {\kb imode} is {\kb INSERT\_VALUES} or {\kb ADD\_VALUES}
\end{itemize}
}

\begin{frame}[fragile]
\frametitle{Bratu Residual Evaluation}

\begin{equation*}
  -\Delta u - \lambda e^{u} = 0
\end{equation*}

\small
% \begin{verbatim}
\begin{minted}{c}
BratuResidualLocal(DALocalInfo *info,Field **x,Field **f,
                   UserCtx *user)
{
  /* Not Shown: Handle boundaries */
  /* Compute over the interior points */
  for(j = info->ys; j < info->ys+info->ym; j++) {
    for(i = info->xs; i < info->xs+info->xm; i++) {
      u       = x[j][i];
      u_xx    = (2.0*u - x[j][i-1] - x[j][i+1])*hydhx;
      u_yy    = (2.0*u - x[j-1][i] - x[j+1][i])*hxdhy;
      f[j][i] = u_xx + u_yy - hx*hy*lambda*exp(u);
    }
  }
}
\end{minted}
%\end{verbatim}

\begin{center}\small
\$PETSC\_DIR/src/snes/examples/tutorials/ex5.c
\end{center}
\end{frame}


\begin{frame}{Start with 2-Laplacian plus Bratu nonlinearity}
  \begin{itemize}
  \item Matrix-free Jacobians, no preconditioning \code{-snes\_mf}
  \item \shell{hg update -r\Rsnesmflambda}
  \item \shell{./pbratu -da\_grid\_x 10 -da\_grid\_y 10 \\ -lambda 6.7 -snes\_mf -snes\_monitor -ksp\_converged\_reason}
  \item \shell{./pbratu -da\_grid\_x 20 -da\_grid\_y 20 \\ -lambda 6.7 -snes\_mf -snes\_monitor -ksp\_converged\_reason}
  \item \shell{./pbratu -da\_grid\_x 40 -da\_grid\_y 40 \\ -lambda 6.7 -snes\_mf -snes\_monitor -ksp\_converged\_reason}
  \item Watch linear and nonlinear convergence
  \end{itemize}
\end{frame}

\begin{frame}{Add $\pfrak$ nonlinearity}
  \begin{itemize}
  \item Matrix-free Jacobians, no preconditioning \code{-snes\_mf}
  \item \shell{hg update -r\Rsnesmfp}
  \item \shell{./pbratu -da\_grid\_x 10 -da\_grid\_y 10 \\ -lambda 1 -p 1.3 -snes\_mf -snes\_monitor -ksp\_converged\_reason}
  \item \shell{./pbratu -da\_grid\_x 20 -da\_grid\_y 20 \\ -lambda 1 -p 1.3 -snes\_mf -snes\_monitor -ksp\_converged\_reason}
  \item \shell{./pbratu -da\_grid\_x 40 -da\_grid\_y 40 \\ -lambda 1 -p 1.3 -snes\_mf -snes\_monitor -ksp\_converged\_reason}
  \item Watch linear and nonlinear convergence
  \end{itemize}
\end{frame}

\subsection{Preconditioning}
\begin{frame}{Preconditioning}
  \begin{block}{Idea: improve the conditioning of the Krylov operator}
    \begin{itemize}
    \item Left preconditioning
      \vspace{-1em}
      \begin{gather*}
        (P^{-1} A) x = P^{-1} b \\
        \{ P^{-1} b, (P^{-1}A) P^{-1} b, (P^{-1}A)^2 P^{-1} b, \dotsc \}
      \end{gather*}
    \item Right preconditioning
      \vspace{-1em}
      \begin{gather*}
        (A P^{-1}) P x = b \\
        \{ b, (A P^{-1}b, (A P^{-1})^2b, \dotsc \}
      \end{gather*}
    \item The product $P^{-1}A$ or $A P^{-1}$ is \emph{not} formed.
    \end{itemize}
  \end{block}
  \begin{definition}[Preconditioner]
      A \emph{preconditioner} $\PP$ is a method for constructing a
matrix (just a linear function, not assembled!)  $P^{-1} = \PP(A,A_p)$
using a matrix $A$ and extra information $A_p$, such that the spectrum
of $P^{-1}A$ (or $A P^{-1}$) is well-behaved.
    \end{definition}
\end{frame}

\begin{frame}{Preconditioning}
  \begin{definition}[Preconditioner]
      A \emph{preconditioner} $\PP$ is a method for constructing a matrix
      $P^{-1} = \PP(A,A_p)$ using a matrix $A$ and extra information $A_p$, such that
      the spectrum of $P^{-1}A$ (or $A P^{-1}$) is well-behaved.
    \end{definition}
    \begin{itemize}
    \item $P^{-1}$ is dense, $P$ is often not available and is not needed
    \item $A$ is rarely used by $\PP$, but $A_p = A$ is common
    \item $A_p$ is often a sparse matrix, the ``preconditioning matrix''
    \item Matrix-based: Jacobi, Gauss-Seidel, SOR, ILU(k), LU
    \item Parallel: Block-Jacobi, Schwarz, Multigrid, FETI-DP, BDDC
    \item Indefinite: Schur-complement, Domain Decomposition, Multigrid
    \end{itemize}
\end{frame}


\begin{frame}{Questions to ask when you see a matrix}
  \begin{enumerate}
  \item What do you want to do with it?
    \begin{itemize}
    \item Multiply with a vector
    \item Solve linear systems or eigen-problems
    \end{itemize}
  \item How is the conditioning/spectrum?
    \begin{itemize}
    \item distinct/clustered eigen/singular values?
    \item symmetric positive definite ($\sigma(A) \subset \R^+$)?
    \item nonsymmetric definite ($\sigma(A) \subset \{z \in \C : \Re [z] > 0 \}$)?
    \item indefinite?
    \end{itemize}
  \item How dense is it?
    \begin{itemize}
    \item block/banded diagonal?
    \item sparse unstructured?
    \item denser than we'd like?
    \end{itemize}
  \item Is there a better way to compute $Ax$?
  \item Is there a different matrix with similar spectrum, but nicer properties?
  \item \alert<2>{How can we precondition $A$?}
  \end{enumerate}
\end{frame}

\begin{frame}{Relaxation}
  Split into lower, diagonal, upper parts: \alert{$ A = L + D + U $}
  \begin{block}{Jacobi}
    Cheapest preconditioner: $P^{-1} = D^{-1}$
  \end{block}
  \begin{block}{Successive over-relaxation (SOR)}
    \begin{gather*}
      \left(L + \frac 1 \omega D\right) x_{n+1} = \left[\left(\frac
          1\omega-1\right)D - U\right] x_n + \omega b \\
      P^{-1} = \text{$k$ iterations starting with $x_0=0$}
    \end{gather*}
    \begin{itemize}
    \item Implemented as a sweep
    \item $\omega = 1$ corresponds to Gauss-Seidel
    \item Very effective at removing high-frequency components of residual
    \end{itemize}
  \end{block}
\end{frame}

\begin{frame}[shrink=5]{Factorization}
  Two phases
  \begin{itemize}
  \item symbolic factorization: find where fill occurs, only uses sparsity pattern
  \item numeric factorization: compute factors
  \end{itemize}
  \begin{block}{LU decomposition}
    \begin{itemize}
    \item Ultimate preconditioner
    \item Expensive, for $m\times m$ sparse matrix with bandwidth $b$, traditionally requires $\bigO(mb^2)$ time and $\bigO(mb)$ space.
      \begin{itemize}
      \item Bandwidth scales as $m^{\frac{d-1}{d}}$ in $d$-dimensions
      \item Optimal in 2D: $\bigO(m \log m)$ space, $\bigO(m^{3/2})$ time
      \item Optimal in 3D: $\bigO(m^{4/3})$ space, $\bigO(m^2)$ time
      \end{itemize}
    \item Symbolic factorization is problematic in parallel
    \end{itemize}
  \end{block}
  \begin{block}{Incomplete LU}
    \begin{itemize}
    \item Allow a limited number of levels of fill:
      ILU($k$)
    \item Only allow fill for entries that exceed threshold: ILUT
    \item Very poor scaling in parallel, don't bother beyond 8 PEs.
    \item No guarantees
    \end{itemize}
  \end{block}
\end{frame}

\begin{frame}{1-level Domain decomposition}
  Domain size $L$, subdomain size $H$, element size $h$
  \begin{block}{Overlapping/Schwarz}
    \begin{itemize}\item Solve Dirichlet problems on overlapping
      subdomains
    \item No overlap: $\textit{its} \in \bigO\big( \frac{L}{\sqrt{Hh}} \big)$
    \item Overlap $\delta$: $\textit{its} \in \big( \frac L {\sqrt{H\delta}} \big)$
    \end{itemize}
  \end{block}
  \begin{block}{Neumann-Neumann}
    \begin{itemize}
    \item Solve Neumann problems on non-overlapping subdomains
    \item $\textit{its} \in \bigO\big( \frac{L}{H}(1+\log\frac H h) \big)$
    \item Tricky null space issues (floating subdomains)
    \end{itemize}
  \end{block}
  \begin{itemize}
  \item Multilevel variants knock off the leading $\frac L H$
  \item Both overlapping and nonoverlapping with this bound
  \end{itemize}
  % \begin{block}{BDDC and FETI-DP}
  %   \begin{itemize}
  %   \item Neumann problems on subdomains with
  %     coarse grid correction
  %   \item $\textit{its} \in \bigO\big(1 + \log\frac H h \big)$
  %   \end{itemize}
  %   \includegraphics[width=0.7\textwidth]{bddc}
  % \end{block}
\end{frame}

\begin{frame}[shrink=5]{Multigrid}
  \begin{block}{Hierarchy: Interpolation and restriction operators}
    \begin{equation*}
    \II^\uparrow : X_{\text{coarse}} \to X_{\text{fine}} \qquad
    \II^\downarrow :  X_{\text{fine}} \to X_{\text{coarse}}
  \end{equation*}
  \end{block}
  \begin{itemize}
  \item Geometric: define problem on multiple levels, use grid to compute hierarchy
  \item Algebraic: define problem only on finest level, use matrix structure to build hierarchy
  \end{itemize}
  \begin{block}{Galerkin approximation}
    Assemble this matrix: $A_{\text{coarse}} = \II^\downarrow A_{\text{fine}} \II^\uparrow$
  \end{block}
  \begin{block}{Application of multigrid preconditioner ($V$-cycle)}
    \begin{itemize}
    \item Apply pre-smoother on fine level (any preconditioner)
    \item Restrict residual to coarse level with $\II^\downarrow$
    \item Solve on coarse level $A_{\text{coarse}} x = r$
    \item Interpolate result back to fine level with $\II^\uparrow$
    \item Apply post-smoother on fine level (any preconditioner)
    \end{itemize}
  \end{block}
\end{frame}

\begin{frame}{Multigrid convergence properties}
  \begin{itemize}
  \item Textbook: $P^{-1}A$ is spectrally equivalent to identity
  \item Most theory applies to SPD systems
  \item nonsymmetric (e.g. advection, shallow water, Euler) \\
    with low-order upwind discretization
  \item Good when coefficients in problem are smooth
    \begin{itemize}
    \item large jumps and anisotropy are harder
    \item build low-energy interpolants
    \item use stronger smoothers
    \end{itemize}
  \item Aggressive coarsening is critical, especially in parallel
  \item Most theory uses SOR smoothers, ILU often more robust
  \item Coarsest component usually solved semi-redundantly with direct solver
  \item Multilevel Schwarz is an extreme case of aggressive coarsening
    and strong smoothers.  Exotic interpolants for robustness.
  \end{itemize}
\end{frame}


\begin{frame}{Finite Difference Jacobians}
  PETSc can compute and explicitly store a Jacobian via 1st-order FD
  \begin{itemize}
  \item Dense
    \begin{itemize}
    \item Activated by {\kb -snes\_fd}
    \item Computed by {\kb SNESDefaultComputeJacobian()}
    \end{itemize}
  \item Sparse via colorings
    \begin{itemize}
    \item Activated by {\kb -snes\_fd\_color} (default when no Jacobian set and using DM)
    \item Coloring is created by {\kb MatFDColoringCreate()}
    \item Computed by {\kb SNESDefaultComputeJacobianColor()}
    \end{itemize}
  \end{itemize}
  Can also use Matrix-free Newton-Krylov via 1st-order FD
  \begin{itemize}
  \item Activated by {\kb -snes\_mf} without preconditioning
  \item Activated by {\kb -snes\_mf\_operator} with user-defined preconditioning
    \begin{itemize}
    \item Uses preconditioning matrix from {\kb SNESSetJacobian()}
    \end{itemize}
  \end{itemize}
\end{frame}


\begin{frame}{Add finite difference Jacobian by coloring}
  \begin{itemize}
  \item \shell{hg update -r\Rcolor}
  \item \shell{./pbratu -da\_grid\_x 10 -da\_grid\_y 10 \\ -lambda 1 -p 1.3 -snes\_fd -snes\_monitor -ksp\_converged\_reason}
  \item \shell{./pbratu -da\_grid\_x 10 -da\_grid\_y 10 \\ -lambda 1 -p 1.3 -fd\_jacobian -snes\_monitor -ksp\_converged\_reason}
  \item \shell{./pbratu -da\_grid\_x 10 -da\_grid\_y 10 \\ -lambda 1 -p 1.3 -fd\_jacobian -snes\_monitor -ksp\_converged\_reason}
  \item Try some different preconditioners (\code{jacobi,sor,asm,hypre,ml})
  \item Try changing the physical parameters
  \item May need \code{-mat\_fd\_type ds}
  \end{itemize}
\end{frame}

\subsection{Matrix Redux}
\begin{frame}{Matrices, redux}
What are PETSc matrices?
\begin{itemize}
\item Linear operators on finite dimensional vector spaces. (snarky)
  \item<2> Fundamental objects for storing stiffness matrices and Jacobians
  \item<2> Each process locally owns a contiguous set of rows
  \item<2> Supports many data types
  \begin{itemize}
    \item AIJ, Block AIJ, Symmetric AIJ, Block Diagonal, etc.
  \end{itemize}
  \item<2> Supports structures for many packages
  \begin{itemize}
    \item MUMPS, Spooles, SuperLU, UMFPack, Hypre
  \end{itemize}
\end{itemize}
\end{frame}

\begin{frame}{How do I create matrices?}

\begin{itemize}
  \item {\kb MatCreate(MPI\_Comm, Mat *)}
  \item {\kb MatSetSizes(Mat, int m, int n, int M, int N)}
  \item {\kb MatSetType(Mat, MatType typeName)}
  \item {\kb MatSetFromOptions(Mat)}
  \begin{itemize}
    \item Can set the type at runtime
  \end{itemize}
  \item {\kb MatMPIBAIJSetPreallocation(Mat,...)} %(Mat,int bs,int d_nz,const int
                                %d_nnz[],int o_nz,const int o_nnz[])}
    \oneitem{important for assembly performance, more tomorrow}
  \item {\kb MatSetBlockSize(Mat, int bs)}
    \oneitem{for vector problems}
  \item {\kb MatSetValues(Mat,...)}
  \begin{itemize}
    \item {\bf MUST} be used, but does automatic communication
    \item \cfunc|MatSetValuesLocal|, \cfunc|MatSetValuesStencil|
    \item \cfunc|MatSetValuesBlocked|
  \end{itemize}
\end{itemize}
\end{frame}

\begin{frame}{Matrix Polymorphism}

The PETSc {\kb Mat} has a single user interface,
\begin{itemize}
  \item Matrix assembly
  \begin{itemize}
    \item {\kb MatSetValues()}
  \end{itemize}

  \item Matrix-vector multiplication
  \begin{itemize}
    \item {\kb MatMult()}
  \end{itemize}

  \item Matrix viewing
  \begin{itemize}
    \item {\kb MatView()}
  \end{itemize}
\end{itemize}
but multiple underlying implementations.
\begin{itemize}
  \item AIJ, Block AIJ, Symmetric Block AIJ,
  \item Dense
  \item Matrix-Free
  \item etc.
\end{itemize}
A matrix is defined by its {\red{interface}}, not by its {\blue{data structure}}.

\end{frame}

\begin{frame}{Matrix Assembly}

\begin{itemize}
  \item A three step process
  \begin{itemize}
    \item Each process sets or adds values
    \item Begin communication to send values to the correct process
    \item Complete the communication
  \end{itemize}
  \item {\kb MatSetValues(Mat A, m, rows[], n, cols[], values[], mode)}
  \begin{itemize}
    \item {\kb mode} is either INSERT\_VALUES or ADD\_VALUES
    \item Logically dense block of values
  \end{itemize}
% TODO picture of MatSetValues
  \item Two phase assembly allows overlap of communication and computation
  \begin{itemize}
    \item {\kb MatAssemblyBegin(Mat m, type)}
    \item {\kb MatAssemblyEnd(Mat m, type)}
    \item {\kb type} is either MAT\_FLUSH\_ASSEMBLY or MAT\_FINAL\_ASSEMBLY
  \end{itemize}
  \item<2-> For vector problems\\
    {\kb MatSetValuesBlocked(Mat A, m, rows[], \\
      \qquad\qquad n, cols[], values[], mode)}
  \item<2-> The same assembly code can build matrices of different format
    \begin{itemize}
    \item choose format at run-time.
    \end{itemize}
\end{itemize}

\end{frame}

\begin{frame}[fragile]{One Way to Set the Elements of a Matrix}{Simple 3-point stencil for 1D Laplacian}
\small
\begin{minted}{c}
v[0] = -1.0; v[1] = 2.0; v[2] = -1.0;
if (rank == 0) {
  for(row = 0;  row < N; row++) {
    cols[0] = row-1; cols[1] = row; cols[2] = row+1;
    if (row == 0) {
      MatSetValues(A,1,&row,2,&cols[1],&v[1],INSERT_VALUES);
    } else if (row == N-1) {
      MatSetValues(A,1,&row,2,cols,v,INSERT_VALUES);
    } else {
      MatSetValues(A,1,&row,3,cols,v,INSERT_VALUES);
    }
  }
}
MatAssemblyBegin(A,MAT_FINAL_ASSEMBLY);
MatAssemblyEnd(A,MAT_FINAL_ASSEMBLY);
\end{minted}
\end{frame}

\input{slides/PETSc/Integration/EfficientMatrixAssembly.tex}
\begin{frame}{Why Are PETSc Matrices That Way?}

\begin{itemize}
  \item No one data structure is appropriate for all problems
  \begin{itemize}
    \item Blocked and diagonal formats provide significant performance benefits
    \item PETSc has many formats and makes it easy to add new data structures
  \end{itemize}

  \item Assembly is difficult enough without worrying about partitioning
  \begin{itemize}
    \item PETSc provides parallel assembly routines
    \item Achieving high performance still requires making most operations local
    \item However, programs can be incrementally developed.
    \item {\kb MatPartitioning} and {\kb MatOrdering} can help
  \end{itemize}

  \item Matrix decomposition in contiguous chunks is simple
  \begin{itemize}
    \item Makes interoperation with other codes easier
    \item For other ordering, PETSc provides ``Application Orderings'' (AO)
  \end{itemize}
\end{itemize}

\end{frame}


\begin{frame}{$\pfrak$-Bratu assembly}
  \begin{itemize}
  \item Use \code{DAGetMatrix()} (can skip matrix preallocation details)
  \item Start by just assembling Bratu nonlinearity
  \item \shell{hg update -r\Rassemblebratu}
  \item Watch \code{-snes\_converged\_reason}, what happens for $p \ne 2$?
  \item Solve exactly with the preconditioner \code{-pc\_type lu}
  \item Try \code{-snes\_mf\_operator}
  \end{itemize}
\end{frame}

\begin{frame}{$\pfrak$-Bratu assembly}
  \begin{itemize}
  \item We need to assemble the $\pfrak$ part
    \begin{align*}
      J(u)w \quad &\sim\quad -\div \big[(\eta\bm 1 + \eta' \grad u \otimes \grad u) \grad w\big]
    \end{align*}
  \item Second part is scary, but what about just using $-\div(\eta\grad w)$?
  \item \shell{hg update -r\Rassemblepicard}
  \item Solve exactly with the preconditioner \code{-pc\_type lu}
  \item Try \code{-snes\_mf\_operator}
  \item Refine the grid, change $\pfrak$
  \item Try algebraic multigrid if available: \code{-pc\_type [ml,hypre]}
  \end{itemize}
\end{frame}

\begin{frame}{Does the preconditioner need Newton linearization?}
  \begin{itemize}
  \item The anisotropic part looks messy.  \\
    \alert{Is it worth writing the code to assemble that part?}
  \item Easy profiling: \code{-log\_summary}
  \item Observation: the Picard linearization uses a ``star'' (5-point)
    stencil while Newton linearization needs a ``box'' (9-point) stencil.
  \item Add support for reduced preallocation with a command-line option
  \item \shell{hg update -r\Rmyprealloc}
  \item Compare performance (time, memory, iteration count) of
    \begin{itemize}
    \item 5-point Picard-linearization assembled by hand
    \item 5-point Newton-linearized Jacobian computed by coloring
    \item 9-point Newton-linearized Jacobian computed by coloring
    \end{itemize}
  \end{itemize}
\end{frame}

\subsection{Debugging}
\begin{frame}{Maybe it's not worth it, but let's assemble it anyway}
  \begin{itemize}
  \item \shell{hg update -r\Rnewtoncrash}
  \item Crash!
  \item You were using the the debug PETSC\_ARCH, right?
  \item Launch the debugger
  \begin{itemize}
    \item {\kb -start\_in\_debugger  [gdb,dbx,noxterm]}
    \item {\kb -on\_error\_attach\_debugger [gdb,dbx,noxterm]}
  \end{itemize}

  \item Attach the debugger only to some parallel processes
  \begin{itemize}
    \item {\kb -debugger\_nodes 0,1}
  \end{itemize}

  \item Set the display (often necessary on a cluster)
  \begin{itemize}
    \item {\kb -display :0}
  \end{itemize}
\end{itemize}
\end{frame}  

\begin{frame}{Debugging Tips}

\begin{itemize}
  \item Put a breakpoint in {\kb PetscError()} to catch errors as they occur

  \item PETSc tracks memory overwrites at both ends of arrays
  \begin{itemize}
    \item The {\kb CHKMEMQ} macro causes a check of all allocated memory

    \item Track memory overwrites by bracketing them with {\kb CHKMEMQ}
  \end{itemize}

  \item PETSc checks for leaked memory
  \begin{itemize}
    \item Use {\kb PetscMalloc()} and {\kb PetscFree()} for all allocation

    \item Print unfreed memory on {\kb PetscFinalize()} with {\kb -malloc\_dump}
  \end{itemize}

  \item Simply the best tool today is \red{Valgrind}
  \begin{itemize}
    \item It checks memory access, cache performance, memory usage, etc.

    \item \href{http://www.valgrind.org}{http://www.valgrind.org}

    \item Pass {\kb -malloc 0} to PETSc when running under Valgrind
    \item Might need {\kb -{}-trace-children=yes} when running under MPI
    \item {\kb -{}-track-origins=yes} handy for uninitialized memory
  \end{itemize}
\end{itemize}

\end{frame}


\begin{frame}{Memory error is gone now}
\begin{itemize}
  \item \shell{hg update -r\Rnewtonbug}
  \item Run with {-snes\_mf\_operator -pc\_type lu}
  \item Do you see quadratic convergence?
  \item<2-> Hmm, there must be a bug in that mess, where is it?
  \end{itemize}
\end{frame}

\begin{frame}{SNES Test}
  \begin{itemize}
  \item PETSc can compute a finite difference Jacobian and compare it to yours
  \item \code{-snes\_type test}
    \oneitem{Is the difference significant?}
  \item \code{-snes\_type test -snes\_test\_display}
    \oneitem{Are the entries in the star stencil correct?}
  \item Find which line has the typo
  \item \shell{hg update -r\Rnewtonfix}
  \item Check with \code{-snes\_type test} 
  \item and \code{-snes\_mf\_operator -pc\_type lu}
  \end{itemize}
\end{frame}


% \subsection{Matrix Preallocation}
% \begin{frame}{Matrix Memory Preallocation}
\begin{itemize}
  \item PETSc sparse matrices are dynamic data structures
  \begin{itemize}
    \item can add additional nonzeros freely
  \end{itemize}

  \item Dynamically adding many nonzeros
  \begin{itemize}
    \item requires additional memory allocations
    \item requires copies
    \item can kill performance
  \end{itemize}

  \item Memory preallocation provides
  \begin{itemize}
    \item the freedom of dynamic data structures
    \item good performance
  \end{itemize}

  \item Easiest solution is to replicate the assembly code
  \begin{itemize}
    \item Remove computation, but preserve the indexing code
    \item Store set of columns for each row
  \end{itemize}

  \item Call preallocation routines for all datatypes
  \begin{itemize}
    \item {\kb MatSeqAIJSetPreallocation()}
    \item {\kb MatMPIBAIJSetPreallocation()}
    \item Only the relevant data will be used.  Or {\kb MatXAIJSetPreallocation()}
  \end{itemize}
\end{itemize}
\end{frame}

\begin{frame}{Sequential Sparse Matrices}
{\kb MatSeqAIJSetPreallocation(Mat A, int nz, int nnz[])}
\hbox{\qquad
\vbox{
\begin{itemize}
  \item[nz:] expected number of nonzeros in any row
  \item[{nnz[i]}:] expected number of nonzeros in row i
\end{itemize}
}
}

\begin{center}
%\includegraphics[width=2in]{figures/Mat/serialSparseMatrix_bcsstk32}
\includegraphics[width=.5\textwidth]{figures/EllipRCMSquare}
\end{center}
\end{frame}

\begin{frame}{Parallel Sparse Matrix}
\begin{itemize}
  \item Each process locally owns a submatrix of contiguous global rows
  \item Each submatrix consists of diagonal and off-diagonal parts
\end{itemize}

\begin{center}
\includegraphics[width=3.5in]{figures/Mat/parallelSparseMatrix}
\end{center}

\begin{itemize}
  \item {\kb MatGetOwnershipRange(Mat A,int *start,int *end)}
  \begin{itemize}
    \item[{\kb start}:] first locally owned row of global matrix
    \item[{\kb end-1}:] last locally owned row of global matrix
  \end{itemize}
\end{itemize}
\end{frame}

\begin{frame}{Parallel Sparse Matrices}
\hbox{
\quad
\vbox{
{\kb MatMPIAIJSetPreallocation(Mat A, int dnz, int dnnz[], \\
  \qquad \qquad int onz, int onnz[])}
\begin{itemize}
  \item[dnz:] expected number of nonzeros in any row in the diagonal block
  \item[{dnnz[i]}:] expected number of nonzeros in row i in the diagonal block
  \item[onz:] expected number of nonzeros in any row in the offdiagonal portion
  \item[{onnz[i]}:] expected number of nonzeros in row i in the offdiagonal portion
\end{itemize}
}
}
\end{frame}

\begin{frame}{Verifying Preallocation}
\begin{itemize}
  \item Use runtime options \\
    {\kb -mat\_new\_nonzero\_location\_err} \\
    {\kb -mat\_new\_nonzero\_allocation\_err}
  \item Use {\kb -ksp\_view} or {\kb -snes\_view} and look for \\
    {\kb total number of mallocs used during MatSetValues calls =0}
  \item Use runtime option {\kb -info}
  \item Output: \\
{\kb
  $[$proc \#$]$ Matrix size: \%d X \%d; storage space: \%d unneeded, \%d used \\
  $[$proc \#$]$ Number of mallocs during MatSetValues( )  is \%d
}
\end{itemize}

\bigskip

\begin{center}
\includegraphics[width=5in]{figures/PETSc/logInfoOutput}
\end{center}
\end{frame}

\begin{frame}{Block and symmetric formats}
  \begin{itemize}
  \item BAIJ
    \begin{itemize}
    \item Like AIJ, but uses static block size
    \item Preallocation is like AIJ, but just one index per block
    \end{itemize}
  \item SBAIJ
    \begin{itemize}
    \item Only stores upper triangular part
    \item Preallocation needs number of nonzeros in upper triangular \\
      parts of on- and off-diagonal blocks
    \end{itemize}
  \item \code{MatSetValuesBlocked()}
    \begin{itemize}
    \item Better performance with blocked formats
    \item Also works with scalar formats, if \code{MatSetBlockSize()} was called
    \item Variants \code{MatSetValuesBlockedLocal()}, \code{MatSetValuesBlockedStencil()}
    \item Change matrix format at runtime, don't need to touch assembly code
    \end{itemize}
  \end{itemize}
\end{frame}


% \input{thi}
% \begin{frame}{SNES Example}
\framesubtitle{Driven Cavity}
\hbox{
\includegraphics[width=.4\textwidth]{figures/SNES/DrivenCavitySolution}
\vbox{
\begin{itemize}
  \item Velocity-vorticity formulation
  \item Flow driven by lid and/or bouyancy
  \item Logically regular grid
  \begin{itemize}
    \item Parallelized with {\kb DA}
  \end{itemize}
  \item Finite difference discretization
  \item Authored by David Keyes
\end{itemize}
}
}
\end{frame}

\begin{frame}[fragile]{SNES Example}
\framesubtitle{Driven Cavity Application Context}
\begin{lstlisting}
/* Collocated at each node */
typedef struct {
  PetscScalar u,v,omega,temp;
} Field;

typedef struct {
       /* physical parameters */
   PassiveReal lidvelocity,prandtl,grashof;
       /* color plots of the solution */
   PetscTruth  draw_contours;
} AppCtx;
\end{lstlisting}
\end{frame}

\begin{frame}[fragile]{SNES Example}
\framesubtitle{Driven Cavity Residual Evaluation}
\small
\begin{lstlisting}
DrivenCavityFunction(SNES snes, Vec X, Vec F, void *ptr) {
  AppCtx        *user = (AppCtx *) ptr;
  /* local starting and ending grid points */
  PetscInt       istart, iend, jstart, jend;
  PetscScalar    *f;             /* local vector data */
  PetscReal      grashof = user->grashof;  
  PetscReal      prandtl = user->prandtl;
  PetscErrorCode ierr;

  /* Code to communicate nonlocal ghost point data */
  VecGetArray(F, &f);
  /* Code to compute local function components */
  VecRestoreArray(F, &f);
  return 0;
}
\end{lstlisting}
\end{frame}

\begin{frame}[fragile]{SNES Example}
\framesubtitle{Better Driven Cavity Residual Evaluation}

\small
\begin{lstlisting}
PetscErrorCode DrivenCavityFuncLocal(DALocalInfo *info,
                    Field **x,Field **f,void *ctx) {
  /* Handle boundaries */
  /* Compute over the interior points */
  for(j = info->ys; j < info->ys+info->ym; j++) {
    for(i = info->xs; i < info->xs+info->xm; i++) {
      /* convective coefficients for upwinding */
      /* U velocity */
      u          = x[j][i].u;
      uxx        = (2.0*u - x[j][i-1].u - x[j][i+1].u)*hydhx;
      uyy        = (2.0*u - x[j-1][i].u - x[j+1][i].u)*hxdhy;
      upw        = 0.5*(x[j+1][i].omega-x[j-1][i].omega)*hx
      f[j][i].u  = uxx + uyy - upw;
      /* V velocity, Omega, Temperature */
}}}
\end{lstlisting}

\begin{center}\small
\$PETSC\_DIR/src/snes/examples/tutorials/ex19.c
\end{center}
\end{frame}

% 
\begin{frame}[fragile]{Running the driven cavity}
\footnotesize
  \begin{itemize}
  \item \code{./ex19 -lidvelocity 100 -grashof \alert{1e2} -da\_grid\_x 16 -da\_grid\_y 16 -snes\_monitor -snes\_view -da\_refine 2}
\only<2>{{\scriptsize \color{green!30!black} \tt \\
lid velocity = 100, prandtl \# = 1, grashof \# = 1000 \\
  0 SNES Function norm 7.682893957872e+02 \\
  1 SNES Function norm 6.574700998832e+02 \\
  2 SNES Function norm 5.285205210713e+02 \\
  3 SNES Function norm 3.770968117421e+02 \\
  4 SNES Function norm 3.030010490879e+02 \\
  5 SNES Function norm 2.655764576535e+00 \\
  6 SNES Function norm 6.208275817215e-03 \\
  7 SNES Function norm 1.191107243692e-07 \\
Number of SNES iterations = 7
}}
  \item \code{./ex19 -lidvelocity 100 -grashof \alert{1e4} -da\_grid\_x 16 -da\_grid\_y 16 -snes\_monitor -snes\_view -da\_refine 2}
\only<3>{{\scriptsize \color{green!30!black} \tt \\
lid velocity = 100, prandtl \# = 1, grashof \# = 10000 \\
  0 SNES Function norm 7.854040793765e+02 \\
  1 SNES Function norm 6.630545177472e+02 \\
  2 SNES Function norm 5.195829874590e+02 \\
  3 SNES Function norm 3.608696664876e+02 \\
  4 SNES Function norm 2.458925075918e+02 \\
  5 SNES Function norm 1.811699413098e+00 \\
  6 SNES Function norm 4.688284580389e-03 \\
  7 SNES Function norm 4.417003604737e-08 \\
Number of SNES iterations = 7
}}
  \item \code{./ex19 -lidvelocity 100 -grashof \alert{1e5} -da\_grid\_x 16 -da\_grid\_y 16 -snes\_monitor -snes\_view -da\_refine 2 -pc\_type lu}
\only<4>{{\scriptsize \color{green!30!black} \tt \\
lid velocity = 100, prandtl \# = 1, grashof \# = 100000 \\
  0 SNES Function norm 1.809960438828e+03 \\
  1 SNES Function norm 1.678372489097e+03 \\
  2 SNES Function norm 1.643759853387e+03 \\
  3 SNES Function norm 1.559341161485e+03 \\
  4 SNES Function norm 1.557604282019e+03 \\
  5 SNES Function norm 1.510711246849e+03 \\
  6 SNES Function norm 1.500472491343e+03 \\
  7 SNES Function norm 1.498930951680e+03 \\
  8 SNES Function norm 1.498440256659e+03 \\
  ...
}}
  \item<5-> Uh oh, we have convergence problems
  \item<5-> Does \code{-snes\_grid\_sequence} help?
  \end{itemize}
\end{frame}

\begin{frame}{Why isn't SNES converging?}
  \begin{itemize}
  \item The Jacobian is wrong (maybe only in parallel)
    \oneitem{Check with \code{-snes\_type test} and \code{-snes\_mf\_operator -pc\_type lu}}
  \item The linear system is not solved accurately enough
    \begin{itemize}
    \item Check with \code{-pc\_type lu}
    \item Check \code{-ksp\_monitor\_true\_residual}, try right preconditioning
    \end{itemize}
  \item The Jacobian is singular with inconsistent right side
    \begin{itemize}
    \item Use \code{MatNullSpace} to inform the \code{KSP} of a known null space
    \item Use a different Krylov method or preconditioner
    \end{itemize}
  \item The nonlinearity is just really strong
    \begin{itemize}
    \item Run with \code{-snes\_linesearch\_monitor}
    \item Try using trust region instead of line search \code{-snes\_type newtontr}
    \item Try grid sequencing if possible
    \item Use a continuation
    \end{itemize}
  \end{itemize}
\end{frame}

\begin{frame}{Globalizing the lid-driven cavity}
  \begin{itemize}
  \item Pseudotransient continuation ($\Psi tc$)
    \begin{itemize}
    \item Do linearly implicit backward-Euler steps, driven by
      steady-state residual
    \item Residual-based adaptive controller retains quadratic
      convergence in terminal phase
    \end{itemize}
  \item Implemented in \code{src/ts/examples/tutorials/ex26.c}
  \item {\footnotesize \shell{./ex26 -ts\_type pseudo -lidvelocity 100 -grashof 1e5 -da\_grid\_x 16 -da\_grid\_y 16 -ts\_monitor}}
\only<2>{{\tiny \color{green!30!black} \tt \\
16x16 grid, lid velocity = 100, prandtl \# = 1, grashof \# = 100000 \\
0 TS dt 0.03125 time 0 \\
1 TS dt 0.034375 time 0.034375 \\
2 TS dt 0.0398544 time 0.0742294 \\
3 TS dt 0.0446815 time 0.118911 \\
4 TS dt 0.0501182 time 0.169029 \\
... \\
24 TS dt 3.30306 time 11.2182 \\
25 TS dt 8.24513 time 19.4634 \\
26 TS dt 28.1903 time 47.6537 \\
27 TS dt 371.986 time 419.64 \\
28 TS dt 379837 time 380257 \\
29 TS dt 3.01247e+10 time 3.01251e+10 \\
30 TS dt 6.80049e+14 time 6.80079e+14 \\
CONVERGED\_TIME at time 6.80079e+14 after 30 steps
}}
  \item<3> Make the method nonlinearly implicit: \code{-snes\_type ls -snes\_monitor}
    \oneitem{Compare required number of linear iterations}
  \item<3> Try error-based adaptivity: \code{-ts\_type rosw -ts\_adapt\_dt\_min 1e-4}
  \item<3> Try increasing \code{-lidvelocity}, \code{-grashof}, and problem size
  \item<3> Coffey, Kelley, and Keyes, \emph{Pseudotransient continuation and differential algebraic equations}, SIAM J. Sci. Comp, 2003.
  \end{itemize}
\end{frame}

\begin{frame}{Nonlinear multigrid (full approximation scheme)}
\footnotesize
  \begin{itemize}
  \item V-cycle structure, but use nonlinear relaxation and skip the matrices
  \item \code{./ex19 -da\_refine 4 -snes\_monitor -snes\_type \alert{nrichardson} -npc\_fas\_levels\_snes\_type gs -npc\_fas\_levels\_snes\_gs\_sweeps 3 -npc\_snes\_type fas -npc\_fas\_levels\_snes\_type gs -npc\_snes\_max\_it 1 -npc\_snes\_fas\_smoothup 6 -npc\_snes\_fas\_smoothdown 6 -lidvelocity 100 -grashof 4e4}
\only<2>{{\tiny \color{green!30!black} \tt \\
lid velocity = 100, prandtl \# = 1, grashof \# = 40000 \\
  0 SNES Function norm 1.065744184802e+03 \\
  1 SNES Function norm 5.213040454436e+02 \\
  2 SNES Function norm 6.416412722900e+01 \\
  3 SNES Function norm 1.052500804577e+01 \\
  4 SNES Function norm 2.520004680363e+00 \\
  5 SNES Function norm 1.183548447702e+00 \\
  6 SNES Function norm 2.074605179017e-01 \\
  7 SNES Function norm 6.782387771395e-02 \\
  8 SNES Function norm 1.421602038667e-02 \\
  9 SNES Function norm 9.849816743803e-03 \\
 10 SNES Function norm 4.168854365044e-03 \\
 11 SNES Function norm 4.392925390996e-04 \\
 12 SNES Function norm 1.433224993633e-04 \\
 13 SNES Function norm 1.074357347213e-04 \\
 14 SNES Function norm 6.107933844115e-05 \\
 15 SNES Function norm 1.509756087413e-05 \\
 16 SNES Function norm 3.478180386598e-06 \\
Number of SNES iterations = 16
}}
\item \code{./ex19 -da\_refine 4 -snes\_monitor -snes\_type \alert{ngmres} -npc\_fas\_levels\_snes\_type gs -npc\_fas\_levels\_snes\_gs\_sweeps 3 -npc\_snes\_type fas -npc\_fas\_levels\_snes\_type gs -npc\_snes\_max\_it 1 -npc\_snes\_fas\_smoothup 6 -npc\_snes\_fas\_smoothdown 6 -lidvelocity 100 -grashof 4e4}
\only<3>{{\tiny \color{green!30!black} \tt \\
lid velocity = 100, prandtl \# = 1, grashof \# = 40000 \\
  0 SNES Function norm 1.065744184802e+03 \\
  1 SNES Function norm 9.413549877567e+01 \\
  2 SNES Function norm 2.117533223215e+01 \\
  3 SNES Function norm 5.858983768704e+00 \\
  4 SNES Function norm 7.303010571089e-01 \\
  5 SNES Function norm 1.585498982242e-01 \\
  6 SNES Function norm 2.963278257962e-02 \\
  7 SNES Function norm 1.152790487670e-02 \\
  8 SNES Function norm 2.092161787185e-03 \\
  9 SNES Function norm 3.129419807458e-04 \\
 10 SNES Function norm 3.503421154426e-05 \\
 11 SNES Function norm 2.898344063176e-06 \\
Number of SNES iterations = 11}}
  \end{itemize}
\end{frame}


% \begin{frame}{Summary}
% \vskip0pt plus.5fill
% \begin{block}{Today}
% \begin{itemize}
%   \item Build, run, and modify examples
%   \item Familiar with debugging capability
%   \item Understand some general properties of the algorithms
%   \item Check Jacobians, diagnose failure to converge
%   \item Get help from the options database and manual pages
% \end{itemize}
% \end{block}

% \begin{block}{Tomorrow}
%   \begin{itemize}
%   \item Performance bottlenecks
%   \item Optimizing memory access patterns
%   \item Parallel scalability
%   \item Integration with your existing projects/lower-level interfaces
%   \item Tricks for hard problems/extending PETSc
%   \end{itemize}
% \end{block}
% \end{frame}

% \begin{frame}{Splitting for Multiphysics}
  \begin{equation*}
    \begin{bmatrix}
      A & B \\ C & D
    \end{bmatrix}
    \begin{bmatrix}
      x \\ y
    \end{bmatrix}
    =
    \begin{bmatrix}
      f \\ g
    \end{bmatrix}
  \end{equation*}
  \begin{itemize}\item Relaxation:
    \code{-pc\_fieldsplit\_type [additive,multiplicative,symmetric\_multiplicative]}
    \begin{equation*}
      \begin{bmatrix}
        A & \\  & D
      \end{bmatrix}^{-1} \qquad 
      \begin{bmatrix}
        A & \\ C & D
      \end{bmatrix}^{-1} \qquad
      \begin{bmatrix}
        A & \\  & \bm 1
      \end{bmatrix}^{-1}
      \left(
        \bm 1 -
        \begin{bmatrix}
          A & B \\ & \bm 1
        \end{bmatrix}
        \begin{bmatrix}
          A & \\ C & D
        \end{bmatrix}^{-1}
      \right)
    \end{equation*}
    \begin{itemize}
    \item Gauss-Seidel inspired, works when fields are loosely coupled
    \end{itemize}
  \item Factorization: \code{-pc\_fieldsplit\_type schur}
    \begin{align*}
      \begin{bmatrix}
        A & B \\ & S
      \end{bmatrix}^{-1}
      \begin{bmatrix}
        1 & \\ CA^{-1} & 1
      \end{bmatrix}^{-1}, \qquad
      S = D - C A^{-1} B
    \end{align*}
    \begin{itemize}
    \item robust (exact factorization), can often drop lower block
    \item how to precondition $S$ which is usually dense?
      \begin{itemize}
      \item interpret as differential operators, use approximate commutators
      \end{itemize}
    \end{itemize}
  \end{itemize}
\end{frame}

% \input{Q1Q1Stokes}
% \section{Performance}
% \input{bandwidth}

\end{document}
