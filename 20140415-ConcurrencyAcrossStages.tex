% \documentclass[handout]{beamer}
\documentclass{beamer}

\mode<presentation>
{
  \usetheme{ANLBlue}
  % \usefonttheme[onlymath]{serif}
  % \usetheme{Singapore}
  % \usetheme{Warsaw}
  % \usetheme{Malmoe}
  % \useinnertheme{circles}
  % \useoutertheme{infolines}
  % \useinnertheme{rounded}

  \setbeamercovered{transparent=20}
}

\usepackage[english]{babel}
\usepackage[latin1]{inputenc}
\usepackage{alltt,listings,multirow,ulem,siunitx}
\usepackage[absolute,overlay]{textpos}
\TPGrid{1}{1}
\usepackage{pdfpages}
\usepackage{ulem}
\usepackage{multimedia}
\usepackage{multicol}
\newcommand\hmmax{0}
\newcommand\bmmax{0}
\usepackage{bm}
\usepackage{comment}
\usepackage{subcaption}

% font definitions, try \usepackage{ae} instead of the following
% three lines if you don't like this look
\usepackage{mathptmx}
\usepackage[scaled=.90]{helvet}
% \usepackage{courier}
\usepackage[T1]{fontenc}
\usepackage{tikz}
\usetikzlibrary{decorations.pathreplacing}
\usetikzlibrary{shadows,arrows,shapes.misc,shapes.arrows,shapes.multipart,arrows,decorations.pathmorphing,backgrounds,positioning,fit,petri,calc,shadows,chains,matrix}

\newcommand\vvec{\bm v}
\newcommand\bvec{\bm b}
\newcommand\bxk{\bvec_0 \times \kappa_0 \cdot \nabla}
\newcommand\delp{\nabla_\perp}

% \usepackage{pgfpages}
% \pgfpagesuselayout{4 on 1}[a4paper,landscape,border shrink=5mm]

\usepackage{JedMacros}

\newcommand{\timeR}{t_{\mathrm{R}}}
\newcommand{\timeW}{t_{\mathrm{W}}}
\newcommand{\mglevel}{\ensuremath{\ell}}
\newcommand{\mglevelcp}{\ensuremath{\mglevel_{\mathrm{cp}}}}
\newcommand{\mglevelcoarse}{\ensuremath{\mglevel_{\mathrm{coarse}}}}
\newcommand{\mglevelfine}{\ensuremath{\mglevel_{\mathrm{fine}}}}

%solution and residual
\newcommand{\vx}{\ensuremath{x}}
\newcommand{\vc}{\ensuremath{\hat{x}}}
\newcommand{\vr}{\ensuremath{r}}
\newcommand{\vb}{\ensuremath{b}}

%operators
\newcommand{\vA}{\ensuremath{A}}
\newcommand{\vP}{\ensuremath{I_H^h}}
\newcommand{\vS}{\ensuremath{S}}
\newcommand{\vR}{\ensuremath{I_h^H}}
\newcommand{\vI}{\ensuremath{\hat I_h^H}}
\newcommand{\vV}{\ensuremath{\mathbf{V}}}
\newcommand{\vF}{\ensuremath{F}}
\newcommand{\vtau}{\ensuremath{\mathbf{\tau}}}


\title{Concurrency across stages: Fast solvers for implicit Runge-Kutta}
\author{{\bf Jed Brown} \texttt{jedbrown@mcs.anl.gov} (ANL and CU Boulder) \\
  Collaborators in this work: \\
  \quad Debojyoti Ghosh (ANL), Mark Adams (LBL), Matt Knepley (UChicago)
}

% - Use the \inst command only if there are several affiliations.
% - Keep it simple, no one is interested in your street address.
% \institute
% {
%   Mathematics and Computer Science Division \\ Argonne National Laboratory
% }

\date{2014-04-15}

% This is only inserted into the PDF information catalog. Can be left
% out.
\subject{Talks}


% If you have a file called "university-logo-filename.xxx", where xxx
% is a graphic format that can be processed by latex or pdflatex,
% resp., then you can add a logo as follows:

% \pgfdeclareimage[height=0.5cm]{university-logo}{university-logo-filename}
% \logo{\pgfuseimage{university-logo}}



% Delete this, if you do not want the table of contents to pop up at
% the beginning of each subsection:
% \AtBeginSubsection[]
% {
% \begin{frame}<beamer>
%   \frametitle{Outline}
%   \tableofcontents[currentsection,currentsubsection]
% \end{frame}
% }

\AtBeginSection[]
{
  \begin{frame}<beamer>
    \frametitle{Outline}
    \tableofcontents[currentsection]
  \end{frame}
}

% If you wish to uncover everything in a step-wise fashion, uncomment
% the following command:

% \beamerdefaultoverlayspecification{<+->}

\begin{document}
\lstset{language=C}
\normalem

\begin{frame}{MatTAIJ: ``sparse'' tensor product matrices}
  \begin{gather*}
    G = I_n \otimes S + J \otimes T
  \end{gather*}
  \begin{itemize}
  \item $J$ is parallel and sparse, $S$ and $T$ are small and dense
  \item More general than multiple RHS (multivectors)
  \item Compare to multiple right hand sides in row-major
  \item Runge-Kutta systems have $T = I_s$ (permuted from Butcher method)
  \item Stream $J$ through cache once, same efficiency as multiple RHS
  \item Unintrusive compared to spatial-domain vectorization or $s$-step
  \end{itemize}
\end{frame}

\begin{frame}{Runge-Kutta methods}
  \begin{gather*}
    \dot u = F(u) \\
    \underbrace{
    \begin{pmatrix}
      y_1 \\
      \vdots \\
      y_s
    \end{pmatrix}}_Y =
    u^{n} + h
    \underbrace{
    \begin{bmatrix}
      a_{11} & \dotsb & a_{1s} \\
      \vdots & \ddots & \vdots \\
      a_{s1} & \dotsb & a_{ss}
    \end{bmatrix}}_A
    F
    \begin{pmatrix}
      y_1 \\
      \vdots \\
      y_s
    \end{pmatrix} \\
    u^{n+1} = b^T Y
  \end{gather*}
  \begin{itemize}
  \item General framework for one-step methods
  \item Diagonally implicit: $A$ lower triangular, stage order $\le 2$
  \item Singly diagonally implicit: all $A_{ii}$ equal, reuse solver setup, stage order $\le 1$
  \item If $A$ is a general full matrix, all stages are coupled, ``implicit RK''
  \end{itemize}
\end{frame}

\begin{frame}{Implicit Runge-Kutta}
  \begin{center}
    \begin{tabular}{>{$}c<{$} | >{$}c<{$} >{$}c<{$} >{$}c<{$}}
      \frac 1 2 - \frac{\sqrt{15}}{10} & \frac{5}{36} & \frac 2 9 - \frac{\sqrt{15}}{15} & \frac{5}{36} - \frac{\sqrt{15}}{30} \\
      \frac 1 2 & \frac{5}{36} + \frac{\sqrt{15}}{24} & \frac 2 9 & \frac{5}{36} - \frac{\sqrt{15}}{24} \\
      \frac 1 2 - \frac{\sqrt{15}}{10} & \frac{5}{36} + \frac{\sqrt{15}}{30} & \frac 2 9 + \frac{\sqrt{15}}{15} & \frac{5}{36} \\[4pt]
      \hline
      \vspace{4pt}
      & \frac{5}{18} & \frac 4 9 & \frac{5}{18}
    \end{tabular}
  \end{center}
  \begin{itemize}
  \item Implicit Runge-Kutta methods have excellent accuracy and stability properties
  \item Gauss methods with $s$ stages
    \begin{itemize}
    \item order $2s$, $(s,s)$ Pad\'e approximation to the exponential
    \item $A$-stable, symplectic
    \end{itemize}
  \item Radau (IIA) methods with $s$ stages
    \begin{itemize}
    \item order $2s-1$, $A$-stable, $L$-stable
    \end{itemize}
  \item Lobatto (IIIC) methods with $s$ stages
    \begin{itemize}
    \item order $2s-2$, $A$-stable, $L$-stable, self-adjoint
    \end{itemize}
  \item Stage order $s$ or $s+1$    
  \end{itemize}
\end{frame}

\begin{frame}{Method of Butcher (1976) and Bickart (1977)}
  \begin{itemize}
  \item Newton linearize Runge-Kutta system
    \begin{equation*}
      Y = u^{n} + h A F(Y)
    \end{equation*}
  \item Solve linear system with tensor product operator
    \begin{equation*}
      S \otimes I_n + I_s \otimes J
    \end{equation*}
    where $S = (hA)^{-1}$ is $s\times s$ dense, $J = -\partial F(u)/\partial u$ sparse
  \item SDC (2000) is Gauss-Seidel with low-order corrector
  \item Butcher/Bickart method: diagonalize $S = X \Lambda X^{-1}$
    \begin{itemize}
    \item $\Lambda \otimes I_n + I_s \otimes J$
    \item $s$ decoupled solves
    \end{itemize}
  \item Problem: $X$ is exponentially ill-conditioned wrt. $s$
  \item We avoid diagonalization
    \begin{itemize}
    \item Same convergence properties
    \item Stages coupled through one register transpose at spatial-point granularity
    \end{itemize}
  \end{itemize}
\end{frame}

\begin{frame}{Blue Gene/Q test}
  128 nodes, 16 procs/node, small diffusion problem, CG/Jacobi solver
  \begin{tabular}{lrrr}
    \toprule
    Method & order & nsteps & time \\
    \midrule
    Gauss 4 & 8 & 10 & 3.4345e-01 \\
    Gauss 2 & 4 & 20 & 7.6320e-01 \\
    Gauss 1 & 2 & 40 & 1.1052e+00 \\
    \bottomrule
  \end{tabular}
\end{frame}


\begin{frame}{Implicit Runge-Kutta for advection}
  \begin{table}
    \centering
    \caption{Total number of iterations (communications or accesses of $J$) to solve linear advection to $t=1$ on a $1024$-point grid using point-block Jacobi preconditioning of implicit Runge-Kutta matrix.
      The relative algebraic solver tolerance is $10^{-8}$.}\label{tab:irk-advection}
    \begin{tabular}{lrrr}
      \toprule
      Family & Stages & Order & Iterations \\
      \midrule
      Crank-Nicolson/Gauss & 1 & 2 & 3627 \\
      Gauss & 2 & 4 & 2560 \\
      Gauss & 4 & 8 & 1735 \\
      Gauss & 8 & 16 & 1442 \\
      \bottomrule
    \end{tabular}
  \end{table}
  \begin{itemize}
  \item Naive centered-difference discretization
  \end{itemize}
\end{frame}

\begin{frame}{Toward AMG for IRK/tensor-product systems}
  \begin{columns}
    \begin{column}{0.3\textwidth}
      \includegraphics[width=\textwidth]{figures/TS/Gauss8-Eig.png}
    \end{column}
    \begin{column}{0.7\textwidth}
      \begin{itemize}
      \item Start with $\hat R = R \otimes I_s$, $\hat P = P \otimes I_s$
        \begin{gather*}
          G_{\text{coarse}} = \hat R (I_n \otimes S + J \otimes I_s) \hat P
        \end{gather*}
      \item Imaginary component slows convergence
      \item Idea: incrementally rotate eigenvalues toward real axis on coarse levels \\
        Enlangga and Nabben \emph{On a multilevel Krylov method for the Helmholtz equation preconditioned by shifted Laplacian}
      \end{itemize}
    \end{column}
  \end{columns}
\end{frame}

\begin{frame}{We really want multigrid}
  \begin{itemize}
  \item Prolongation: $P \otimes I_s$
  \item Coarse operator: $I_n \otimes S + (R J P) \otimes I_s$
  \item Larger time steps
  \item GMRES(2)/point-block Jacobi smoothing
  \item FGMRES outer
  \end{itemize}
  \begin{tabular}{lrrrrrr}
    \toprule
    Method & order & nsteps & Tot.~Krylov & Krylov/stage & Krylov/step \\
    \midrule
    Gauss 1 & 2 & 16 & 82 & 5.1 & 5.1 \\
    Gauss 2 & 4 & 8 & 64 & 4 & 8\\
    Gauss 4 & 8 & 4 & 44 & 2.75 & 11 \\
    Gauss 8 & 16 & 2 & 42 & 2.63 & 21 \\
    \bottomrule
  \end{tabular}
\end{frame}

\end{document}
