% \documentclass[handout]{beamer}
\documentclass{beamer}

\mode<presentation>
{
  \usetheme{ANLBlue}
  % \usefonttheme[onlymath]{serif}
  % \usetheme{Singapore}
  % \usetheme{Warsaw}
  % \usetheme{Malmoe}
  % \useinnertheme{circles}
  % \useoutertheme{infolines}
  % \useinnertheme{rounded}

  \setbeamercovered{transparent=20}
}

\usepackage[english]{babel}
\usepackage[latin1]{inputenc}
\usepackage{alltt,listings,multirow,ulem,siunitx}
\usepackage[absolute,overlay]{textpos}
\TPGrid{1}{1}
\usepackage{pdfpages}
\usepackage{ulem}
\usepackage{multimedia}
\usepackage{multicol}
\newcommand\hmmax{0}
\newcommand\bmmax{0}
\usepackage{bm}
\usepackage{comment}
%\usepackage{subcaption}

% font definitions, try \usepackage{ae} instead of the following
% three lines if you don't like this look
\usepackage{mathptmx}
\usepackage[scaled=.90]{helvet}
% \usepackage{courier}
\usepackage[T1]{fontenc}
\usepackage{tikz}
\usetikzlibrary{decorations.pathreplacing}
\usetikzlibrary{shadows,arrows,shapes.misc,shapes.arrows,shapes.multipart,arrows,decorations.pathmorphing,backgrounds,positioning,fit,petri,calc,shadows,chains,matrix}

\newcommand\vvec{\bm v}
\newcommand\bvec{\bm b}
\newcommand\bxk{\bvec_0 \times \kappa_0 \cdot \nabla}
\newcommand\delp{\nabla_\perp}

% \usepackage{pgfpages}
% \pgfpagesuselayout{4 on 1}[a4paper,landscape,border shrink=5mm]

% \usepackage{JedMacros}

\newcommand{\timeR}{t_{\mathrm{R}}}
\newcommand{\timeW}{t_{\mathrm{W}}}
\newcommand{\mglevel}{\ensuremath{\ell}}
\newcommand{\mglevelcp}{\ensuremath{\mglevel_{\mathrm{cp}}}}
\newcommand{\mglevelcoarse}{\ensuremath{\mglevel_{\mathrm{coarse}}}}
\newcommand{\mglevelfine}{\ensuremath{\mglevel_{\mathrm{fine}}}}

%solution and residual
\newcommand{\vx}{\ensuremath{x}}
\newcommand{\vc}{\ensuremath{\hat{x}}}
\newcommand{\vr}{\ensuremath{r}}
\newcommand{\vb}{\ensuremath{b}}

%operators
\newcommand{\vA}{\ensuremath{A}}
\newcommand{\vP}{\ensuremath{I_H^h}}
\newcommand{\vS}{\ensuremath{S}}
\newcommand{\vR}{\ensuremath{I_h^H}}
\newcommand{\vI}{\ensuremath{\hat I_h^H}}
\newcommand{\vV}{\ensuremath{\mathbf{V}}}
\newcommand{\vF}{\ensuremath{F}}
\newcommand{\vtau}{\ensuremath{\mathbf{\tau}}}


\title{Finite-element FAS GMG}
\author{{\bf Jed Brown} \texttt{jedbrown@mcs.anl.gov}}

% - Use the \inst command only if there are several affiliations.
% - Keep it simple, no one is interested in your street address.
\institute
{ \small
  Mathematics and Computer Science Division, Argonne National Laboratory \\
  {\small Department of Computer Science, University of Colorado Boulder}
}

\date{HPCG \\ 2014-03-25}

% This is only inserted into the PDF information catalog. Can be left
% out.
\subject{Talks}


% If you have a file called "university-logo-filename.xxx", where xxx
% is a graphic format that can be processed by latex or pdflatex,
% resp., then you can add a logo as follows:

% \pgfdeclareimage[height=0.5cm]{university-logo}{university-logo-filename}
% \logo{\pgfuseimage{university-logo}}



% Delete this, if you do not want the table of contents to pop up at
% the beginning of each subsection:
% \AtBeginSubsection[]
% {
% \begin{frame}<beamer>
%   \frametitle{Outline}
%   \tableofcontents[currentsection,currentsubsection]
% \end{frame}
% }

% If you wish to uncover everything in a step-wise fashion, uncomment
% the following command:

% \beamerdefaultoverlayspecification{<+->}

\begin{document}
\lstset{language=C}
\normalem

\begin{frame}{Objectives}
  \begin{itemize}
  \item Scale-free problem definition
    \begin{itemize}
    \item Do not adjudicate ``cores'', ``processors''
    \end{itemize}
  \item Scale-free communication
    \begin{itemize}
    \item Globally-coupled problems need long-range messages
    \end{itemize}
  \item Quantify dynamic range
    \begin{itemize}
    \item $N_{\text{max}}$ to fill memory, find $N_{1/2}$ at half efficiency
    \item ``dynamic range'' $N_{\text{max}}/N_{1/2}$
    \end{itemize}
  \item Stress conflict between vectorization and cache reuse
  \item 
  \end{itemize}
\end{frame}

\begin{frame}{Finite-element FAS}
  \begin{itemize}
  \item 
  \end{itemize}
\end{frame}

\begin{frame}{Newton linearization of $p$-Laplacian using $Q_2$ elements}
  \begin{equation*}
    -\nabla\cdot \Big[ (\eta + \eta' \nabla u \otimes \nabla u) \nabla w \Big] = f
  \end{equation*}
  \begin{itemize}
  \item Asymptotic data motion
    \begin{itemize}
    \item Coordinates: 24 \texttt{double}s per element (81 without reuse)
    \item State+Residual: 16 \texttt{double}s per element (54 without reuse)
    \item Coefficients: $4\cdot 27$ \texttt{double}s per element (no reuse)
    \end{itemize}
  \item Floating point (counting FMA)
    \begin{itemize}
    \item Evaluate metric terms to quadrature points: $3^4 3^2 3 = 2187$
    \item Inversion/determinants at quadrature points: $3^3 22 = 594$
    \item State/residual evaluation: $2 \cdot 3^4 3^2 = 1458$
    \item Pointwise ``physics'': $\approx 30 \cdot 3^3 = 810$
    \end{itemize}
  \item Arithmetic intensity: $5049 FMA/1184 B = 4.26 FMA/B$
  \end{itemize}
\end{frame}

\begin{frame}{Absorb metric terms into coefficients}
  \begin{equation*}
    -\nabla_\xi\cdot \Big[ (\nabla_x\xi)^T(\eta + \eta' \nabla u \otimes \nabla u)(\nabla_x\xi) \nabla_\xi w \Big] = f
  \end{equation*}
  \begin{itemize}
  \item Asymptotic data motion
    \begin{itemize}
    \item State+Residual: 16 \texttt{double}s per element (54 without reuse)
    \item Coefficients: $6\cdot 27$ \texttt{double}s per element (no reuse)
    \end{itemize}
  \item Floating point (counting FMA)
    \begin{itemize}
    \item State/residual evaluation: $2 \cdot 3^4 3^2 = 1458$
    \item Pointwise ``physics'': $9 \cdot 3^3 = 243$
    \end{itemize}
  \item Arithmetic intensity: $1701 FMA/1424 B = 1.2 FMA/B$
  \end{itemize}
\end{frame}


\begin{frame}{Proposed changes to HPCG}
  \begin{itemize}
  \item F-cycle multigrid
    \begin{itemize}
    \item Multiscale communication, actually solves a problem (optimally)
    \item Mathematically clean, less discretionary decisions needed
    \end{itemize}
  \item Chebyshev(2) smoother
    \begin{itemize}
    \item Avoids fixing a scale for multiplicativeness in Red-Black GS
    \item Enables pipelining/fusion, stresses cache
    \end{itemize}
  \item Unassembled matrix representation
    \begin{itemize}
    \item Increase arithmetic intensity (1/6 to $\approx 10$) and integer issue
    \item FE vs FD/FV?  Scalar or vector problem?
    \item $Q_2$ FE and/or elasticity puts more pressure on cache and registers
    \end{itemize}
  \item Quantify dynamic range
    \begin{itemize}
    \item Ratio between filling memory and $50\%$ efficiency
    \item Helps understand strong scaling/turn-around time
    \end{itemize}
  \end{itemize}
\end{frame}

\end{document}
