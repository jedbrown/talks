% \documentclass[handout]{beamer}
\documentclass{beamer}

\mode<presentation>
{
  \usetheme{ANLBlue}
  % \usefonttheme[onlymath]{serif}
  % \usetheme{Singapore}
  % \usetheme{Warsaw}
  % \usetheme{Malmoe}
  % \useinnertheme{circles}
  % \useoutertheme{infolines}
  % \useinnertheme{rounded}

  \setbeamercovered{transparent=5}
}

\usepackage[english]{babel}
\usepackage[latin1]{inputenc}
\usepackage{alltt,listings,multirow,ulem,siunitx}
\usepackage[absolute,overlay]{textpos}
\TPGrid{1}{1}
\usepackage{pdfpages}
\usepackage{ulem}
\usepackage{multimedia}
\usepackage{multicol}
\newcommand\hmmax{0}
\newcommand\bmmax{0}
\usepackage{bm}
\usepackage{comment}
\usepackage{subcaption}

% font definitions, try \usepackage{ae} instead of the following
% three lines if you don't like this look
\usepackage{mathptmx}
\usepackage[scaled=.90]{helvet}
% \usepackage{courier}
\usepackage[T1]{fontenc}
\usepackage{tikz}
\usetikzlibrary{decorations.pathreplacing}
\usetikzlibrary{shadows,arrows,shapes.misc,shapes.arrows,shapes.multipart,arrows,decorations.pathmorphing,backgrounds,positioning,fit,petri,calc,shadows,chains,matrix,mindmap}

\newcommand\vvec{\bm v}
\newcommand\bvec{\bm b}
\newcommand\bxk{\bvec_0 \times \kappa_0 \cdot \nabla}
\newcommand\delp{\nabla_\perp}

% \usepackage{pgfpages}
% \pgfpagesuselayout{4 on 1}[a4paper,landscape,border shrink=5mm]

\usepackage{JedMacros}

\newcommand{\timeR}{t_{\mathrm{R}}}
\newcommand{\timeW}{t_{\mathrm{W}}}
\newcommand{\mglevel}{\ensuremath{\ell}}
\newcommand{\mglevelcp}{\ensuremath{\mglevel_{\mathrm{cp}}}}
\newcommand{\mglevelcoarse}{\ensuremath{\mglevel_{\mathrm{coarse}}}}
\newcommand{\mglevelfine}{\ensuremath{\mglevel_{\mathrm{fine}}}}

%solution and residual
\newcommand{\vx}{\ensuremath{x}}
\newcommand{\vc}{\ensuremath{\hat{x}}}
\newcommand{\vr}{\ensuremath{r}}
\newcommand{\vb}{\ensuremath{b}}

%operators
\newcommand{\vA}{\ensuremath{A}}
\newcommand{\vP}{\ensuremath{I_H^h}}
\newcommand{\vS}{\ensuremath{S}}
\newcommand{\vR}{\ensuremath{I_h^H}}
\newcommand{\vI}{\ensuremath{\hat I_h^H}}
\newcommand{\vV}{\ensuremath{\mathbf{V}}}
\newcommand{\vF}{\ensuremath{F}}
\newcommand{\vtau}{\ensuremath{\mathbf{\tau}}}


\title{Sean Vitousek: Non-hydrostatic ocean modeling}

\author{Sponsors: {\bf Rob Jacob} \texttt{jacob@mcs.anl.gov} \\
  \qquad Jed Brown \texttt{jedbrown@mcs.anl.gov}}

% - Use the \inst command only if there are several affiliations.
% - Keep it simple, no one is interested in your street address.
% \institute
% {
%   Mathematics and Computer Science Division \\ Argonne National Laboratory
% }

\date{Director's Postdoctoral Fellowship, 2015-04-08}

% This is only inserted into the PDF information catalog. Can be left
% out.
\subject{Talks}


% If you have a file called "university-logo-filename.xxx", where xxx
% is a graphic format that can be processed by latex or pdflatex,
% resp., then you can add a logo as follows:

% \pgfdeclareimage[height=0.5cm]{university-logo}{university-logo-filename}
% \logo{\pgfuseimage{university-logo}}



% Delete this, if you do not want the table of contents to pop up at
% the beginning of each subsection:
% \AtBeginSubsection[]
% {
% \begin{frame}<beamer>
%   \frametitle{Outline}
%   \tableofcontents[currentsection,currentsubsection]
% \end{frame}
% }

\AtBeginSection[]
{
  \begin{frame}<beamer>
    \frametitle{Outline}
    \tableofcontents[currentsection]
  \end{frame}
}

% If you wish to uncover everything in a step-wise fashion, uncomment
% the following command:

% \beamerdefaultoverlayspecification{<+->}

\begin{document}
\lstset{language=C}
\normalem

\begin{frame}
  \titlepage
\end{frame}

\begin{frame}{Sean Vitousek}
  \begin{itemize}
  \item PhD: Stanford (2014, Civil \& Environmental Eng.)
  \item DOE Computational Science Graduate Fellow
  \item Developed non-hydrostatic ocean model during PhD work
  \item Mendenhall Postdoctoral Fellowship (USGS; 2014-present)
  \item Several first author papers, e.g., IJNMF and Ocean Modelling (x2)
  \item Invited talks at major conferences including SIAM CS\&E, ICIAM
  \item Has research faculty offer at UIC
  \item Sean is the complete package: domain science, computation, mathematics
  \end{itemize}
\end{frame}

\begin{frame}{Non-hydrostatic Ocean: strategic to Argonne}
  \begin{columns}
    \begin{column}{0.42\textwidth}
      \begin{itemize}
      \item Fills gap in DOE simulation capability
      \item Important societal questions: climate change, sea level rise from cryosphere
      \item Star HPC application: high resolution, moderate simulation period, \# time steps
      \item PETSc solvers
      \item Hybrid/ALE coordinates: moving mesh technology
      \item Aligns with long-term DOE BER/ASCR goals
      \end{itemize}
    \end{column}
    \begin{column}{0.58\textwidth}
      \begin{center}
        \includegraphics[width=\textwidth]{figures/OceanModelScales.png}
      \end{center}
    \end{column}
  \end{columns}
\end{frame}

\begin{frame}{Opportunities for Sean}
  \begin{itemize}
  \item Build production-grade non-hydrostatic simulation capability
  \item Work with leading solver and HPC researchers
  \item Develop new algorithms and analysis capability
  \item Push scalability/performance limits at LCFs
  \item {\bf Director's Postdoctoral Fellowship}
    \begin{itemize}
    \item Sean has UIC research faculty offer
    \item Director's Postdoc is only opportunity to hire him at ANL
    \item Proposed work is outside scope of existing projects
    \end{itemize}
  \end{itemize}
\end{frame}

\end{document}
