% \documentclass[handout]{beamer}
\documentclass[aspectratio=1610]{beamer}

\mode<presentation>
{
  % \usefonttheme[onlymath]{serif}
  \usetheme{default}
  % \usetheme{Warsaw}
  % \usetheme{Malmoe}
  % \useinnertheme{circles}
  % \useoutertheme{default}
  % \useinnertheme{rounded}

  \setbeamercovered{transparent=1}
}

\usepackage[english]{babel}
\usepackage[latin1]{inputenc}
\usepackage{alltt,listings,multirow,ulem,siunitx}
\usepackage[absolute,overlay]{textpos}
\TPGrid{1}{1}
\usepackage{pdfpages}
\usepackage{animate}
\usepackage{ulem}
\usepackage{multimedia}
\usepackage{multicol}
\newcommand\hmmax{0}
\newcommand\bmmax{0}
\usepackage{bm}
\usepackage{comment}
\usepackage{subcaption}
\usepackage{amscd}

% font definitions, try \usepackage{ae} instead of the following
% three lines if you don't like this look
\usepackage{mathptmx}
\usepackage[scaled=.90]{helvet}
% \usepackage{courier}
\usepackage[T1]{fontenc}
\usepackage{tikz}
\usetikzlibrary{decorations.pathreplacing}
\usetikzlibrary{shadows,arrows,shapes.misc,shapes.arrows,shapes.multipart,arrows,decorations.pathmorphing,backgrounds,positioning,fit,petri,calc,shadows,chains,matrix,mindmap}

\newcommand\vvec{\bm v}
\newcommand\bvec{\bm b}
\newcommand\bxk{\bvec_0 \times \kappa_0 \cdot \nabla}
\newcommand\delp{\nabla_\perp}

% \usepackage{pgfpages}
% \pgfpagesuselayout{4 on 1}[a4paper,landscape,border shrink=5mm]

\usepackage{JedMacros}
\usepackage{standalone}

\newcommand{\timeR}{t_{\mathrm{R}}}
\newcommand{\timeW}{t_{\mathrm{W}}}
\newcommand{\mglevel}{\ensuremath{\ell}}
\newcommand{\mglevelcp}{\ensuremath{\mglevel_{\mathrm{cp}}}}
\newcommand{\mglevelcoarse}{\ensuremath{\mglevel_{\mathrm{coarse}}}}
\newcommand{\mglevelfine}{\ensuremath{\mglevel_{\mathrm{fine}}}}

%solution and residual
\newcommand{\vx}{\ensuremath{x}}
\newcommand{\vc}{\ensuremath{\hat{x}}}
\newcommand{\vr}{\ensuremath{r}}
\newcommand{\vb}{\ensuremath{b}}

%operators
\newcommand{\vA}{\ensuremath{A}}
\newcommand{\vP}{\ensuremath{I_H^h}}
\newcommand{\vS}{\ensuremath{S}}
\newcommand{\vR}{\ensuremath{I_h^H}}
\newcommand{\vI}{\ensuremath{\hat I_h^H}}
\newcommand{\vV}{\ensuremath{\mathbf{V}}}
\newcommand{\vF}{\ensuremath{F}}
\newcommand{\vtau}{\ensuremath{\mathbf{\tau}}}


\title{From Wolves to Computational Science \& Engineering}
\author{{\bf Jed Brown}}

% - Use the \inst command only if there are several affiliations.
% - Keep it simple, no one is interested in your street address.
% \institute
% {
%   Mathematics and Computer Science Division \\ Argonne National Laboratory
% }

\date{Early Career Panel, SIAM CSE, 2019-02-25}
%\\[1em]
%This talk: \url{https://jedbrown.org/files/20181002-PhyPID.pdf}}

% This is only inserted into the PDF information catalog. Can be left
% out.
\subject{Talks}


% If you have a file called "university-logo-filename.xxx", where xxx
% is a graphic format that can be processed by latex or pdflatex,
% resp., then you can add a logo as follows:

% \pgfdeclareimage[height=0.5cm]{university-logo}{university-logo-filename}
% \logo{\pgfuseimage{university-logo}}



% Delete this, if you do not want the table of contents to pop up at
% the beginning of each subsection:
% \AtBeginSubsection[]
% {
% \begin{frame}<beamer>
%   \frametitle{Outline}
%   \tableofcontents[currentsection,currentsubsection]
% \end{frame}
% }

% \AtBeginSection[]
% {
%   \begin{frame}<beamer>
%     \frametitle{Outline}
%     \tableofcontents[currentsection]
%   \end{frame}
% }

% If you wish to uncover everything in a step-wise fashion, uncomment
% the following command:

% \beamerdefaultoverlayspecification{<+->}

\begin{document}
\lstset{language=C}
\normalem

\begin{frame}
  \titlepage
\end{frame}

\begin{frame}{University of Alaska Fairbanks}
  \begin{columns}
    \begin{column}{0.6\textwidth}
      \includegraphics[width=\textwidth]{$HOME/pics/antarctica/dscf1399.jpg}
    \end{column}
    \begin{column}{0.4\textwidth}
      \begin{itemize}
      \item 2004 BS Math, BS Physics
      \item NCAA Cross-country ski racing
      \item 2003 Began undergraduate research in ice sheet modeling
        \begin{itemize}
        \item Inherited Fortran code; heavily refactored, added
          OpenMP, many features
        \item Started over in C++ with PETSc: Parallel Ice Sheet Model
        \end{itemize}
      \item 2006 MS Mathematics (PISM)
      \item Worked part-time developing PISM as Research Technician
      \item Accepted PhD position at ETH Z\"urich on high fidelity flow modeling
      \end{itemize}
    \end{column}
  \end{columns}
\end{frame}

\begin{frame}{ETH Z\"urich to Argonne}
  \begin{columns}
    \begin{column}{0.5\textwidth}
      \includegraphics[width=\textwidth]{$HOME/pics/xuelian/037-img_0112.jpg}
    \end{column}
    \begin{column}{0.5\textwidth}
      \begin{itemize}
      \item Advisor promptly lost interest in high fidelity 3D
        modeling
      \item I hung out in Applied Math, Geodynamics, and petsc-dev
      \item Developed solvers for saddle point problems, lots of other
        PETSc features
      \item Co-wrote DOE ASCR proposal for ice sheet modeling
        $\to$ Postdoc at Argonne
      \item PETSc core development: time integration, composable
        linear and nonlinear solvers
      \item Lots of travel representing PETSc
      \item Permanent after two years; working remotely from Boulder,
        Adjoint Professor
      \item SIAM/ACM CS\&E, SIAG/SC Jr. Scientist
      \end{itemize}
    \end{column}
  \end{columns}
\end{frame}

\begin{frame}{CU Boulder}
  \begin{columns}
    \begin{column}{0.4\textwidth}
      \includegraphics[width=\textwidth]{$HOME/dl/IMG_20190126_114410.jpg}
    \end{column}
    \begin{column}{0.6\textwidth}
      \begin{itemize}
      \item 2015 Asst. Professor of CS at CU Boulder
      \item Research autonomy
      \item Feasible to bootstrap deep/risky work before funding
      \item Deeper and more personal work on education and career development
      \item Major course revisions; more hands-on activities
      \item Faculty hiring, committee work
      \end{itemize}
    \end{column}
  \end{columns}
\end{frame}

\begin{frame}{Group Meeting}
  \includegraphics[width=.8\textwidth]{figures/phypid/2018-group-meeting.jpg}
\end{frame}

\end{document}
