%\documentclass[handout]{beamer}
\documentclass{beamer}

\mode<presentation>
{
\usetheme{default}
%\usetheme{Singapore}
%\usetheme{Warsaw}
%\usetheme{Malmoe}
% \useinnertheme{circles}
% \useoutertheme{infolines}
% \useinnertheme{rounded}

\setbeamercovered{transparent}
}

\usepackage[english]{babel}
\usepackage[latin1]{inputenc}
\usepackage{bm,textpos,alltt,listings,multirow,ulem}
\usepackage[amssymb]{SIunits}

% font definitions, try \usepackage{ae} instead of the following
% three lines if you don't like this look
\usepackage{mathptmx}
\usepackage[scaled=.90]{helvet}
\usepackage{courier}
\usepackage[T1]{fontenc}
\usepackage{tikz}
\usetikzlibrary[shapes.arrows,arrows,shapes.misc]

% \usepackage{pgfpages}
% \pgfpagesuselayout{4 on 1}[a4paper,landscape,border shrink=5mm]

\usepackage{xspace}
\makeatletter
\DeclareRobustCommand\onedot{\futurelet\@let@token\@onedot}
\def\@onedot{\ifx\@let@token.\else.\null\fi\xspace}
\def\eg{{e.g}\onedot} \def\Eg{{E.g}\onedot}
\def\ie{{i.e}\onedot} \def\Ie{{I.e}\onedot}
\def\cf{{c.f}\onedot} \def\Cf{{C.f}\onedot}
\def\etc{{etc}\onedot}
\def\vs{{vs}\onedot}
\def\wrt{w.r.t\onedot}
\def\dof{d.o.f\onedot}
\def\etal{{et al}\onedot}
\makeatother

\newcommand{\II}{\mathcal{I}}
\newcommand{\C}{\mathbb{C}}
\newcommand{\D}{\mathcal{D}}
\newcommand{\E}{\mathcal{E}}
\newcommand{\F}{\mathcal{F}}
\newcommand{\I}{\mathcal{I}}
\newcommand{\N}{\mathcal{N}}
\newcommand{\PP}{\mathcal{P}}
\newcommand{\bigO}{\mathcal{O}}
\newcommand{\R}{\mathbb{R}}
\newcommand{\Rz}{\mathcal{R}}
\newcommand{\kb}{\tt}
\newcommand{\blue}{\textcolor{blue}}
\newcommand{\green}{\textcolor{green!70!black}}
\newcommand{\red}{\textcolor{red}}
\newcommand{\brown}{\textcolor{brown}}
\newcommand{\cyan}{\textcolor{cyan}}
\newcommand{\magenta}{\textcolor{magenta}}
\newcommand{\yellow}{\textcolor{yellow}}
\newcommand{\mini}{\mathop{\rm minimize}}
\newcommand{\st}{\mbox{subject to }}
\newcommand{\lap}{\Delta}
\newcommand{\grad}{\nabla}
%\renewcommand{\div}{\nabla \cdot}
\DeclareMathOperator{\divrg}{div}
\def\code#1{{\tt #1}}
\def\shell#1{{\tt \$ #1}}
\newcommand\mtab{\hspace{\stretch{1}}}
\newcommand\ud{\,\mathrm{d}}
\newcommand\bslash{{$\backslash$}}
\newcommand\half{{\frac 1 2}}
\newcommand{\abs}[1]{\left\lvert #1 \right\rvert}
\newcommand{\bigabs}[1]{\big\lvert #1 \big\rvert}
\newcommand{\norm}[1]{\left\lVert #1 \right\rVert}
\newcommand\oneitem[1]{\begin{itemize} \item #1 \end{itemize}}
\newcommand\pp{{\mathfrak p}}
\newcommand\ff{\bm f}
\newcommand\uu{\bm u}
\newcommand\vv{\bm v}
\newcommand\ww{\bm w}
\newcommand\DD{D}
\newcommand{\tcolon}{\!:\!}
\DeclareMathOperator{\sgn}{sgn}
\DeclareMathOperator{\card}{card}
\DeclareMathOperator{\trace}{tr}
\DeclareMathOperator{\sspan}{span}
\renewcommand{\bar}{\overline}
\DeclareMathOperator{\divergence}{div}
\renewcommand\div\divergence


\title{Computing free surface flows}

\author{Jed Brown}


% - Use the \inst command only if there are several affiliations.
% - Keep it simple, no one is interested in your street address.
\institute[ETH Z\"urich]
{
  Laboratory of Hydrology, Hydraulics, and Glaciology \\
  ETH Z\"urich
}

\date[2010-11-18]{Fachgespr\"ach 2010-11-18}

% This is only inserted into the PDF information catalog. Can be left
% out.
\subject{Talks}


% If you have a file called "university-logo-filename.xxx", where xxx
% is a graphic format that can be processed by latex or pdflatex,
% resp., then you can add a logo as follows:

% \pgfdeclareimage[height=0.5cm]{university-logo}{university-logo-filename}
% \logo{\pgfuseimage{university-logo}}



% Delete this, if you do not want the table of contents to pop up at
% the beginning of each subsection:
% \AtBeginSubsection[]
% {
% \begin{frame}<beamer>
% \frametitle{Outline}
% \tableofcontents[currentsection,currentsubsection]
% \end{frame}
% }

% If you wish to uncover everything in a step-wise fashion, uncomment
% the following command:

%\beamerdefaultoverlayspecification{<+->}

\begin{document}
\lstset{language=C}
\normalem

\begin{frame}
\titlepage
\end{frame}

\begin{frame}
  \frametitle{Outline}
  \tableofcontents
  % You might wish to add the option [pausesections]
\end{frame}

\section{Motivation}
\begin{frame}
  \includegraphics[width=\textwidth]{figures/GroundingLine/NaturalHistory2008} \\
  \vspace{-.5em}
  {\tiny Bindschadler 2008}
\end{frame}

\begin{frame}
  \includegraphics[width=0.5\textwidth]{figures/GroundingLine/SchoofNature2010}
\end{frame}

\begin{frame}{Grounding lines}
  \begin{columns}
    \begin{column}{0.45\textwidth}
      \centering
      \includegraphics[width=\textwidth]{figures/GroundingLine/circulation} \\
      \vspace{-.5em}
      {\tiny Schoof 2007} \\
      \includegraphics[width=\textwidth]{figures/GroundingLine/isothermal-Linfty} \\
      \vspace{-.5em}
      {\tiny Bueler et. al. 2005}
    \end{column}
    \begin{column}{0.55\textwidth}
      \begin{itemize}
      \item ocean circulation is very sensitive to grounding line
        geometry, feedback
      \item non-shallow physics applies in vicinity of grounding line
      \item current models are less than first-order accurate at margins
      \item extremely high resolution needed for qualitatively correct
        results on Eulerian meshes
      \end{itemize}
    \end{column}
  \end{columns}
\end{frame}

\begin{frame}{}
  \begin{columns}
    \begin{column}{0.63\textwidth}
      \includegraphics[width=\textwidth]{figures/GroundingLine/MeshDependence}
    \end{column}
    \begin{column}{0.35\textwidth}
      {\centering\includegraphics[width=1.2\textwidth]{figures/GroundingLine/SchoofGeom} \\
      \footnotesize{(Schoof 2007)}} \\
      \bigskip
      \bigskip
      Evolution of grounding line location on 20, 15, 10, 7.5 and 2.5 kilometer
        meshes in one horizontal dimension.  \footnotesize{(\emph{Durand et al. 2009})}
    \end{column}
  \end{columns}
\end{frame}

\begin{frame}{} %{Thwaites and Pine Island glaciers}
  \begin{columns}
    \begin{column}{0.1\textwidth} 
    \end{column}
    \begin{column}{.8\textwidth}
      \centering
      \includegraphics[width=0.95\textwidth]{figures/GroundingLine/thwaites}
    \end{column}
    \begin{column}{0.1\textwidth}
      {\tiny Holt \etal 2006}
    \end{column}
  \end{columns}
\end{frame}

\begin{frame}
  \begin{columns}
    \begin{column}{0.45\textwidth}
      \vspace{-1em}
      \includegraphics[width=\textwidth]{figures/GroundingLine/SchoofNature2010} \\
\footnotesize{(Schoof 2010)}
    \end{column}
    \begin{column}{0.55\textwidth}
      {\large \color{blue}{$y^+$ underneath an ice shelf}}
    \begin{itemize}
    \item Order of magnitude dimensions: length \unit{100}{\meter},       speed \unit{10}{\centi\meter\per\second}
    \item Viscous boundary layer: $y^+ \in \bigO(1) \implies       \unit{1}{\milli\meter}$ grid
    \item No-slip boundary conditions requires \emph{resolution} of this       layer % (wall resolution)
    \item Otherwise we need nonlinear slip
      \begin{itemize} \item still usually $y^+ \in \bigO(100)$       \end{itemize}
    \item Estimates come from validation (lab experiments) with heat       transfer in industrial and aerospace applications
    \item Thermohaline boundary layer: \unit{\text{1--10}}{\meter}
    \item Boundary layer equations require solution of a Riemann problem
    %\item Is simulation of this process plausible without boundary-fitted meshes?
    \end{itemize}
   \end{column}
  \end{columns}
\end{frame}

% \begin{frame}{$y^+$ underneath an ice shelf}
%   \begin{itemize}
%   \item Order of magnitude dimensions:
%     length \unit{100}{\meter}, speed \unit{10}{\centi\meter\per\second}
%   \item Viscous boundary layer: $y^+ \in \bigO(1) \implies % \unit{1}{\milli\meter}$ grid spacing
%   \item No-slip boundary conditions requires \emph{resolution} of this % layer % (wall resolution)
%   \item Otherwise we need nonlinear slip conditions
%     \begin{itemize} \item still usually $y^+ \in \bigO(100)$ % \end{itemize}
%   \item Estimates come from validation (lab experiments) with heat
%     transfer in industrial and aerospace applications
%   \item Thermohaline boundary layer is \unit{\text{1--10}}{\meter}
%   \item Boundary layer equations require solution of a Riemann problem
%   \item Is simulation of this process plausible without \\
%     boundary-fitted meshes?
%   \end{itemize}
% \end{frame}

\begin{frame}{LES+RANS with wall modeling}
  \begin{itemize}
  \item State of the art for high-Reynolds separating flows
  \item Subshelf circulation separates when it reaches neutral buoyancy \\ (this is a crucial limiting process)
  \item Is it possible to accurately predict heat transfer, separation, and overturning with $y^+ \in \bigO(10^5)$?
  \end{itemize}
  \begin{quotation}
    %Also, it
    It has been repeatedly observed, especially at high Reynolds     numbers and coarse grids and with the interface location being around     $y^+ = \bigO(100-200)$, that the high turbulent viscosity generated by     the turbulunce model in the inner region extends, as subgrid-scale     viscosity, deeply into the outer LES region, causing severe damping in     the resolved motion and a misrepresentation of the resolved structure as     well as the time-mean properties.
  \end{quotation}
  (Tessicini, Li, Leschziner, \emph{Simulation of Separation from Curved Surfaces with Combined LES and RANS Schemes}, 2007)
\end{frame}

\section{ALE Formulation}
\begin{frame}{Non-Newtonian Stokes system: velocity $\bm u$, pressure $p$}
\begin{columns}
\begin{column}{0.5\textwidth}
  \alert{\begin{align*}
    -\nabla \cdot(\eta D\uu) + \nabla p - \ff &= 0 \\
    \nabla \cdot \uu &= 0
  \end{align*}}
\end{column}
\begin{column}{0.5\textwidth}
    \begin{align*}
      D\uu &= \tfrac 1 2 \left(\nabla \uu + (\nabla \uu)^T \right) \\
      \gamma(D\uu) &= \tfrac 1 2 D\uu \tcolon D\uu \\
      \eta(\gamma) &= B(\Theta,\dotsc)\big(\epsilon + \gamma \big)^{\frac{\mathfrak{p}-2}{2}} \\
      \mathfrak{p} &= 1 + \tfrac{1}{\mathfrak{n}} \approx \tfrac 4 3 \\
      T &= \bm 1 - \bm n \otimes \bm n \\
    \end{align*}
\end{column}
\end{columns}
\vspace{-1.5em}
    with boundary conditions
    \begin{align*}
      (\eta D\bm u - p\bm 1)\cdot\bm n =
      \begin{cases}\bm 0 & \text{free surface} \\
        -\rho_w z \bm n & \text{ice-ocean interface}\end{cases} \\
      \bm u = \bm 0\qquad\qquad \text{frozen bed}, \Theta < \Theta_0 \\
      \left. \begin{aligned}
          \bm u \cdot \bm n &= \bm g_{\text{melt}}(T\uu,\dotsc) \\
          T (\eta D\bm u - p\bm 1)\cdot\bm n &= \bm g_{\text{slip}}(T \bm u,\dotsc) \end{aligned}\right\}
      \text{nonlinear slip}, \Theta \ge \Theta_0 \\
    \end{align*}
    \vspace{-3em}
    \[ \bm g_{\text{slip}}(T\uu) = \beta_{\mathfrak{m}}(\dotsc) \lvert T\bm u \rvert^{\mathfrak{m}-1} T \bm u \]
    Navier $\mathfrak{m}=1$, \quad Weertman $\mathfrak{m}\approx \frac 1 3$, \quad Coulomb $\mathfrak{m}=0$.
\end{frame}

\begin{frame}{Other critical equations}
  \vspace{-0.3em}
  \begin{itemize}
  \item Mesh motion: $\bm x$
    \begin{columns}
      \begin{column}{0.4\textwidth}
        \vspace{-1.5em}
        \begin{gather*}
          \alert{-\nabla\cdot\bm \sigma = 0} \\
          \text{surface: }(\bm {\dot{x}}- \bm u)\cdot\bm n = q_{BL},\;
          T\bm \sigma \cdot \bm n = 0
        \end{gather*}
      \end{column}
      \begin{column}{0.6\textwidth}
        \begin{gather*}
          \bm \sigma = \mu \Big[ 2 D\bm w + (\nabla \bm w)^T\nabla \bm w \Big] + \lambda|\nabla\bm w|\bm 1 \\
          \bm w = \bm x - \bm x_0 \\
        \end{gather*}
      \end{column}
    \end{columns}
\vspace{-1em}
  \item Heat transport: $\Theta$ (enthalpy)
    \begin{multline*}
      \frac{\partial}{\partial t} \Theta + {\color{blue} (\bm u - \bm{\dot{x}})}\cdot \nabla \Theta \\
      - \nabla\cdot \Big[ {\color{green!70!black} \kappa_T(\Theta)\nabla T(\Theta)} + {\color{magenta!70!black} \kappa_\omega\nabla \omega(\Theta) + \bm q_D(\Theta)} \Big] - {\color{cyan!70!black} \eta D\bm u\tcolon D\bm u} = 0
    \end{multline*}
    \vspace{-0.5em}
    \begin{itemize}
      \begin{columns}
        \begin{column}{0.2\textwidth}\end{column}
        \begin{column}{0.4\textwidth}
    \item {\color{blue} ALE advection}
    \item {\color{green!70!black} Thermal diffusion}
    \end{column}
    \begin{column}{0.4\textwidth}
    \item {\color{magenta!70!black} Moisture diffusion/Darcy flow}
    \item {\color{cyan!70!black} Strain heating}
    \end{column}
  \end{columns}
\end{itemize}
\vspace{0.3em}
    Note: $\kappa(\Theta)$ and $\bm q_D(\Theta)$ are very sensitive near $\Theta=\Theta_0$
\end{itemize}
\vspace{-.6em}
\begin{block}{Summary of primal variables in DAE}
  \begin{tabular}{lll}
    $u$ & velocity & algebraic \\
    $p$ & pressure & algebraic \\
    $x$ & mesh location & algebraic in domain, differential at surface \\
    $\Theta$ & enthalpy & differential
  \end{tabular}
\end{block}
\end{frame}

%\input{slides/Dohp/Resolution.tex}
\newcommand{\colorA}[1]{{\color{red} #1}}
\newcommand{\colorB}[1]{{\color{green!60!black} #1}}
\newcommand{\colorC}[1]{{\color{blue} #1}}
\newcommand{\colorD}[1]{{\color{magenta!70!black} #1}}
\newcommand{\colorE}[1]{{\color{cyan!70!black} #1}}
\newcommand{\colorF}[1]{{\color{yellow!60!black} #1}}
\newcommand{\colorG}[1]{{\color{red!50!white} #1}}

\begin{frame}{ALE form}
  After discretization in time ($\alpha \propto 1/\Delta t$) we have a Jacobian
  \begin{equation*}
    \begin{bmatrix}
      \colorA{A_{II}} & \colorA{A_{I\Gamma}}             &                       &                             &                     &   \\
      & \colorB{\alpha M_{\Gamma\Gamma}} &                       & \colorB{- N_{\Gamma\Gamma}} &                       &  \\
      \colorG{G_{II}}      & \colorG{G_{\Gamma I}} & \colorC{B_{II}}       & \colorC{B_{I\Gamma}}        & \colorC{C_{I}^T}    & \colorD{D_I} \\
      \colorG{G_{I\Gamma}} &        \colorG{G_{\Gamma\Gamma}}                          & \colorC{B_{\Gamma I}} & \colorC{B_{\Gamma\Gamma}}   & \colorC{C_{\Gamma}^T} & \colorD{D_\Gamma} \\
      \colorG{G_{Ip}}        &  \colorG{G_{\Gamma p}}                                & \colorC{C_{I}}        & \colorC{C_{\Gamma}}         &                   & \\
      \colorE{\alpha E_I}    & \colorE{\alpha E_\Gamma} & \colorE{F_I} & \colorE{F_\Gamma} & & \colorF{\alpha M_\Theta + J}
    \end{bmatrix}
    \begin{bmatrix}
      x_I \\ x_\Gamma \\ u_I \\ u_\Gamma \\ p \\ \Theta
    \end{bmatrix}
  \end{equation*}
  \begin{itemize}
  \item \colorA{mesh motion equations (Laplace-Beltrami or pseudo-elasticity)}
  \item \colorB{$(\dot{\bm x} - \bm u)\cdot \bm n = \text{accumulution}$}
  \item \colorG{``just'' geometry}
  \item \colorC{Stokes problem}
  \item \colorD{temperature dependence of rheology}
  \item \colorE{convective terms and strain heating in heat transport}
  \item \colorF{thermal advection-diffusion}
  \end{itemize}
\end{frame}

\begin{frame}{Power-law Stokes Scaling}
  \centering
  \includegraphics[width=\textwidth]{figures/Dohp/Stokes2} \\
  Only assemble $Q_1$ matrices, ML+PETSc smoothers for elliptic pieces \\
  (easy geometry and coefficients)
\end{frame}


\section{Conservation}
\begin{frame}{Artifacts of stabilization}
  \begin{columns}
    \begin{column}{0.4\textwidth}
      \includegraphics[width=\textwidth]{figures/Stabilization/RayleighTaylor} \\
      Rayleigh-Taylor initiation, isoviscous \\
      (Dave May and Yury Mishin)
    \end{column}
    \begin{column}{0.6\textwidth}
      $Q_2-P_{-1}$ (stable, locally conservative) \\
        \includegraphics[width=\textwidth]{figures/Stabilization/Q2Pm1} \\
        \medskip
        $Q_1-Q_1$ (stabilized) \\
        \includegraphics[width=\textwidth]{figures/Stabilization/Q1Q1stab} \\
        \vbox{\hspace{4em} $u$ \hspace{8em} $v$}
    \end{column}
  \end{columns}
\end{frame}


\section[Slip]{Slip boundary conditions on bumpy surfaces}
\begin{frame}{Construction of conservative nodal normals}
  \begin{gather*}
    \bm n^i = \int_\Gamma \phi^i \bm n
  \end{gather*}
  \begin{itemize}
  \item Exact conservation even with rough surfaces
  \item Definition is robust in 2D and for first-order elements in 3D
  \item $\int_\Gamma \phi^i = 0$ for corner basis function of undeformed $P_2$ triangle
  \item May be negative for sufficiently deformed quadrilaterals
  \item Mesh motion should use normals from CAD model
    \begin{itemize}
    \item Difference between CAD normal and conservative normal introduces correction term to conserve mass within the mesh
\item Anomolous velocities if disagreement is large \\ (fast moving mesh, rough surface)
    \end{itemize}
  \item Normal field not as smooth/accurate as desirable \\ (and achievable with non-conservative normals)
    \begin{itemize}
    \item Mostly problematic for surface tension
    \item Walkley et al, \emph{On calculation of normals in free-surface flow problems}, 2004
    \end{itemize}
  \end{itemize}
\end{frame}

\begin{frame}{Need for well-balancing}
  \includegraphics[width=\textwidth]{figures/slip/Behr2004-NormalVsResidual} \\
  \includegraphics[width=\textwidth]{figures/slip/Behr2004-NavierSloshing} \\
  \footnotesize{(Behr, \emph{On the application of slip boundary condition on curved surfaces}, 2004)}
\end{frame}

\begin{frame}{``No'' boundary condition}
  \begin{itemize}
  \item Integration by parts produces
    \begin{gather*}
      \int_\Gamma \bm v \cdot T \bm\sigma \cdot \bm n, \qquad \bm\sigma = \eta D \bm u - p\bm 1, \qquad T = \bm 1 - \bm n \otimes \bm n
    \end{gather*}
  \item Continuous weak form requires either
    \begin{itemize}
    \item Dirichlet: $\bm u |_{\Gamma} = \bm f \implies \bm v|_\Gamma = 0$
    \item Neumann/Robin: $\bm\sigma\cdot\bm n |_\Gamma = \bm g(\bm u,p)$
    \end{itemize}
  \item Discrete problem allows integration of $\bm\sigma\cdot\bm n$ ``as is''
    \begin{itemize}
    \item Extends validity of equations to include $\Gamma$
    \item \alert{Not valid} for continuum equations
    \item Introduced by Papanastasiou, Malamataris, and Ellwood, 1992 for Navier-Stokes outflow boundaries
    \item Griffiths, {\small \emph{The `no boundary condition' outflow boundary condition}, 1997}
      \begin{itemize}
      \item Proves $L^\infty$ order of accuracy $\bigO((h + 1/\Peclet)^{p+1})$ \\
        for Galerkin finite elements of order $p$ (linear advection-diffusion)
      \item Demonstrates equivalence with collocation at Radau points \\ in outflow element
      \end{itemize}
    \item Used in slip boundary conditions by Behr 2004
    \end{itemize}
  \end{itemize}
\end{frame}


\section[Hydrostatic]{A robust multigrid solver for the hydrostatic equations}
\begin{frame}{Hydrostatic equations for ice sheet flow}
  \begin{itemize}
  \item Valid when $w_x \ll u_z$, independent of basal friction {\small (Schoof\&Hindmarsh 2010)}
  \item Eliminate $p$ and $w$ from Stokes by incompressibility:\\
    \quad 3D elliptic system for $\bm u = (u,v)$
    \begin{align*}
      - \nabla\cdot \left[ \eta
        \begin{pmatrix}
          4 u_x + 2 v_y & u_y + v_x & u_z \\
          u_y + v_x & 2 u_x + 4 v_y & v_z
        \end{pmatrix} \right] + \rho g \bar\nabla h & = 0
    \end{align*}
    \begin{align*}
      \eta(\theta,\gamma) &= \frac{B(\theta)}{2} (\gamma_0 + \gamma)^{\frac{1-\mathfrak n}{2\mathfrak n}}, \qquad \mathfrak n \approx 3 \\
      \gamma &= u_x^2 + v_y^2 + u_xv_y + \frac 1 4 (u_y+v_x)^2 + \frac 1 4 u_z^2 + \frac 1 4 v_z^2
    \end{align*}
    and slip boundary $\sigma \cdot \bm n = \beta^2 \bm u$ where
    \begin{align*}
      \beta^2(\gamma_b) &= \beta_0^2 (\epsilon_b^2 + \gamma_b)^{\frac{\mathfrak m-1}{2}}, \qquad 0 < \mathfrak m \le 1 \\
      \gamma_b &= \frac 1 2 (u^2 + v^2)
    \end{align*}
  \item $Q_1$ FEM with Newton-Krylov-Multigrid solver in PETSc: \code{src/snes/examples/tutorials/ex48.c}
  \end{itemize}
\end{frame}

\frame{
  \vspace{-8em}
  \includegraphics[width=1.2\textwidth]{figures/THI/x-5km-m8p5l5-clip}
}

\frame{
  \vspace{-1em}
  \includegraphics[width=\textwidth]{figures/THI/y-5km-m6p5l4-clip}
  \vspace{-3.5em}
  \begin{itemize}
  \item Bathymetry is essentially discontinuous on any grid
  \item Shallow ice approximation produces oscillatory solutions
  \item Nonlinear and linear solvers have major problems or fail
  \item Grid sequenced Newton-Krylov multigrid works \\
    as well as in the smooth case
  \end{itemize}
}

\input{slides/THI/WhatAboutSplitting.tex}

\begin{frame}{Outlook}
  \begin{block}{}
    \begin{itemize}
    \item Exact local conservation is critical for problems with discontinuous geometry and coefficients
    \item Nonlinear slip on irregular surfaces is hard but tractable (mostly)
    \item Smooth manufactured solutions are necessary, but not sufficient to study solver and discretization performance
    \item Need good software to combine relaxation for loosely coupled processes and factorization
      for stiff/indefinite coupling
    \item Modeling of boundary layer processes in highly anisotropic geometry likely requires conforming to the interface
    \end{itemize}    
  \end{block}
  \begin{block}{Tools}
    \begin{itemize}
    \item PETSc\ \url{http://mcs.anl.gov/petsc}
      \begin{itemize}\item ML, Hypre, MUMPS
      \end{itemize}
    \item ITAPS \url{http://itaps.org}
      \begin{itemize}\item MOAB, CGM, Lasso
      \end{itemize}
    \end{itemize}
  \end{block}
\end{frame}
\end{document}
