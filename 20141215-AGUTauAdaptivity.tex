\documentclass[final,t]{beamer}
\mode<presentation>
{
  \usetheme{MCSposter}
  \usecolortheme{default}
  \usefonttheme[onlymath]{serif}
}
\usepackage[orientation=landscape,size=custom,width=120,height=90,scale=1.3]{beamerposter}
%\usepackage[orientation=landscape,size=a0,scale=1.4]{beamerposter}
\usepackage{pgfpages}
%\pgfpagesuselayout{resize to}[a4paper,landscape,border shrink=5mm]
%\pgfpagesuselayout{resize to}[a0paper,landscape]

% additional settings
\setbeamerfont{itemize}{size=\normalsize}
\setbeamerfont{itemize/enumerate body}{size=\normalsize}
\setbeamerfont{itemize/enumerate subbody}{size=\normalsize}

% additional packages
\usepackage{times}
%\usepackage{subcaption}
\usepackage{subfig}
\usepackage{textpos}
\usepackage{wrapfig}
\usepackage{sidecap}
%\usepackage{sfmath} 
\usepackage{amsmath,amsthm,bm}
\usepackage{microtype}
\usepackage{siunitx}
\DeclareSIUnit\year{a}
\sisetup{retain-unity-mantissa = false}
\usepackage{exscale}
\usepackage{multicol,multirow}
\usepackage{booktabs}
%\usepackage[english]{babel}
\usepackage[latin1]{inputenc}
\newcommand\todo[1]{{\color{red}\bf [TODO: #1]}}
\usepackage{tikz}
\usetikzlibrary{decorations.pathreplacing}
\usetikzlibrary{shadows,arrows,shapes.misc,shapes.arrows,shapes.multipart,arrows,decorations.pathmorphing,backgrounds,positioning,fit,petri,calc,shadows,chains,matrix}
\usepackage{fancyvrb}
\newcommand\cverb[1][]{\SaveVerb[%
   aftersave={\textnormal{\UseVerb[#1]{vsave}}}]{vsave}}

\bibliographystyle{unsrt-shortauthor}

%\def\newblock{\hskip .11em plus .33em minus .07em} % for natbib and beamer IMPORTANT
\listfiles
\graphicspath{{/home/jed/talks/figures/}{/home/jed/src/aterrel-presentations/figures/}}
% Display a grid to help align images
%\beamertemplategridbackground[1cm]


\title{\huge $\tau$-adaptivity for nonsmooth processes in heterogeneous media {\normalsize \texttt{DI11A-4256}}}
\author[Jed Brown]{Jed Brown {\texttt{jed@jedbrown.org}} ANL and CU Boulder}
\institute[MCS]{Download this poster from \url{https://jedbrown.org/files/20141215-AGUTauAdaptivity.pdf}}

\newcommand{\newt}[1]{\tilde{#1}}
\newcommand{\btab}{\hspace{\stretch{1}}}
\newcommand{\C}{\mathbb{C}}
\newcommand{\N}{\mathbb{N}}
\newcommand{\Q}{\mathbb{Q}}
\newcommand{\R}{\mathbb{R}}
\newcommand{\Rz}{\mathcal{R}}
\newcommand{\RR}{{\bar{\mathbb{R}}}}
\newcommand{\II}{\mathcal{I}}
\newcommand{\Z}{\mathbb{Z}}
\newcommand{\Zp}{\mathbb{Z}_+}
\newcommand{\B}{\mathcal{B}}
\newcommand{\M}{\mathcal{M}}
\newcommand{\LL}{\mathcal{L}}
\newcommand{\PP}{\mathscr{P}}
\newcommand{\ff}{\bm f}
\newcommand{\uu}{\bm u}
\newcommand{\vv}{\bm v}
\newcommand{\ww}{\bm w}
\newcommand{\DD}{D}
\newcommand{\EE}{\mathcal E}
\newcommand{\VV}{\bm{\mathcal{V}}}
\newcommand{\Pspace}{\mathcal{P}}
\newcommand\pfrak{{\mathfrak p}}
\newcommand{\di}{\partial}
\newcommand{\bigO}{\mathcal{O}}
\newcommand{\abs}[1]{\left\lvert #1 \right\rvert}
\newcommand{\bigabs}[1]{\big\lvert #1 \big\rvert}
\newcommand{\norm}[1]{\left\lVert #1 \right\rVert}
\newcommand{\ceil}[1]{\left\lceil #1 \right\rceil}
\newcommand{\floor}[1]{\left\lfloor #1 \right\rfloor}
\newcommand{\dif}{\bigtriangleup}
\newcommand{\ud}{\,\mathrm{d}}
\newcommand{\tcolon}{\!:\!}
\DeclareMathOperator{\sgn}{sgn}
\DeclareMathOperator{\card}{card}
\DeclareMathOperator{\trace}{tr}
\DeclareMathOperator{\sspan}{span}
\renewcommand{\bar}{\overline}
\newcommand{\ed}{\dot{\epsilon}}
\newcommand\citet[1]{\cite{#1}}
\newcommand\citep[1]{\cite{#1}}

% abbreviations
%%%%%%%%%%%%%%%%%%%%%%%%%%%%%%%%%%%%%%%%%%%%%%%%%%%%%%%%%%%%%%%%%%%%%%%%%%%%%%%%%%%%%%%%%%%%%%%%%%%%%%%%%%%%

% Need to declare these here because newcommand inside a frame is fragile
\newcommand\mgdx{1.9em}
\newcommand\mgdy{2.5em}
\newcommand\mgloc[4]{(#1 + #4*\mgdx*#3,#2 + \mgdy*#3)}
\newcommand{\mglevel}{\ensuremath{\ell}}
\newcommand{\mglevelcp}{\ensuremath{\mglevel_{\mathrm{cp}}}}
\newcommand{\mglevelfine}{\ensuremath{\mglevel_{\mathrm{fine}}}}

\begin{document}
\begin{frame}{} 
  \vspace{-3em}
  \begin{columns}
    \begin{column}{0.31\textwidth}
      \begin{block}{Implicit solution of localized nonlinearities}
        Localized nonsmooth processes play a leading role in many geophysical problems, e.g.,
        \begin{itemize}
        \item plastic yielding, fracture
        \item frictional contact: faults, sub-glacial
        \item contact/collisions: marine glaciers, sedimentation
        \item phase change: ice/water/steam, magma
        \end{itemize}
        If the effects are primarily \emph{local} (e.g., wetting and drying in coastal inundation), the nonsmoothness can be treated explicitly.
        But long-range stress transmission is instantaneous on the time scales of most geophysical problems, necessitating \emph{implicit} treatment if time steps are to be chosen based on accuracy rather than stability.
      \end{block}
      \vspace{-2em}
      \begin{block}{Nonlinear solvers}
        The prevailing nonlinear solution algorithms are based on global linearization, using either Newton or Picard iteration.
        \begin{gather*}
          F(u) = 0 \\
          \text{Solve:}\quad J(u) v = -F(u), \quad u \gets u + v \\
          \text{where}\quad J(u) \approx \nabla_u F(u)
        \end{gather*}
        \begin{itemize}
        \item Each iteration requires a global linear solve (e.g., Krylov-Multigrid).
        \item Each iteration moves important information over large distances.
        \item Superlinear convergence not realized for nonsmooth problems.
        \item The number of iterations depends on the strength of the nonlinearity.
        \end{itemize}
      \end{block}
      \vspace{-2em}
      \begin{block}{Model problem: $\mathfrak p$-Laplacian with friction}        
        \begin{columns}
          \begin{column}{0.7\textwidth}
            \begin{itemize}
            \item 2-dimensional model problem for power-law fluid cross-section, $1 \le \pfrak \le \infty$
              \begin{gather*}
                -\nabla\cdot (\eta \nabla u) - f = 0 \\
                \eta(\gamma) = \gamma_0(x) (\epsilon^2 + \gamma)^{\frac{\pfrak-2}{2}} \qquad \gamma(u) = \frac 1 2 \abs{\nabla u}^2
              \end{gather*}
            \item Friction boundary condition, $0 \le q \le 1$
              \begin{gather*}
                \nabla u \cdot \bm n + A(x) \abs{u}^{q-1} u = 0
              \end{gather*}
            \end{itemize}
          \end{column}
          \begin{column}{0.3\textwidth}
            \begin{figure}
              \includegraphics[width=\textwidth]{figures/MG/newton-convergence.png}
              \caption{Convergence of residual norm.}
            \end{figure}
          \end{column}
        \end{columns}

        \begin{figure}
          \centering
          \subfloat[It 65]{\includegraphics[width=0.3\textwidth]{figures/MG/ex15-friction/visit0011.png}} ~
          \subfloat[It 74]{\includegraphics[width=0.3\textwidth]{figures/MG/ex15-friction/visit0012.png}} ~
          \subfloat[It 115]{\includegraphics[width=0.3\textwidth]{figures/MG/ex15-friction/visit0015.png}}
          \caption{Convergence of heterogeneous $\pfrak = 1.3$, $\gamma_0 \in [10^{-2},1]$ with $q = 0.2$ friction at right boundary.}
          \label{fig:plaplace-friction}
        \end{figure}
      \end{block}
    \end{column}
    %
    \begin{column}{0.31\textwidth}
      \begin{block}{Heterogeneous media: the bane of adaptive mesh refinement}
        \begin{figure}
          \centering
          \subfloat[Zagros Mtns {\scriptsize [Yamato et al (2011)]}]{\includegraphics[width=0.49\textwidth]{figures/davemay/MayZagrosMountains.png}} ~
          \subfloat[Layered granite and diorite on Mt Moffit]{\includegraphics[width=0.49\textwidth]{figures/MoffitLayers.jpg}}
          \caption{Geology is complex at all scales}
          \label{fig:moffit}
        \end{figure}
        \begin{itemize}
        \item Adaptive spatial discretizations coarsen where acceptable accuracy can be achieved on coarse grids.
        \item Heterogeneous media requires high resolution throughout the domain.
        \end{itemize}
      \end{block}
      \vspace{-2em}
      \begin{block}{Full Approximation Scheme and $\tau$ corrections}
        The Full Approximation Scheme is a naturally nonlinear multigrid algorithm that allows flexible incorporation of multilevel information.
        \begin{itemize}
        \item classical formulation: ``coarse grid \emph{accelerates} fine grid solution''
        \item $\tau$ formulation: ``fine grid improves accuracy of coarse grid''
        \item To solve $N u = f$, recursively apply
          \begin{equation*}
            \begin{split}
              \text{pre-smooth} \:\: & \quad \tilde u^h \gets S^h_{\text{pre}}(u^h_0, f^h) \\
              \text{solve coarse problem for $u^H$} \:\: & \quad N^H u^H = \underbrace{I_h^H f^h}_{f^H} + \underbrace{N^H \hat I_h^H \tilde u^h - I_h^H N^h \tilde u^h}_{\tau_h^H} \\
              \text{correction and post-smooth} \:\: & \quad u^h \gets S^h_{\text{post}} \Big( \tilde u^h + I_H^h (u^H - \hat I_h^H \tilde u^h), f^h \Big) \\
            \end{split}
          \end{equation*}
          \begin{tabular}{llll}
            \toprule
            $I_h^H$ & residual restriction & $\hat I_h^H$ & solution restriction \\
            $I_H^h$ & solution interpolation & $f^H = I_h^H f^h$ & restricted forcing \\
            $\{S^h_{\text{pre}},S^h_{\text{post}}\}$ & \multicolumn{3}{l}{smoothing operations on the fine grid} \\
            \bottomrule
          \end{tabular}
        \item At convergence, $u^{H*} = \hat I_h^H u^{h*}$ solves the $\tau$-corrected coarse grid equation
            $N^H u^H = f^H + \tau_h^H$,
          thus $\tau_h^H$ is the ``fine grid feedback'' that makes the coarse grid equation accurate.
        % \item $\tau_h^H$ is \emph{local} and need only be recomputed where it becomes stale.
        \end{itemize}
      \end{block}
      \vspace{-2em}
      \begin{block}{Removing data dependencies with Segmental Refinement}
        \begin{columns}
          \begin{column}{0.7\textwidth}
            % Introduction of overlap~\cite{adams2014segmental,brandt1994multigrid} enables removal of ``horizontal'' communication in the multigrid cycle.
            % \begin{itemize}
            % \item Rely strictly on coarse grid to move information.
            % \item Robust coarsening crucial.
            % \end{itemize}
              \centering
              \includegraphics[width=0.8\textwidth]{figures/MG/weak_scaling_edison-eps-converted-to.pdf}
          \end{column}
          \begin{column}{0.3\textwidth}
              \centering
              \includegraphics[width=0.7\textwidth]{figures/MG/LowCommunication}
          \end{column}
        \end{columns}
        Introduce overlap to avoid horizontal communication in fine-grid visits.~\cite{adams2014segmental}
      \end{block}

      % \begin{block}{Nonlinearity, heterogeneity, and frozen $\tau$ techniques}
      %   Strong material nonlinearities such as plasticity cause methods based on global linearization, such as Newton and Picard, to require many iterations.
      %   Nonlinear multigrid avoids global linearization, leading to faster convergence rates when effective nonlinear smoothers are available.
      %   With a nonlinear smoother, we naturally want to build interpolation and the coarse operator without global assembly of a fine-grid operator.
      %   Unfortunately, traditional geometric multigrid does not accurately interpolate low-frequency modes and rediscretized coarse operators are notoriously inaccurate in highly heterogeneous cases.
      %   A subdomain-centric coarse grid construction only involves solving local problems, thus allowing it to be updated only in regions with rapidly-changing nonlinearities.
      % \end{block}

      % \begin{block}{Subdomain-centric matrix-free coarsening}
      %   {\bf Objective}: construct robust interpolation and coarse grid operator using only (a) local neighbor information, (b) application of local nonlinear operator, (c) point-block diagonal of principle linearization, and (d) application of triangular distribution operator or splitting~\cite{bkmms2012} for saddle points.
      %   \begin{enumerate}
      %   \item Select subdomains to become ``coarse elements'', add minimal stable node set to preliminary set of coarse dofs $C$.
      %   \item If available, add approximate null space to set of ``low-energy'' modes $B$ that must be approximated accurately.
      %   \item Use compatible relaxation with point-block preconditioned polynomial smoother to determine deficiencies of initial coarse basis.
      %   \item Enrich $C$ by adding poorly-converging points.
      %   \item Optimize energy of local basis functions by computing partition of coarse space $B$ on the boundary, then (approximately) harmonically extending to subdomain interior.
      %   \item Optionally, use (non-local) bootstrap cycle~\cite{brandt2011bootstrap} to improve $B$.
      %   \end{enumerate}
      % \end{block}
      % \vspace{-2.4em}
      % \begin{block}{Low-communication cycling}
      %       \begin{wrapfigure}{r}{0.45\textwidth}
      %         \includegraphics[width=0.4\textwidth]{figures/MG/TauDeps.pdf}
      %       \end{wrapfigure}
      %       The $\tau$ formulation removes communication from all levels except the coarsest.
      %       Instead of starting and ending on the fine grid, a cycle starts and ends on the coarse grid.
      %       The figure shows the dependency graph of 3-level multigrid cycle that computes the correction $\tau_1^0$ (red) on the coarse grid equation starting with coarse grid state $u_0$ (blue).
      %       A traditional multigrid cycle which has ``horizontal'' dependencies at every level.
      % \end{block}
    \end{column}
    %
    \begin{column}{0.31\textwidth}
      \begin{block}{$\tau$-adaptivity}
        \begin{figure}
          \centering
          \subfloat[Initial solution]{\includegraphics[width=0.18\textwidth]{figures/MG/ElasticityCompressTrim}} ~
          \subfloat[Increment]{\includegraphics[width=0.18\textwidth]{figures/MG/ElasticityCompressShearTrim}} ~
          \subfloat[Smoothed error (no $\tau$)]{\includegraphics[width=0.28\textwidth]{figures/MG/ElasticityCompressErrorNoTauTrim}} ~
          \subfloat[Smoothed error (with $\tau$)]{\includegraphics[width=0.28\textwidth]{figures/MG/ElasticityCompressErrorTauTrim}}
          \caption{Heterogeneous strain test using 2-level multigrid with coarsening factor of $3^2$.
            The coarse (respectively fine) grid has 3 (9) $Q_1$ elements across each block and 2 (6) elements across each gap.
            Panes (a) and (b) show the deformed body colored by strain.
            The initial problem of compression by 0.2 from the right is solved (a) and $\tau = A^H \hat I_h^H u^h - I_h^H A^h u^h$ is computed.
            Then a shear increment of 0.1 in the $y$ direction is added to the boundary condition, and the coarse-level problem is resolved, interpolated to the fine-grid, and a post-smoother is applied.
            When the coarse problem is solved without a $\tau$ correction (c), the displacement error is nearly $10\times$ larger than when $\tau$ is included in the right hand side of the coarse problem (d).
          }\label{fig:tau-valid}
        \end{figure}
        \begin{center}
          {\LARGE \bf Only visit fine grid where $\tau$ is ``stale''. }
        \end{center}
      \end{block}
      \vspace{-2em}
      \begin{block}{Comparison to nonlinear domain decomposition}
        \begin{itemize}
        \item ASPIN (Additive Schwarz preconditioned inexact Newton) \cite{cai2003npi} \\
          \begin{itemize}
          \item More local iterations in strongly nonlinear regions
          \item Each nonlinear iteration only propagates information locally
          \item Many real nonlinearities are activated by long-range forces
            \begin{itemize}
            \item faults, friction, locking in granular media
            \end{itemize}
          \item Two-stage algorithm has different load balancing
            \begin{itemize}
            \item Nonlinear subdomain solves
            \item Global linear solve
            \end{itemize}
          \end{itemize}
        \item $\tau$ adaptivity
          \begin{itemize}
          \item Minimum effort to communicate long-range information
          \item Nonlinearity sees effects as accurate as with global fine-grid feedback
          \item Fine-grid work always proportional to ``interesting'' changes
          \end{itemize}
        \end{itemize}
      \end{block}
      \vspace{-2em}
      \begin{block}{Status}
        \begin{itemize}
        \item Running proof of concept experiments
        \item Library implementation underway
        \item Need dynamic load balancing
        \item Need locally computable estimates for refreshing $\tau$
        \item Robust local coarsening, perhaps GenEO~\cite{spillane2011abstract,jolivet2013scalabledd}
        \end{itemize}
      \end{block}
      \vspace{-2em}
      \begin{block}{References}
        \scriptsize
        \nocite{adams2014segmental,brandt1984,brandt1994multigrid}
        \begin{minipage}{\textwidth}
          \begin{multicols}{2}
            \bibliography{$HOME/jedbib/jedbib,$PETSC_DIR/src/docs/tex/petsc,$PETSC_DIR/src/docs/tex/petscapp}
          \end{multicols}
        \end{minipage}
      \end{block}
    \end{column}
  \end{columns}
\end{frame}
\end{document}

        % Recursive application of a technique called \emph{segmental refinement}, in which the fine grid state is populated on the fly from the coarser grid followed by smoothing, permits solution of problems using only poly-logarithmic memory, provided that the right hand side and coefficients are similarly compressible.
        % For highly heterogenous problems, the constants associated with this aggressive memory reduction become large, and we have to choose which ingredients are most crucial to a robust multigrid cycle.
        % \begin{description}
        % \item[Solution state] is small and frequently needed in complete form due to coupling to other equations and time-dependence.
        %   It is relatively expensive to reconstruct on the fly, especially on multiple coarse levels.
        % \item[Grid transfer operators] (coarse basis functions) are a critical ingredient for robust multigrid and also expensive to reconstruct in each visit.
        % \item[Coarse operators] provide a systematic compression of fine-scale heterogeneity.
        %   With a fast coarsening rate, coarse grid operators are relatively affordable.
        % \item[Fine grid operators] typically consume significantly more memory than the rest of the multigrid hierarchy, are performance-limited by memory bandwidth, and too expensive to reassemble for highly nonlinear problems.
        % \end{description}

%%%%%%%%%%%%%%%%%%%%%%%%%%%%%%%%%%%%%%%%%%%%%%%%%%%%%%%%%%%%%%%%%%%%%%%%%%%%%%%%%%%%%%%%%%%%%%%%%%%% 
%%% Local Variables: 
%%% mode: latex
%%% TeX-PDF-mode: t
%%% End: 
