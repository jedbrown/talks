% \documentclass[handout]{beamer}
\documentclass{beamer}

\mode<presentation>
{
  \usetheme{default}
  % \usefonttheme[onlymath]{serif}
  % \usetheme{Singapore}
  % \usetheme{Warsaw}
  % \usetheme{Malmoe}
  % \useinnertheme{circles}
  % \useoutertheme{infolines}
  % \useinnertheme{rounded}

  \setbeamercovered{transparent=20}
}

\usepackage[english]{babel}
\usepackage[latin1]{inputenc}
\usepackage{alltt,listings,multirow,ulem,siunitx}
\usepackage[absolute,overlay]{textpos}
\TPGrid{1}{1}
\usepackage{pdfpages}
\usepackage{ulem}
\usepackage{multimedia}
\usepackage{multicol}
\newcommand\hmmax{0}
\newcommand\bmmax{0}
\usepackage{bm}
\usepackage{comment}
\usepackage{subcaption}

% font definitions, try \usepackage{ae} instead of the following
% three lines if you don't like this look
\usepackage{mathptmx}
\usepackage[scaled=.90]{helvet}
% \usepackage{courier}
\usepackage[T1]{fontenc}
\usepackage{tikz}
\usetikzlibrary{decorations.pathreplacing}
\usetikzlibrary{shadows,arrows,shapes.misc,shapes.arrows,shapes.multipart,arrows,decorations.pathmorphing,backgrounds,positioning,fit,petri,calc,shadows,chains,matrix}

\newcommand\vvec{\bm v}
\newcommand\bvec{\bm b}
\newcommand\bxk{\bvec_0 \times \kappa_0 \cdot \nabla}
\newcommand\delp{\nabla_\perp}
\newcommand{\matr}[1]{\bm{#1}}

% \usepackage{pgfpages}
% \pgfpagesuselayout{4 on 1}[a4paper,landscape,border shrink=5mm]

\usepackage{JedMacros}

\newcommand{\timeR}{t_{\mathrm{R}}}
\newcommand{\timeW}{t_{\mathrm{W}}}
\newcommand{\mglevel}{\ensuremath{\ell}}
\newcommand{\mglevelcp}{\ensuremath{\mglevel_{\mathrm{cp}}}}
\newcommand{\mglevelcoarse}{\ensuremath{\mglevel_{\mathrm{coarse}}}}
\newcommand{\mglevelfine}{\ensuremath{\mglevel_{\mathrm{fine}}}}

%solution and residual
\newcommand{\vx}{\ensuremath{x}}
\newcommand{\vc}{\ensuremath{\hat{x}}}
\newcommand{\vr}{\ensuremath{r}}
\newcommand{\vb}{\ensuremath{b}}

%operators
\newcommand{\vA}{\ensuremath{A}}
\newcommand{\vP}{\ensuremath{I_H^h}}
\newcommand{\vS}{\ensuremath{S}}
\newcommand{\vR}{\ensuremath{I_h^H}}
\newcommand{\vI}{\ensuremath{\hat I_h^H}}
\newcommand{\vV}{\ensuremath{\mathbf{V}}}
\newcommand{\vF}{\ensuremath{F}}
\newcommand{\vtau}{\ensuremath{\mathbf{\tau}}}


\title{Active learning for cost-aware model reduction}
\author{Dmitry Duplyakin (U Utah and CU), {\bf Jed Brown} (CU Boulder), Donna Calhoun (Boise State) \\
  \texttt{jed.brown@colorado.edu}}

% - Use the \inst command only if there are several affiliations.
% - Keep it simple, no one is interested in your street address.
% \institute
% {
%   Mathematics and Computer Science Division \\ Argonne National Laboratory
% }

\date{Copper Mountain Conference on Iterative Methods \\
  {\small \url{https://jedbrown.org/files/20180327-ActiveLearning.pdf}}}

% This is only inserted into the PDF information catalog. Can be left
% out.
\subject{Talks}


% If you have a file called "university-logo-filename.xxx", where xxx
% is a graphic format that can be processed by latex or pdflatex,
% resp., then you can add a logo as follows:

% \pgfdeclareimage[height=0.5cm]{university-logo}{university-logo-filename}
% \logo{\pgfuseimage{university-logo}}



% Delete this, if you do not want the table of contents to pop up at
% the beginning of each subsection:
% \AtBeginSubsection[]
% {
% \begin{frame}<beamer>
%   \frametitle{Outline}
%   \tableofcontents[currentsection,currentsubsection]
% \end{frame}
% }

\AtBeginSection[]
{
  \begin{frame}<beamer>
    \frametitle{Outline}
    \tableofcontents[currentsection]
  \end{frame}
}

% If you wish to uncover everything in a step-wise fashion, uncomment
% the following command:

% \beamerdefaultoverlayspecification{<+->}

\begin{document}
\lstset{language=C}
\normalem

\begin{frame}
  \titlepage
\end{frame}

\begin{frame}{AMR shock-bubble with 4, 5, and 6 levels}
  \begin{figure}
    \centering
    \begin{subfigure}[b]{0.9\columnwidth}
      \includegraphics[width=1.0\columnwidth]{figures/al-amr/shockbubble_level_4a.pdf}
    \end{subfigure}
    \begin{subfigure}[b]{0.9\columnwidth}
      \includegraphics[width=1.0\columnwidth]{figures/al-amr/shockbubble_level_5a.pdf}
    \end{subfigure}
    \centering 
    \begin{subfigure}[b]{0.9\columnwidth}
      \includegraphics[width=1.0\columnwidth]{figures/al-amr/shockbubble_level_6a.pdf}
    \end{subfigure}
    \label{fig:shockbubble}
  \end{figure}
\end{frame}

\begin{frame}{What goes wrong?}
{\small \texttt{
=================================================================================== \\
=   BAD TERMINATION OF ONE OF YOUR APPLICATION PROCESSES \\
=   PID 17873 RUNNING AT joule \\
=   EXIT CODE: 139 \\
=   CLEANING UP REMAINING PROCESSES \\
=   YOU CAN IGNORE THE BELOW CLEANUP MESSAGES \\
=================================================================================== \\
YOUR APPLICATION TERMINATED WITH THE EXIT STRING: Segmentation fault (signal 11) \\
}}
\begin{itemize}
\item Resubmit batch job, this time using more nodes.
\item Tweak a refinement parameter.
\end{itemize}
\end{frame}
\begin{frame}{On the last time step?}
{\small \texttt{
=================================================================================== \\
=   BAD TERMINATION OF ONE OF YOUR APPLICATION PROCESSES \\
=   PID 16824 RUNNING AT joule \\
=   EXIT CODE: 152 \\
=   CLEANING UP REMAINING PROCESSES \\
=   YOU CAN IGNORE THE BELOW CLEANUP MESSAGES \\
=================================================================================== \\
YOUR APPLICATION TERMINATED WITH THE EXIT STRING: CPU time limit exceeded (signal 24) \\
}}
\begin{itemize}
\item Checkpoint more often?
\item Need to wait through the queue again.
\end{itemize}
\end{frame}

\begin{frame}{New computer}
  \begin{center}
    \includegraphics[width=\textwidth]{figures/hpgmg/hpgmg-fv-201706.png}
  \end{center}
\end{frame}

\begin{frame}{Modeling}
  \begin{equation*}
    \textrm{(response)} = f(x) + \mathcal N(0, \sigma_n^2)
  \end{equation*}
  \begin{description}
  \item[$x$] user-relevant parameters
    \begin{itemize}
    \item Physics: bubble size, density, shock intensity
    \item Numerics: box size, max levels, refinement criteria
    \item Machine: \# nodes, MPI/OpenMP, compilers
    \end{itemize}
  \item[$f$] Response
    \begin{itemize}
    \item CPU time
    \item Wall-clock time
    \item Peak memory usage
    \item Physics: $\Delta$ entropy, decay time
    \end{itemize}
  \item[$\sigma_n$] unbiased Gaussian noise
  \end{description}
\end{frame}

\begin{frame}{LOL Gaussian!}
  \begin{center}
    \includegraphics[width=\textwidth]{figures/HoeflerBelli-HPL50.png} \\
    {\small [Hoefler \& Belli, SC15]}
  \end{center}
\end{frame}

\begin{frame}{Gaussian process regression}
  \begin{equation*}
    p\big( f(\matr{x}_\star)~|~\matr{X},\matr{y} \big) \sim 
    \mathcal{N} \left( \matr{\mu}_\star, \matr{\sigma}_{\star}^2 \right)
  \end{equation*}
  \begin{align*}	
    \matr{\mu}_\star &= \matr{k}_\star^\mathsf{T} \matr{K}_y ^{-1} \matr{y} \\
    \matr{\sigma}^2_\star  &= \matr{k}_{\star\star} - \matr{k}_\star^\mathsf{T} \matr{K}_y^{-1} \matr{k}_\star \\
    \matr{K}_y &= \matr{K} + \sigma^2_n\matr{I}
  \end{align*}
  where
  \begin{align*}
    [\matr{K}]_{ij} = k(\matr{x}_i, \matr{x}_j) &= \sigma^2_f  \mathsf{exp}\left( - \frac{ | \matr{x}_i-\matr{x}_j | ^2}{2\ell^2}  \right)
  \end{align*}
  in terms of \emph{hyperparameters} $\ell, \sigma_f, \sigma_n$.
\end{frame}

\begin{frame}{Optimizing hyperparameters}
  The evidence provided by $(\matr X, \matr y)$ in support of $(\ell, \sigma_f, \sigma_n)$ is quantified by
  \begin{equation*}
    \mathcal L(\ell, \sigma_f, \sigma_n) = \log p(\matr{y}~|~\matr{X},l,\sigma^2_f,\sigma^2_n)= -\frac{1}{2}\left( \matr{y}^\mathsf{T}\matr{K}^{-1}_y \matr{y} + \log \lvert\matr{K}_y\rvert \right) + \matr{C}
  \end{equation*}
  \begin{itemize}
  \item Non-convex optimization problem
  \item Determinant $\lvert\matr{K}_y\rvert$ due to normalization
  \end{itemize}
\end{frame}

\begin{frame}{Offline Active Learning}
  \begin{itemize}
  \item Precompute database of features and responses
  \item Partition data $(\matr X, \matr y)$ into Initial, Active, Test
  \item Compare many ``trajectories'' using different partitions
  \end{itemize}
  \begin{block}{Algorithm}
    \begin{itemize}
    \item Train GPR for each feature (e.g., cost and memory) in Initial set
    \item Repeat
      \begin{enumerate}
      \item Consult GPR models to select next observation from Active
      \item Make that observation (incurring cost, etc.)
      \item Retrain GPRs with new observation (including failure)
      \end{enumerate}
    \end{itemize}
  \end{block}
\end{frame}

\begin{frame}{Selection procedure}
  \begin{center}
    \includegraphics[width=\textwidth]{figures/al-amr/sequence-150-only_violin.pdf}
  \end{center}
  \begin{description}
  \item[RandUniform] Uniform random sampling
  \item[MaxSigma] Choose candidate with largest uncertainty
  \item[MinPred] Maximize $\sigma_{\text{cost}} / \mu_{\text{cost}}$
  \item[RandGoodness] Probability density $\sim \sigma_{\text{cost}} / \mu_{\text{cost}}$
  \item[RandGoodness with Memory Awarness] As above, but exclude cases that violate memory bound
  \end{description}
\end{frame}

\begin{frame}{Metrics}
  \begin{description}
  \item[Accuracy] $$ RMSE = \sqrt{\frac 1 n_{\text{Test}} \bm e^T e } $$
    where $\bm e = \bm{\mu}^{\text{Test}} - \bm{y}^{\text{Test}}$
  \item[Cumulative Cost] Total cost of selected experiments, including failures
  \item[Cumulative Regret] Costs incurred attempting failures
  \end{description}
\end{frame}

\begin{frame}{Memory limits}
  \begin{center}
    \includegraphics[width=\textwidth]{figures/al-amr/partitions.pdf}
  \end{center}
\end{frame}

\begin{frame}{Cumulative Regret by Algorithm}
  \begin{center}
    \includegraphics[width=\textwidth]{figures/al-amr/CR-by-alg.pdf}
  \end{center}
\end{frame}

\begin{frame}{Cumulative Regret by nInit}
  \begin{center}
    \includegraphics[width=\textwidth]{figures/al-amr/CR-by-nInit.pdf}
  \end{center}
\end{frame}

\begin{frame}{RMSE vs Cumulative Cost}
  \begin{center}
     \includegraphics[width=\textwidth]{figures/al-amr/rmse-RGMA-and-RG.pdf}
  \end{center}
\end{frame}
    
\begin{frame}{RMSE vs Cumulative Cost}
  \begin{center}
    \includegraphics[width=\textwidth]{figures/al-amr/ProgressAndTradeoff/complete-dataset/tradeoff-595-up-to-60.png}
  \end{center}
\end{frame}

\begin{frame}{RMSE vs Cumulative Cost}
  \begin{center}
    \includegraphics[width=\textwidth]{figures/al-amr/rmse.pdf}
  \end{center}
\end{frame}

\begin{frame}{Outlook}
  \begin{itemize}
  \item Better measure of ``risk'' of exceeding memory bound
  \item Domain-specific kernel functions
  \item Non-Gaussian distributions
  \item Online mode
  \item Thanks to NERSC Edison and Cori, and to DOE ASCR
  \end{itemize}
\end{frame}
\end{document}
