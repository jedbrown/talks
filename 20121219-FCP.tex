% \documentclass[handout]{beamer}
\documentclass{beamer}

\mode<presentation>
{
  \usetheme{default}
  \usefonttheme[onlymath]{serif}
  % \usetheme{Singapore}
  % \usetheme{Warsaw}
  % \usetheme{Malmoe}
  % \useinnertheme{circles}
  % \useoutertheme{infolines}
  % \useinnertheme{rounded}

  \setbeamercovered{transparent=100}
}

\usepackage[english]{babel}
\usepackage[latin1]{inputenc}
\usepackage{alltt,listings,multirow,ulem,siunitx}
\usepackage[absolute,overlay]{textpos}
\TPGrid{1}{1}
\usepackage{pdfpages}
\usepackage{multimedia}
\usepackage{multicol}
\newcommand\hmmax{0}
\newcommand\bmmax{0}
\usepackage{bm}

% font definitions, try \usepackage{ae} instead of the following
% three lines if you don't like this look
\usepackage{mathptmx}
\usepackage[scaled=.90]{helvet}
% \usepackage{courier}
\usepackage[T1]{fontenc}
\usepackage{tikz}
\usetikzlibrary{decorations.pathreplacing}
\usetikzlibrary{shadows,arrows,shapes.misc,shapes.arrows,shapes.multipart,arrows,decorations.pathmorphing,backgrounds,positioning,fit,petri,calc,shadows,chains,matrix}


% \usepackage{pgfpages}
% \pgfpagesuselayout{4 on 1}[a4paper,landscape,border shrink=5mm]

\usepackage{JedMacros}

\title{Multilevel solvers with adaptive coarse space construction for lithosphere dynamics}
\author{{\bf Jed Brown}\inst{1}, Mark Adams\inst{2}, Matt Knepley\inst{3}, Barry Smith\inst{1}}

% - Use the \inst command only if there are several affiliations.
% - Keep it simple, no one is interested in your street address.
\institute
{
  \inst{1}{Mathematics and Computer Science Division, Argonne National Laboratory} \\
  \inst{2}{Columbia University} \\
  \inst{3}{University of Chicago} \\
}

\date{Frontiers in Computational Physics, 2012-12-19}

% This is only inserted into the PDF information catalog. Can be left
% out.
\subject{Talks}


% If you have a file called "university-logo-filename.xxx", where xxx
% is a graphic format that can be processed by latex or pdflatex,
% resp., then you can add a logo as follows:

% \pgfdeclareimage[height=0.5cm]{university-logo}{university-logo-filename}
% \logo{\pgfuseimage{university-logo}}



% Delete this, if you do not want the table of contents to pop up at
% the beginning of each subsection:
% \AtBeginSubsection[]
% {
% \begin{frame}<beamer>
%   \frametitle{Outline}
%   \tableofcontents[currentsection,currentsubsection]
% \end{frame}
% }

\AtBeginSection[]
{
  \begin{frame}<beamer>
    \frametitle{Outline}
    \tableofcontents[currentsection]
  \end{frame}
}

% If you wish to uncover everything in a step-wise fashion, uncomment
% the following command:

% \beamerdefaultoverlayspecification{<+->}

\begin{document}
\lstset{language=C}
\normalem

\begin{frame}
  \titlepage
\end{frame}

\begin{frame}{The Great Solver Schism: Monolithic or Split?}
  \begin{columns}
    \begin{column}{0.5\textwidth}
      \begin{block}{Monolithic}
        \begin{itemize}
        \item Direct solvers
        \item Coupled Schwarz
        \item Coupled Neumann-Neumann \\
          (need unassembled matrices)
        \item Coupled multigrid
        \item[X] Need to understand local spectral and compatibility properties of the coupled system
        \end{itemize}
      \end{block}
    \end{column}
    \begin{column}{0.5\textwidth}
      \begin{block}{Split}
        \begin{itemize}
        \item Physics-split Schwarz \\
          (based on relaxation)
        \item Physics-split Schur \\
          (based on factorization)
          \begin{itemize}
          \item  approximate commutators \\
            SIMPLE, PCD, LSC
          \item segregated smoothers
          \item Augmented Lagrangian
          \item ``parabolization'' for stiff waves
          \end{itemize}
        \item[X] Need to understand global coupling strengths
        \end{itemize}
      \end{block}
    \end{column}
  \end{columns}
  \begin{itemize}
  \item Preferred data structures depend on which method is used.
  \item Interplay with geometric multigrid.
  \end{itemize}
\end{frame}

\begin{frame}{Status quo for implicit solves in lithosphere dynamics}
  \begin{itemize}
  \item global linearization using Newton or Picard
  \item assembly of a sparse matrix
  \item ``block'' factorization preconditioner with approximate Schur complement
  \item algebraic or geometric multigrid on positive-definite systems
  \end{itemize}
  \begin{block}{Why is this bad?}
    \vspace{-1em}
    \begin{itemize}
    \item nonlinearities (e.g., plastic yield) are mostly local
      \begin{itemize}
      \item feed back through nearly linear large scales
      \item frequent visits to fine-scales even in nearly-linear regions
      \item no way to locally update coarse grid operator
      \item Newton linearization introduces anisotropy
      \end{itemize}
    \item assembled sparse matrices are terrible for performance on modern hardware
      \begin{itemize}
      \item memory bandwidth is very expensive compared to flops
      \item fine-scale assembly costs a lot of memory
      \item assembled matrices are good for algorithmic experimentation
      \end{itemize}
    \item block preconditioners require more parallel communication
    \end{itemize}
  \end{block}
\end{frame}

\begin{frame}{Hardware Arithmetic Intensity}
  \begin{tabular}{lc}
    \toprule
    Operation                         & Arithmetic Intensity (flops/B) \\
    \midrule
    Sparse matrix-vector product      & 1/6                  \\
    Dense matrix-vector product       & 1/4                  \\
    Unassembled matrix-vector product & $\approx 8$          \\
    High-order residual evaluation    & $> 5$                \\
    \bottomrule
  \end{tabular}

  \bigskip

  \begin{tabular}{lrrr}
    \toprule
    Processor & BW (GB/s) & Peak (GF/s) & Balanced AI (F/B) \\
    \midrule
    E5-2670 8-core      & 35   & 166  & 4.7 \\
    Magny Cours 16-core & 49   & 281  & 5.7 \\
    Blue Gene/Q node    & 43   & 205  & 4.8 \\
    Tesla M2090         & 120  & 665  & 5.5 \\
    Kepler K20Xm        & 160 & 1310 & 8.2 \\ % http://www.elekslabs.com/2012/11/nvidia-tesla-k20-benchmark-facts.html
    Xeon Phi            & 150 & 1248 & 8.3 \\
    \bottomrule
  \end{tabular}
\end{frame}

\begin{frame}[shrink=5]{Performance of assembled versus unassembled}
  \includegraphics[width=\textwidth]{figures/TensorVsAssembly} \\
  \begin{itemize}
  \item High order Jacobian stored unassembled using coefficients at quadrature points, can use local AD
  \item Choose approximation order at run-time, independent for each field
  \item Precondition high order using assembled lowest order method
  \item Implementation $> 70\%$ of FPU peak, SpMV bandwidth wall $< 4\%$
  \end{itemize}
\end{frame}


\begin{frame}{$\tau$ formulation of Full Approximation Scheme (FAS)}
  \begin{itemize}
  \item classical formulation: ``coarse grid \emph{accelerates} fine grid solution''
  \item $\tau$ formulation: ``fine grid improves accuracy of coarse grid''
  \item To solve $N u = f$, recursively apply
    \begin{equation*}
      \begin{split}
        \text{pre-smooth} \:\: & \quad \tilde u^h \gets S^h_{\text{pre}}(u^h_0, f^h) \\
        \text{solve coarse problem for $u^H$} \:\: & \quad N^H u^H = f^H + \underbrace{N^H \hat I_h^H \tilde u^h - I_h^H N^h \tilde u^h}_{\tau_h^H} \\
        \text{correction and post-smooth} \:\: & \quad u^h \gets S^h_{\text{post}} \Big( \tilde u^h + I_H^h (u^H - \hat I_h^H \tilde u^h), f^h \Big) \\
      \end{split}
    \end{equation*}
    \begin{tabular}{ll}
      \toprule
      $I_h^H$ & residual restriction \\
      $\hat I_h^H$ & solution restriction \\
      $I_H^h$ & solution interpolation \\
      $f^H = I_h^H f^h$ & restriction of forcing term \\
      $\{S^h_{\text{pre}},S^h_{\text{post}}\}$ & smoothing operations on the fine grid \\
      \bottomrule
    \end{tabular}
  \end{itemize}
\end{frame}

\begin{frame}{Multiscale compression and recovery using $\tau$}
  % \begin{tikzpicture}
  %   [>=stealth,
  %   every node/.style={inner sep=2pt},
  %   restrict/.style={thick,double},
  %   prolong/.style={thick,double},
  %   cprestrict/.style={green!50!black,thick,double,dashed},
  %   control/.style={rectangle,red!40!black,draw=red!40!black,thick},
  %   mglevel/.style={rounded rectangle,draw=blue!50!black,fill=blue!20,thick,minimum size=4mm},
  %   checkpoint/.style={rectangle,draw=green!50!black,fill=green!20,thick,minimum size=4mm},
  %   mglevelhide/.style={rounded rectangle,draw=gray!50!black,fill=gray!20,thick,minimum size=4mm},
  %   tau/.style={text=red!50!black,draw=red!50!black,fill=red!10,inner sep=1pt}
  %   ]
  %   \begin{scope}\scriptsize
  %     \newcommand\mgdx{2.1em}
  %     \newcommand\mgdy{1.9em}
  %     \newcommand\mgloc[4]{(#1 + #4*\mgdx*#3,#2 + \mgdy*#3)}
  %     \node[mglevel] (fine0) at \mgloc{0}{0}{4}{-1} {\mglevelfine};
  %     \node[mglevel] (finem1down0) at \mgloc{0}{0}{3}{-1} {};
  %     \node[mglevel] (cp1down0) at \mgloc{0}{0}{2}{-1} {$\mglevelcp+1$};
  %     \node[mglevel] (cpdown0) at \mgloc{0}{0}{1}{-1} {\mglevelcp};
  %     \node[mglevel] (coarser0) at \mgloc{0}{0}{0}{0} {\ldots};

  %     \node[mglevelhide] (cpup0) at \mgloc{0}{0}{1}{1} {};
  %     \node (cp1up0) at \mgloc{0}{0}{2}{1} {};

  %     \node (cpdown1) at \mgloc{4em}{0}{1}{-1} {};
  %     \node[mglevelhide] (coarser1) at \mgloc{4em}{0}{0}{1} {\ldots};
  %     \node[mglevel] (cpup1) at \mgloc{4em}{0}{1}{1} {\mglevelcp};
  %     \node[mglevel] (cp1up1) at \mgloc{4em}{0}{2}{1} {$\mglevelcp+1$};
  %     \node[mglevel] (finem1up1) at \mgloc{4em}{0}{3}{1} {};
  %     \node[mglevel] (fine1) at \mgloc{4em}{0}{4}{1} {\mglevelfine};

  %     \draw[->,restrict,dashed] (fine0) -- (finem1down0);
  %     \draw[->,restrict] (finem1down0) -- (cp1down0);
  %     \draw[->,restrict] (cp1down0) -- (cpdown0);
  %     \draw[->,restrict,dashed] (cpdown0) -- (coarser0);
  %     \draw[->,prolong,dashed] (coarser0) -- (cpup0);
  %     \draw[->,prolong,dashed] (cpup0) -- (cp1up0);

  %     \draw[->,restrict,dashed] (cpdown1) -- (coarser1);
  %     \draw[->,prolong,dashed] (coarser1) -- (cpup1);
  %     \draw[->,prolong] (cpup1) -- (cp1up1);
  %     \draw[->,prolong] (cp1up1) -- (finem1up1);
  %     \draw[->,prolong,dashed] (finem1up1) -- (fine1);

  %     \node[checkpoint] at (4em + \mgdx*4,\mgdy) (cp) {CP};
  %     \draw[>->,cprestrict] (fine1) -- node[below,sloped] {Restrict} (cp);

  %     \node[left=\mgdx of fine0] (bnanchor) {};
  %     \node[control,fill=red!20] at (1.1*\mgdx,3*\mgdy) {Solve $F(u^n;b^n) = 0$};
  %     \node[mglevel,right=of fine1] (finedt) {next solve};
  %     \draw[->, >=stealth, control] (fine1) to[out=20,in=170] node[above] {$b^{n+1}(u^n,b^n)$} (finedt);
  %     \draw[->, >=stealth, control] (bnanchor) to[out=45,in=155] node[above] {$b^n$} (fine0);

  %     % Recovery process
  %     \begin{scope}[xshift=7*\mgdx]
  %       \node[checkpoint] (rcp) at \mgloc{0}{0}{0}{0} {CP};
  %       \node[mglevel] (r0a) at \mgloc{0}{\mgdy}{0}{0} {CR};
  %       \node[mglevel] (r1a) at \mgloc{0}{\mgdy}{1}{1} {};
  %       \node[mglevel] (r0b) at \mgloc{2*\mgdx}{\mgdy}{0}{0} {CR};
  %       \node[mglevel] (r1b) at \mgloc{2*\mgdx}{\mgdy}{1}{1} {};
  %       \node[mglevel] (r2b) at \mgloc{2*\mgdx}{\mgdy}{2}{1} {\mglevelfine};
  %       \node[mglevel] (r1c) at \mgloc{6*\mgdx}{\mgdy}{1}{-1} {};
  %       \node[mglevel] (r0d) at \mgloc{6*\mgdx}{\mgdy}{0}{0} {CR};
  %       \node[mglevel] (r1d) at \mgloc{6*\mgdx}{\mgdy}{1}{1} {};
  %       \node[mglevel] (r2d) at \mgloc{6*\mgdx}{\mgdy}{2}{1} {\mglevelfine};

  %       \draw[-,prolong,green!50!black] (rcp) -- (r0a);
  %       \draw[->,prolong] (r0a) -- (r1a);
  %       \draw[->,restrict] (r1a) -- (r0b);
  %       \draw[->,restrict] (r0b) -- (r1b);
  %       \draw[->,restrict,dashed] (r1b) -- (r2b);
  %       \draw[->,restrict,dashed] (r2b) -- (r1c);
  %       \draw[->,restrict] (r1c) -- (r0d);
  %       \draw[->,restrict] (r0d) -- (r1d);
  %       \draw[->,restrict,dashed] (r1d) -- (r2d);

  %       \foreach \smooth in {finem1down0, cp1down0, cpdown0, coarser0,
  %         cpup1, cp1up1, finem1up1,
  %         r0b,r1c,r0d,r1d} {
  %         \node[above left=-5pt of \smooth.west,tau] {$\tau$};
  %       }
  %       \node[rectangle,fill=none,draw=green!50!black,thick,fit=(rcp)(r2d)] (recoverbox) {};
  %       \node[rectangle,draw=green!50!black,fill=green!20,thick,minimum size=6mm,above={0cm of recoverbox.south east},anchor=south east] (recover) {FMG Recovery};
  %     \end{scope}
  %     \node (notation) at (-7.5*\mgdx,2*\mgdy) {
  %       \tiny
  %       \begin{minipage}{22em}\raggedright \sf
  %         $\bullet$ checkpoint converged coarse state \\
  %         $\bullet$ recover using FMG anchored at $\mglevelcp+1$ \\
  %         $\bullet$ compatible relaxation (CR) as coarse solve \\
  %         $\bullet$ $\tau$ correction is local, only $\mglevelcp$ neighbor points \\
  %         $\bullet$ survivors continue MG cycles with stale $\tau$
  %       \end{minipage}
  %     };
  %   \end{scope}
  % \end{tikzpicture}
  \includegraphics[width=\textwidth]{FMGRecovery}
    \begin{itemize}
    \item Compress transient simulation with local decompression
    \item Remove communication from all but coarse grid
      \begin{itemize}
      \item Convergence speed not affected, modest redundant computation
      \end{itemize}
    \item In-situ visualization and reanalysis with very few full checkpoints
    \item Checkpointing for discrete adjoints
    \item Resiliency to hardware failure
    \end{itemize}
\end{frame}

\newcommand{\mglevelfine}{\ensuremath{\mglevel_{\mathrm{fine}}}}
\newcommand{\mglevel}{\ensuremath{\ell}}
\newcommand{\mglevelcp}{\ensuremath{\mglevel_{\mathrm{cp}}}}
\newcommand\mgdx{2.1em}
\newcommand\mgdy{2.2em}
\newcommand\mgloc[4]{(#1 + #4*\mgdx*#3,#2 + \mgdy*#3)}
\begin{frame}{Multiscale compression and recovery using $\tau$}
  \begin{tikzpicture}
    [scale=0.7,every node/.style={scale=0.7},
    >=stealth,
    restrict/.style={thick,double},
    prolong/.style={thick,double},
    cprestrict/.style={green!50!black,thick,double,dashed},
    control/.style={rectangle,red!40!black,draw=red!40!black,thick},
    mglevel/.style={rounded rectangle,draw=blue!50!black,fill=blue!20,thick,minimum size=6mm},
    checkpoint/.style={rectangle,draw=green!50!black,fill=green!20,thick,minimum size=6mm},
    mglevelhide/.style={rounded rectangle,draw=gray!50!black,fill=gray!20,thick,minimum size=6mm},
    tau/.style={text=red!50!black,draw=red!50!black,fill=red!10,inner sep=1pt},
    crelax/.style={text=green!50!black,fill=green!10,inner sep=0pt}
    ]
    \begin{scope}
      \node[mglevel] (fine0) at \mgloc{0}{0}{4}{-1} {\mglevelfine};
      \node[mglevel] (finem1down0) at \mgloc{0}{0}{3}{-1} {};
      \node[mglevel] (cp1down0) at \mgloc{0}{0}{2}{-1} {$\mglevelcp+1$};
      \node[mglevel] (cpdown0) at \mgloc{0}{0}{1}{-1} {\mglevelcp};
      \node[mglevel] (coarser0) at \mgloc{0}{0}{0}{0} {\ldots};

      \node[mglevelhide] (cpup0) at \mgloc{0}{0}{1}{1} {};
      \node (cp1up0) at \mgloc{0}{0}{2}{1} {};

      \node (cpdown1) at \mgloc{4em}{0}{1}{-1} {};
      \node[mglevelhide] (coarser1) at \mgloc{4em}{0}{0}{1} {\ldots};
      \node[mglevel] (cpup1) at \mgloc{4em}{0}{1}{1} {\mglevelcp};
      \node[mglevel] (cp1up1) at \mgloc{4em}{0}{2}{1} {$\mglevelcp+1$};
      \node[mglevel] (finem1up1) at \mgloc{4em}{0}{3}{1} {};
      \node[mglevel] (fine1) at \mgloc{4em}{0}{4}{1} {\mglevelfine};

      \draw[->,restrict,dashed] (fine0) -- (finem1down0);
      \draw[->,restrict] (finem1down0) -- (cp1down0);
      \draw[->,restrict] (cp1down0) -- (cpdown0);
      \draw[->,restrict,dashed] (cpdown0) -- (coarser0);
      \draw[->,prolong,dashed] (coarser0) -- (cpup0);
      \draw[->,prolong,dashed] (cpup0) -- (cp1up0);

      \draw[->,restrict,dashed] (cpdown1) -- (coarser1);
      \draw[->,prolong,dashed] (coarser1) -- (cpup1);
      \draw[->,prolong] (cpup1) -- (cp1up1);
      \draw[->,prolong] (cp1up1) -- (finem1up1);
      \draw[->,prolong,dashed] (finem1up1) -- (fine1);

      \node[checkpoint] at (4em + \mgdx*4,\mgdy) (cp) {CP};
      \draw[>->,cprestrict] (fine1) -- node[below,sloped] {Restrict} (cp);

      \node[left=\mgdx of fine0] (bnanchor) {};
      \node[control,fill=red!20] at (1.1*\mgdx,3*\mgdy) {Solve $F(u^n;b^n) = 0$};
      \node[mglevel,right=of fine1] (finedt) {next solve};
      \draw[->, >=stealth, control] (fine1) to[out=20,in=170] node[above] {$b^{n+1}(u^n,b^n)$} (finedt);
      \draw[->, >=stealth, control] (bnanchor) to[out=45,in=155] node[above] {$b^n$} (fine0);

      % Recovery process
      \begin{scope}[xshift=7.5*\mgdx]
        \node[checkpoint] (rcp) at \mgloc{0}{0}{0}{0} {CP};
        \node[mglevel] (r0a) at \mgloc{0}{\mgdy}{0}{0} {CR};
        \node[mglevel] (r1a) at \mgloc{0}{\mgdy}{1}{1} {};
        \node[mglevel] (r0b) at \mgloc{2*\mgdx}{\mgdy}{0}{0} {CR};
        \node[mglevel] (r1b) at \mgloc{2*\mgdx}{\mgdy}{1}{1} {};
        \node[mglevel] (r2b) at \mgloc{2*\mgdx}{\mgdy}{2}{1} {\mglevelfine};
        \node[mglevel] (r1c) at \mgloc{6*\mgdx}{\mgdy}{1}{-1} {};
        \node[mglevel] (r0d) at \mgloc{6*\mgdx}{\mgdy}{0}{0} {CR};
        \node[mglevel] (r1d) at \mgloc{6*\mgdx}{\mgdy}{1}{1} {};
        \node[mglevel] (r2d) at \mgloc{6*\mgdx}{\mgdy}{2}{1} {\mglevelfine};

        \draw[-,prolong,green!50!black] (rcp) -- (r0a);
        \draw[->,prolong] (r0a) -- (r1a);
        \draw[->,restrict] (r1a) -- (r0b);
        \draw[->,restrict] (r0b) -- (r1b);
        \draw[->,restrict,dashed] (r1b) -- (r2b);
        \draw[->,restrict,dashed] (r2b) -- (r1c);
        \draw[->,restrict] (r1c) -- (r0d);
        \draw[->,restrict] (r0d) -- (r1d);
        \draw[->,restrict,dashed] (r1d) -- (r2d);

        \foreach \smooth in {finem1down0, cp1down0, cpdown0, coarser0,
          cpup1, cp1up1, finem1up1,
          r0b,r1c,r0d,r1d} {
          \node[above left=-5pt of \smooth.west,tau] {$\tau$};
        }
        \node[rectangle,fill=none,draw=green!50!black,thick,fit=(rcp)(r2d)] (recoverbox) {};
        \node[rectangle,draw=green!50!black,fill=green!20,thick,minimum size=6mm,above={0cm of recoverbox.south east},anchor=south east] (recover) {FMG Decompression};
      \end{scope}
      \node (notation) at (2*\mgdx,5*\mgdy) {
        \begin{minipage}{18em}\small\sf
          \begin{itemize}\addtolength{\itemsep}{-5pt}
          \item checkpoint converged coarse state
          \item recover using FMG anchored at $\mglevelcp+1$
          \item needs only $\mglevelcp$ neighbor points
          \item $\tau$ correction is local
          \end{itemize}
        \end{minipage}
      };
    \end{scope}
  \end{tikzpicture}
  \vspace{-1em}
  \begin{itemize}
  \item Fine state $u^{h*}$ recovered \emph{locally} from converged coarse state $u^{H*} = \hat I_h^H u^{h*}$
  \item Normal multigrid cycles visit all levels moving from $n \to n+1$
  \item FMG recovery only accesses levels finer than $\ell_{CP}$
  \item Only need neighborhood of desired region for decompression
  \item Lightweight checkpointing for transient adjoint computation
  \item Postprocessing applications, e.g., in-situ visualization at high temporal resolution in part of the domain
  \end{itemize}
\end{frame}

\begin{frame}[shrink=5]{Four Schools of Thought for Multilevel Methods}
  \begin{itemize}
  \item Multigrid (Brandt, Hackbusch, $\dotsc$)
    \begin{itemize}
    \item originally for resolved/asymptotic spatial discretizations
    \item ``textbook'': reach discretization error in one F-cycle
    \item matrix-light/free, good for memory bandwidth
    \item FAS well-developed for nonlinear problems
    \end{itemize}
  \item Multilevel Domain Decomposition (Mandel, Dohrmann, Widlund)
    \begin{itemize}
    \item leverage direct subdomain solvers, minimize communication
    \item rapid coarsening $\kappa(P^{-1}A) \sim \big(1 + \log \frac H h \big)^{2(L-1)}$
    \item often formulated only as two-level methods, domain-conforming coefficients
    \item lightly developed for nonlinear (e.g. ASPIN [Cai and Keyes])
    \end{itemize}
  \item Multiscale Finite Elements (Babuska, Arbogast, $\dotsc$)
    \begin{itemize}
    \item local preprocessing to construct linear coarse operator
    \item popular in porous media and composite materials (robust theory)
    \end{itemize}
  \item Equation-based multiscale models (many)
    \begin{itemize}
    \item Renormalization multigrid/systematic upscaling (Brandt)
      \begin{itemize}
      \item interpolation, equilibriation (compatible relaxation/Monte-Carlo), restriction
      \end{itemize}
    \item Heterogeneous multiscale method (E, Engquist)
      \begin{itemize}
      \item reconstruction, constrained microscale simulation, data processing/compression
      \end{itemize}
    \end{itemize}
  \end{itemize}
\end{frame}


% Coarsening
\begin{frame}{Computable Convergence Measures (Linear correction notation)}
\newcommand\Vcoarse{V_{\text{coarse}}}
\newcommand\Vfine{V_{\text{fine}}}
  \begin{itemize}
  \item Prolongation $P: \Vcoarse \to \Vfine$
  \item Restriction $R: \Vfine \to \Vcoarse$
  \item Smoother $S^{-1} : \Vfine \to \Vfine$ should remove high-frequency component of error
  \item $I - PR: \Vfine \to \Vfine$ removes part of vector visible in coarse space
  \item Error iteration $I - M^{-1}A$, worst-case convergence factor is $\lambda_{\max}$
  \item ``Interpolation must be able to approximate an eigenvector with error bound proportional to the size of the associated eigenvalue.''
    \begin{itemize}
    \item Upper bound for convergence rate: $\max_x \norm{x}_{(I-PR)S(I-PR)}/{\norm{x}_A}$
    \end{itemize}
  \item Distinct challenges to constructing coarse space and operator
    \begin{itemize}
    \item Is the coarse space large enough to distinguish all low-energy modes?
    \item Are those modes accurately represented? (Is $P$ accurate enough?)
    \item Is the coarse operator accurate? (Automatic with Galerkin-type $RAP$ for nice problems.)
    \end{itemize}
  \end{itemize}
\end{frame}

\begin{frame}{Compatible Relaxation}
  \begin{columns}
    \begin{column}{0.5\textwidth}
      \includegraphics[width=\textwidth]{figures/LivneHabituatedCR} \\
      {\small [Livne 2004]}
    \end{column}
    \begin{column}{0.5\textwidth}
      \begin{itemize}
      \item Apply smoother subject to constraint $\hat R x = 0$
        \begin{enumerate}
        \item $\tilde x_n = x_{n-1} + S_A^{-1}\big(r(x_{n-1}) \big)$
        \item $x_n = \tilde x_n + S_R^{-1}\big(\hat R\tilde x_n) \big)$
        \end{enumerate}
      \item Method to determine when coarse space is rich enough
      \item Slow to relax points/regions good candidates for coarse points/aggregates
      \item If subdomain solves used for smoothing, only interfaces are candidates
      \end{itemize}
    \end{column}
  \end{columns}
\end{frame}

\begin{frame}{Coarse basis functions}
  \begin{itemize}
  \item $\norm{PR x}_A + \norm{(I - PR)x}_A \le C \norm{x}_A$
  \item ``decompose any $x$ into parts without increasing energy much''
  \item near-null spaces must be represented exactly (partition of unity)
  \item number of rows of $R$ determined already, usually $P = R^T$
  \item energy minimization with specified support [Wan, Chan, Smith; Mandel, Brezina, Vanek; Xu, Zikatanov]
  \item smoothed aggregation: $P_{\text{smooth}} = (I - \omega D^{-1} A) P_{\text{agg}}$
  \item classical AMG: each fine point processed independently
  \item domain decomposition/multiscale FEM: solve subdomain problems
  \end{itemize}
\end{frame}

\begin{frame}{Example: BDDC/FETI-DP coarse basis function}
    \begin{columns}
    \begin{column}{0.6\textwidth}
      \includegraphics[width=\textwidth]{figures/MandelSousedikBDDCCoarseBasis} \\
      {\small [Mandel and Sousedik 2010]}
    \end{column}
    \begin{column}{0.4\textwidth}
      \begin{itemize}
      \item only low-order continuity between subdomains
      \item corrected by more technical subdomain smoother
      \end{itemize}
    \end{column}
  \end{columns}
\end{frame}

\begin{frame}{Why I like subdomain problems}
  \begin{columns}
    \begin{column}{0.4\textwidth}
      \includegraphics[width=\textwidth]{figures/ArbogastCoarse} \\
      \includegraphics[width=\textwidth]{figures/ArbogastCoarseMs} \\
      {\small [Arbogast 2011]}
    \end{column}
    \begin{column}{0.6\textwidth}
  \begin{itemize}
  \item subassembly avoids explicit matrix triple product $A_{\text{coarse}} \gets P^T A_{\text{fine}} P$
  \item can update the coarse operator locally (e.g.~local nonlinearity)
  \item need not assemble entire fine grid operator
  \item can coarsen very rapidly (at least in smooth regions)
  \item lower communication setup phase
  \end{itemize}      
    \end{column}
  \end{columns}
\end{frame}

\begin{frame}{Subdomain Interfaces and Energy Minimization}
    \begin{columns}
    \begin{column}{0.5\textwidth}
      \includegraphics[width=\textwidth]{figures/MG/XuZikatanovLambda} \\
      {\small [Xu and Zikatanov 2004]}
    \end{column}
    \begin{column}{0.5\textwidth}
      \begin{itemize}
      \item minimize energy of all basis functions (columns of $P$) subject to
        \begin{itemize}
        \item fixed compact support
        \item partition of unity (near-null space)
        \end{itemize}
      \item enforce partition of unity using Lagrange multipliers
        \begin{itemize}
        \item $\lambda(x) = 0$ in coarse element interiors
        \item means that globally optimal coarse basis functions are harmonic extensions of \emph{some} interface values
        \end{itemize}
      \end{itemize}
    \end{column}
  \end{columns}
\end{frame}

\begin{frame}{Local edge/face-centered problems}
  \includegraphics[width=0.9\textwidth]{figures/MG/ArbogastMultiscaleDual}
  \begin{itemize}
  \item Arbogast's multiscale dual-support elements for porous media
    \begin{itemize}
    \item inconsistent for unaligned anisotropy
    \item homogenization approach: upscale effective conductivity tensor from solution of periodic dual-support problem
    \end{itemize}
  \item Dohrmann and Pechstein's balancing domain decomposition for elasticity with unaligned coefficients
    \begin{itemize}
    \item balance ``torn'' interface values $u_{ie},u_{je}$, written in terms of subdomain Schur complements
    \item $\bar f_e = S_{iee} u_{ie} + S_{jee} u_{je}$: sum of forces required along face $e$ to displace subdomains $i$ and $j$ by $u_{ie}, u_{je}$
    \item $\bar u_e = (S_{iee} + S_{jee})^{-1} \bar f_e$: continuous displacement
    \item equivalent to a (different) dual-support basis
    \end{itemize}
  \end{itemize}
\end{frame}

\begin{frame}{Complication for saddle point problems}
  \begin{columns}
    \begin{column}{0.2\textwidth}
      \[ \begin{pmatrix}
        A & B^T \\ B & 0
      \end{pmatrix} \]
    \end{column}
    \begin{column}{0.8\textwidth}
      \includegraphics[width=\textwidth]{figures/MG/StokesDualProblem}
    \end{column}
  \end{columns}
  \begin{itemize}
  \item want uniform stability for coarse problem
    \begin{itemize}
    \item respect inf-sup condition, similar to fine grid
    \item make coarse grid mimic fine grid ($Q_2-P_1^{\text{disc}}$)
    \end{itemize}
  \item \emph{exact} representation of volumetric mode
    \begin{itemize}
    \item we can't cheat on conservation while upscaling
    \item naturally involves face integrals (inconvenient for recursive application)
    \item obtain similar quantity through solution of inhomogeneous Stokes problems
    \end{itemize}
  \item heuristic algebraic coarsening also possible [Adams 2004]
  \end{itemize}
\end{frame}


% Smoothing
\begin{frame}{Nonlinear problems}
  \begin{itemize}
  \item matrix-based smoothers require global linearization
  \item nonlinearity often more efficiently resolved locally
  \item nonlinear additive or multiplicative Schwarz
  \item nonlinear/matrix-free is good if
    \[ C = \frac{(\text{cost to evaluate residual at one point}) \cdot N}{(\text{cost of global residual})} \sim 1 \]
    \begin{itemize}
    \item finite difference: $C < 2$
    \item finite volume: $C \sim 2$, depends on reconstruction
    \item finite element: $C \sim \text{number of vertices per cell}$
    \end{itemize}
  \item larger block smoothers help reduce $C$
  \end{itemize}
  \vspace{-2.5em}
  \hfill \includegraphics[width=0.3\textwidth]{figures/NodeStencil}
\end{frame}

\begin{frame}{Smoothing for saddle point systems}
  \begin{equation*}
  \begin{pmatrix}
    A & B^T \\ B & 0
  \end{pmatrix}
  \begin{pmatrix}
    u \\ p
  \end{pmatrix}
  =
  \begin{pmatrix}
    b \\ 0
  \end{pmatrix}
\end{equation*}
  \begin{itemize}
  \item pressure has no self-coupling
  \item pressure error modes not spectrally separated
  \item approaches
    \begin{itemize}
    \item block smoothers (Vanka)
    \item amplify fine-grid modes (distributive relaxation)
    \item splitting with approximate Schur complement
    \end{itemize}
  \end{itemize}
\end{frame}

\begin{frame}{Vanka block smoothers}
  \includegraphics[width=0.9\textwidth]{figures/VankaStaggeredGrid} \\
  \begin{itemize}
  \item solve pressure-centered cell problems \\
    \quad (better for discontinuous pressure)
  \item robust convergence factor $\sim 0.3$ \emph{if} coarse grids are accurate
  \item 1D energy minimizing interpolants easy and effective
  \item can use assembled sparse matrices, but more efficient without
  \end{itemize}
\end{frame}

\begin{frame}{Changing Associativity: Distributive Smoothing}
  \begin{align*}
    P A x &= P b & AP y = b, & \quad x = Py
  \end{align*}
  \begin{itemize}
  \item Normal Preconditioning: make $PA$ or $AP$ well-conditioned
  \item Alternative: amplify high-frequency modes
    \begin{itemize}
    \item Multigrid smoothers only need to relax high-frequency modes
    \item Easier to do when spectrally separated: $h$-ellipticity
      \begin{itemize}
      \item pointwise smoothers (Gauss-Seidel) and polynomial/multistage methods
      \end{itemize}
    \item Mechanics: form the product $PA$ or $AP$ and apply ``normal'' method
    \item Example (Stokes)
      \begin{align*}
        A &\sim \begin{pmatrix} -\nabla^2 & \nabla \\ \nabla\cdot &
          0 \end{pmatrix} &
        P &\sim \begin{pmatrix} \bm 1 & -\nabla \\ 0 & -\nabla^2 \end{pmatrix} &
        AP &\sim
        \begin{pmatrix}
          -\nabla^2 & \text{``0''} \\ \nabla\cdot & -\nabla^2
        \end{pmatrix}
      \end{align*}
    \end{itemize}
  \item Convergence factor \num{0.32} (as good as Laplace) for smooth problems
  \end{itemize}
\end{frame}

\begin{frame}[fragile]{Coupled MG for Stokes, split smoothers}
\begin{columns}
  \begin{column}{0.3\textwidth}
    \begin{align*}
      J &=
      \begin{pmatrix}
        A & B^T \\ B & C
      \end{pmatrix} \\
      P_{\text{smooth}} &=
      \begin{pmatrix}
        A_{\text{SOR}} & 0 \\
        B & M
      \end{pmatrix}
    \end{align*}
  \end{column}
  \begin{column}{0.7\textwidth}
    \includegraphics[width=\textwidth]{figures/Sinker2}
  \end{column}
\end{columns}
\begin{Verbatim}[formatcom=\footnotesize]
-pc_type mg -pc_mg_levels 5 -pc_mg_galerkin
-mg_levels_pc_type fieldsplit
-mg_levels_pc_fieldsplit_block_size 3
-mg_levels_pc_fieldsplit_0_fields 0,1
-mg_levels_pc_fieldsplit_1_fields 2
-mg_levels_fieldsplit_0_pc_type sor
\end{Verbatim}
\end{frame}


\begin{frame}{Outlook}
  \begin{itemize}
  \item smoothing with point-block Jacobi Chebyshev and scaled diagonal for pressure
  \item needs only (subdomain ``Neumann'') nonlinear function evaluations and assembly of point-block diagonal matrices
  \item convergence rates similar to smoothed aggregation, but without fine-grid assembly
  \item allows local updates of coarse operator, but currently slower due to naive implementation
  \item Development in progress within PETSc
    \begin{itemize}
    \item parallel implementation of dual-support problems without duplicating lots of work
    \item homogenization-based nonlinear coarsening
    \item true $\tau$ formulation with adaptive fine-grid visits and partial coarse operator updates
    \item microstructure-compatible pressure interpolation
    \item ``spectrally-correct'' nonlinear saddle-point smoothers
    \item locally-computable spectral estimates for guaranteed-stable additive smoothers
    \end{itemize}
  \end{itemize}
\end{frame}

\end{document}
