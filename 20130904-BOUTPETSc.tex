% \documentclass[handout]{beamer}
\documentclass{beamer}

\mode<presentation>
{
  \usetheme{ANLBlue}
  % \usefonttheme[onlymath]{serif}
  % \usetheme{Singapore}
  % \usetheme{Warsaw}
  % \usetheme{Malmoe}
  % \useinnertheme{circles}
  % \useoutertheme{infolines}
  % \useinnertheme{rounded}

  \setbeamercovered{transparent=20}
}

\usepackage[english]{babel}
\usepackage[latin1]{inputenc}
\usepackage{alltt,listings,multirow,ulem,siunitx}
\usepackage[absolute,overlay]{textpos}
\TPGrid{1}{1}
\usepackage{pdfpages}
\usepackage{ulem}
\usepackage{multimedia}
\usepackage{multicol}
\newcommand\hmmax{0}
\newcommand\bmmax{0}
\usepackage{bm}
\usepackage{comment}

% font definitions, try \usepackage{ae} instead of the following
% three lines if you don't like this look
\usepackage{mathptmx}
\usepackage[scaled=.90]{helvet}
% \usepackage{courier}
\usepackage[T1]{fontenc}
\usepackage{tikz}
\usetikzlibrary{decorations.pathreplacing}
\usetikzlibrary{shadows,arrows,shapes.misc,shapes.arrows,shapes.multipart,arrows,decorations.pathmorphing,backgrounds,positioning,fit,petri,calc,shadows,chains,matrix}


% \usepackage{pgfpages}
% \pgfpagesuselayout{4 on 1}[a4paper,landscape,border shrink=5mm]

\usepackage{JedMacros}

\newcommand{\timeR}{t_{\mathrm{R}}}
\newcommand{\timeW}{t_{\mathrm{W}}}
\newcommand{\mglevel}{\ensuremath{\ell}}
\newcommand{\mglevelcp}{\ensuremath{\mglevel_{\mathrm{cp}}}}
\newcommand{\mglevelcoarse}{\ensuremath{\mglevel_{\mathrm{coarse}}}}
\newcommand{\mglevelfine}{\ensuremath{\mglevel_{\mathrm{fine}}}}

%solution and residual
\newcommand{\vx}{\ensuremath{x}}
\newcommand{\vc}{\ensuremath{\hat{x}}}
\newcommand{\vr}{\ensuremath{r}}
\newcommand{\vb}{\ensuremath{b}}

%operators
\newcommand{\vA}{\ensuremath{A}}
\newcommand{\vP}{\ensuremath{I_H^h}}
\newcommand{\vS}{\ensuremath{S}}
\newcommand{\vR}{\ensuremath{I_h^H}}
\newcommand{\vI}{\ensuremath{\hat I_h^H}}
\newcommand{\vV}{\ensuremath{\mathbf{V}}}
\newcommand{\vF}{\ensuremath{F}}
\newcommand{\vtau}{\ensuremath{\mathbf{\tau}}}


\title{PETSc and BOUT++}
\author{{\bf Jed Brown} \\
Peter Brune, Emil Constantinescu, \\
Debojyoti Ghosh, Lois Curfman McInnes \\
\texttt{\{jedbrown,brune,emconsta,ghosh,curfman\}@mcs.anl.gov}
}

% - Use the \inst command only if there are several affiliations.
% - Keep it simple, no one is interested in your street address.
\institute
{
  Mathematics and Computer Science Division \\ Argonne National Laboratory
}

\date{BOUT++ Workshop, 2013-09-04}

% This is only inserted into the PDF information catalog. Can be left
% out.
\subject{Talks}


% If you have a file called "university-logo-filename.xxx", where xxx
% is a graphic format that can be processed by latex or pdflatex,
% resp., then you can add a logo as follows:

% \pgfdeclareimage[height=0.5cm]{university-logo}{university-logo-filename}
% \logo{\pgfuseimage{university-logo}}



% Delete this, if you do not want the table of contents to pop up at
% the beginning of each subsection:
% \AtBeginSubsection[]
% {
% \begin{frame}<beamer>
%   \frametitle{Outline}
%   \tableofcontents[currentsection,currentsubsection]
% \end{frame}
% }

\AtBeginSection[]
{
  \begin{frame}<beamer>
    \frametitle{Outline}
    \tableofcontents[currentsection]
  \end{frame}
}

% If you wish to uncover everything in a step-wise fashion, uncomment
% the following command:

% \beamerdefaultoverlayspecification{<+->}

\begin{document}
\lstset{language=C}
\normalem

\begin{frame}
  \titlepage
\end{frame}

\begin{frame}{Portable {\bf Extensible} Toolkit for Scientific computing}
\begin{block}{Philosophy: Everything has a plugin architecture}
\begin{itemize}
  \item Vectors, Matrices, Coloring/ordering/partitioning algorithms
  \item Preconditioners, Krylov accelerators
  \item Nonlinear solvers, Time integrators
  \item Spatial discretizations/topology$^*$
\end{itemize}
\end{block}
\begin{example}
	Vendor supplies matrix format and associated preconditioner, distributes
	compiled shared library.  Application user loads plugin at runtime, no source
	code in sight.
\end{example}
\end{frame}

\begin{frame}{Portable Extensible {\bf Toolkit} for Scientific computing}
Algorithms, (parallel) debugging aids, low-overhead profiling
\begin{block}{Composability}
Try new algorithms by choosing from product space and composing
existing algorithms (multilevel, domain decomposition, splitting).
\end{block}
\begin{block}{Experimentation}
\begin{itemize}
  \item It is not possible to pick the solver \emph{a priori}. \\
  What will deliver best/competitive performance for a given physics, discretization, architecture, and problem size?
  \item PETSc's response: expose an algebra of composition so new solvers can be created at runtime.
  \item Important to keep solvers decoupled from physics and discretization because we also experiment with those. 
\end{itemize}
\end{block}
\end{frame}

\section{Time Integration}
\begin{frame}{Trade-offs in time integration}
  \begin{itemize}
  \item Properties
    \begin{itemize}
    \item Nonlinear stability (e.g., positivity preservation)
    \item Stability along imaginary axis
    \item $L$-stability (damping at infinity)
    \item Implicitness and reuse
    \end{itemize}
  \item What is expensive?
    \begin{itemize}
    \item Function evaluation
    \item Operator assembly/preconditioner setup
      \begin{itemize}
      \item How much can be reused for how long?
      \end{itemize}
    \item Implicit solves
      \begin{itemize}
      \item Can we find better solver algorithm?
      \item More effort in setup?
      \end{itemize}
    \end{itemize}
  \item What is ``convergence''?
    \begin{itemize}
    \item Wave propagation: implicitness useless for convergence \emph{in a norm}
    \item Non-norm functionals could be robust
    \end{itemize}
  \end{itemize}
\end{frame}

\begin{frame}{Reusing implicit solver setup}
  \begin{itemize}
  \item Linearization
  \item MG interpolants
  \item Lagged preconditioner
  \item Modified Newton
  \item Quasi-Newton
  \item IMEX with linear implicit part
  \item Rosenbrock/W
  \end{itemize}
\end{frame}

\begin{frame}[shrink=5]{IMEX time integration in PETSc}
  \begin{itemize}
  \item Additive Runge-Kutta IMEX methods
    \begin{gather*}
      G(t,x,\dot x) = F(t,x) \\
      J_\alpha = \alpha G_{\dot x} + G_x
    \end{gather*}
    \begin{itemize}
    \item User provides:
      \begin{itemize}
      \item \texttt{FormRHSFunction(ts,$t$,$x$,$F$,void *ctx);}
      \item \texttt{FormIFunction(ts,$t$,$x$,$\dot x$,$G$,void *ctx);}
      \item \texttt{FormIJacobian(ts,$t$,$x$,$\dot x$,$\alpha$,$J$,$J_{p}$,mstr,void *ctx);}
      \end{itemize}
    \item L-stable DIRK for stiff part $G$
    \item Choice of explicit method, \eg SSP
    \item Orders 2 through 5, embedded error estimates
    \item Dense output, hot starts for Newton
    \item More accurate methods if $G$ is linear, also Rosenbrock-W
    \item Can use preconditioner from classical ``semi-implicit'' methods
    \item Extensible adaptive controllers, can change order within a family
    \item Easy to register new methods: \code{TSARKIMEXRegister()}
    \end{itemize}
  \item Eliminate many interface quirks
  \item Single step interface so user can have own time loop
  \end{itemize}
\end{frame}

% \begin{frame}{TS Examples}
  \begin{itemize}
  \item 1D nonlinear hyperbolic conservation laws
    \begin{itemize}
    \item \code{src/ts/examples/tutorials/ex9.c}
    \item {\footnotesize \code{./ex9 -da\_grid\_x 100 -initial 1 -physics shallow -limit minmod -ts\_ssp\_type rks2 -ts\_ssp\_nstages 8 -ts\_monitor\_draw\_solution}}
    \end{itemize}
  \item Stiff linear advection-reaction test problem
    \begin{itemize}
    \item \code{src/ts/examples/tutorials/ex22.c}
    \item {\footnotesize \code{./ex22 -da\_grid\_x 200 -ts\_monitor\_draw\_solution -ts\_type rosw -ts\_rosw\_type ra34pw2 -ts\_adapt\_monitor}}
    \end{itemize}
  \item 1D Brusselator (reaction-diffusion)
    \begin{itemize}
    \item \code{src/ts/examples/tutorials/ex25.c}
    \item {\footnotesize \code{./ex25 -da\_grid\_x 40 -ts\_monitor\_draw\_solution -ts\_type rosw -ts\_rosw\_type 2p -ts\_adapt\_monitor}}
    \end{itemize}
  \end{itemize}
\end{frame}

%\input{slides/TS/}

\begin{frame}[fragile]{Time integration method design}
  \begin{figure}
    \centering
    \includegraphics[width=.8\textwidth]{figures/TS/EmilMethodDesignFeatures.png}
  \end{figure}
  \begin{itemize}
  \item Select order, number of stages, required properties
  \item Optimize properties like SSP coefficient, accuracy, or linear stability
  \item \cverb|TSARKIMEXRegister("my-method", ...coefficients...)|
  \item \cverb|-ts_type arkimex -ts_arkimex_type my-method|
  \end{itemize}
\end{frame}

\begin{frame}{Example: Additive Runge-Kutta design}
  \begin{itemize}
  \item 3-stage, second order, $L$-stable implicit part
  \item one-parameter family of solutions
  \end{itemize}
  \begin{description}
  \item[ARK2c] Maximize SSP coefficient
  \item[ARK2E] Minimize leading error coefficient
  \end{description}
  \begin{figure}
    \centering
    \includegraphics[width=0.55\textwidth]{figures/TS/ssp_ark_poster.png}
    \includegraphics[width=0.49\textwidth]{figures/TS/Stability_ARK2E_ARK2C.pdf}
  \end{figure}
\end{frame}

\begin{frame}{Some TS methods}
  \begin{description}
  \item[TSSSPRK104] 10-stage, fourth order, low-storage, optimal explicit SSP Runge-Kutta $c_{\text{eff}} = 0.6$ (Ketcheson 2008)
  \item[TSARKIMEX2E] second order, one explicit and two implicit stages, $L$-stable, optimal (Constantinescu)
  \item[TSARKIMEX3] (and 4 and 5), $L$-stable (Kennedy and Carpenter, 2003)
  \item[TSROSWRA3PW] three stage, third order, for index-1 PDAE, $A$-stable, $R(\infty) = 0.73$, second order strongly $A$-stable embedded method (Rang and Angermann, 2005)
  \item[TSROSWRA34PW2] four stage, third order, $L$-stable, for index 1 PDAE, second order strongly $A$-stable embedded method (Rang and Angermann, 2005)
  \item[TSROSWLLSSP3P4S2C] four stage, third order, $L$-stable implicit, SSP explicit, $L$-stable embedded method (Constantinescu)
  \end{description}
\end{frame}


\begin{frame}{Adaptive controllers}
  \begin{itemize}
  \item ``Stiff'' waves are not stiff if one wants to converge \emph{in a norm}
  \item PETSc integrators provide embedded methods to estimate errors
  \item Automatic controllers optimize local truncation error and nonlinear solve cost
  \item User can register custom controllers
  \item Use a priori knowledge of the physics, robust functionals
  \item Choose from list of methods, choose next step size
  \end{itemize}
\end{frame}

\section{Nonlinear solvers}
\begin{frame}{Which nonlinear solver?}
  \begin{itemize}
  \item Global linearization (NewtonLS, NewtonTR)
    \begin{itemize}
    \item Preconditioning libraries for assembled matrices
    \item Low arithmetic intensity
    \end{itemize}
  \item Quasi-Newton
    \begin{itemize}
    \item Build low-rank updates to Jacobian inverse
    \item Brown and Brune, ``Low-rank quasi-Newton updates for robust Jacobian lagging in Newton-type methods'', ANS MC13.
    \end{itemize}
  \item Nonlinear multigrid and domain decomposition
    \begin{itemize}
    \item ASPIN (left-preconditioned nonlinear Schwarz), also right-preconditioned
    \item Full Approximation Scheme with linear or nonlinear smoothers
    \item More intrusive, but freakishly efficient for difficult problems
    \end{itemize}
  \item Nonlinear GMRES, Anderson mixing, nonlinear CG
    \begin{itemize}
    \item Accelerator for nonlinear preconditioning
    \item Good alternative to matrix-free finite differencing
    \item More robust line search possible: operates in reduced basis
    \end{itemize}
  \end{itemize}
\end{frame}
\begin{frame}
  \includegraphics[width=\textwidth]{figures/BruneNGMRESFAS2.png}
\end{frame}

\begin{frame}{The Great Solver Schism: Monolithic or Split?}
  \begin{columns}
    \begin{column}{0.5\textwidth}
      \begin{block}{Monolithic}
        \begin{itemize}
        \item Direct solvers
        \item Coupled Schwarz
        \item Coupled Neumann-Neumann \\
          (need unassembled matrices)
        \item Coupled multigrid
        \item[X] Need to understand local spectral and compatibility properties of the coupled system
        \end{itemize}
      \end{block}
    \end{column}
    \begin{column}{0.5\textwidth}
      \begin{block}{Split}
        \begin{itemize}
        \item Physics-split Schwarz \\
          (based on relaxation)
        \item Physics-split Schur \\
          (based on factorization)
          \begin{itemize}
          \item  approximate commutators \\
            SIMPLE, PCD, LSC
          \item segregated smoothers
          \item Augmented Lagrangian
          \item ``parabolization'' for stiff waves
          \end{itemize}
        \item[X] Need to understand global coupling strengths
        \end{itemize}
      \end{block}
    \end{column}
  \end{columns}
  \begin{itemize}
  \item Preferred data structures depend on which method is used.
  \item Interplay with geometric multigrid.
  \end{itemize}
\end{frame}

\begin{frame}
  \includegraphics[width=\textwidth]{figures/PETSc/LocalSpaces} \\[-.5em]
  Work in Split Local space, matrix data structures reside in any space.
\end{frame}


\section{Comments on performance}
\begin{frame}{Bottlenecks of (Jacobian-free) Newton-Krylov}
  \begin{columns}
    \begin{column}{0.4\textwidth}
      \includegraphics[width=1.15\textwidth]{figures/Dohp/EllipRCM}
    \end{column}
    \begin{column}{0.6\textwidth}
      \begin{itemize}
      \item Matrix assembly
        \begin{itemize}
        \item integration/fluxes: FPU
        \item insertion: memory/branching
        \end{itemize}
      \item Preconditioner setup
        \begin{itemize}
        \item coarse level operators
        \item overlapping subdomains
        \item (incomplete) factorization
        \end{itemize}
      \item Preconditioner application
        \begin{itemize}
        \item triangular solves/relaxation: memory
        \item coarse levels: network latency
        \end{itemize}
      \item Matrix multiplication
        \begin{itemize}
        \item Sparse storage: memory
        \item Matrix-free: FPU
        \end{itemize}
      \item Globalization
      \end{itemize}
    \end{column}
  \end{columns}
\end{frame}

\begin{frame}{Scalability Warning}
  \begin{quote}\Large \centering
    The easiest way to make software scalable \\
    is to make it sequentially inefficient. \\
    (Gropp 1999)
  \end{quote}

  \begin{itemize}
  \item We really want \emph{efficient} software
  \item Need a performance model
    \begin{itemize}
    \item memory bandwidth and latency
    \item algorithmically critical operations (\eg dot products, scatters)
    \item floating point unit
    \end{itemize}
  \item Scalability shows marginal benefit of adding more cores, nothing more
  \item Constants hidden in the choice of algorithm
  \item Constants hidden in implementation
  \end{itemize}
\end{frame}

\begin{frame}[shrink=5]{Performance of assembled versus unassembled}
  \includegraphics[width=\textwidth]{figures/TensorVsAssembly} \\
  \begin{itemize}
  \item High order Jacobian stored unassembled using coefficients at quadrature points, can use local AD
  \item Choose approximation order at run-time, independent for each field
  \item Precondition high order using assembled lowest order method
  \item Implementation $> 70\%$ of FPU peak, SpMV bandwidth wall $< 4\%$
  \end{itemize}
\end{frame}

\begin{frame}{Hardware Arithmetic Intensity}
  \begin{tabular}{lc}
    \toprule
    Operation                         & Arithmetic Intensity (flops/B) \\
    \midrule
    Sparse matrix-vector product      & 1/6                  \\
    Dense matrix-vector product       & 1/4                  \\
    Unassembled matrix-vector product & $\approx 8$          \\
    High-order residual evaluation    & $> 5$                \\
    \bottomrule
  \end{tabular}

  \bigskip

  \begin{tabular}{lrrr}
    \toprule
    Processor & BW (GB/s) & Peak (GF/s) & Balanced AI (F/B) \\
    \midrule
    E5-2670 8-core      & 35   & 166  & 4.7 \\
    Magny Cours 16-core & 49   & 281  & 5.7 \\
    Blue Gene/Q node    & 43   & 205  & 4.8 \\
    Tesla M2090         & 120  & 665  & 5.5 \\
    Kepler K20Xm        & 160 & 1310 & 8.2 \\ % http://www.elekslabs.com/2012/11/nvidia-tesla-k20-benchmark-facts.html
    Xeon Phi            & 150 & 1248 & 8.3 \\
    \bottomrule
  \end{tabular}
\end{frame}


\end{document}
