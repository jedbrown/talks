% \documentclass[handout]{beamer}
\documentclass{beamer}

\mode<presentation>
{
  \usetheme{default}
  \usefonttheme[onlymath]{serif}
  % \usetheme{Singapore}
  % \usetheme{Warsaw}
  % \usetheme{Malmoe}
  % \useinnertheme{circles}
  % \useoutertheme{infolines}
  % \useinnertheme{rounded}

  \setbeamercovered{transparent=5}
}

\usepackage[english]{babel}
\usepackage[latin1]{inputenc}
\usepackage{textpos,alltt,listings,multirow,ulem,siunitx}
\newcommand\hmmax{0}
\newcommand\bmmax{0}
\usepackage{bm}

% font definitions, try \usepackage{ae} instead of the following
% three lines if you don't like this look
\usepackage{mathptmx}
\usepackage[scaled=.90]{helvet}
% \usepackage{courier}
\usepackage[T1]{fontenc}
\usepackage{tikz}
\usetikzlibrary[shapes,shapes.arrows,arrows,shapes.misc,fit,positioning]

% \usepackage{pgfpages}
% \pgfpagesuselayout{4 on 1}[a4paper,landscape,border shrink=5mm]

\usepackage{JedMacros}

\title{Utilizing Multicore and GPU Hardware for Multiphysics Simulation through Implicit High-order Finite Element Methods with Tensor Product Structure}
\author{Jed Brown\inst{1}, Aron Ahmadia\inst{2}, Matt Knepley\inst{3}, Barry Smith\inst{1}}


% - Use the \inst command only if there are several affiliations.
% - Keep it simple, no one is interested in your street address.
\institute
{
  \inst{1}{Mathematics and Computer Science Division, Argonne National Laboratory} \\
  \inst{2}{King Abdullah University of Science and Technology} \\
  \inst{3}{Computation Institute, University of Chicago}
}

\date{2012-02-15}

% This is only inserted into the PDF information catalog. Can be left
% out.
\subject{Talks}


% If you have a file called "university-logo-filename.xxx", where xxx
% is a graphic format that can be processed by latex or pdflatex,
% resp., then you can add a logo as follows:

% \pgfdeclareimage[height=0.5cm]{university-logo}{university-logo-filename}
% \logo{\pgfuseimage{university-logo}}



% Delete this, if you do not want the table of contents to pop up at
% the beginning of each subsection:
% \AtBeginSubsection[]
% {
% \begin{frame}<beamer>
%   \frametitle{Outline}
%   \tableofcontents[currentsection,currentsubsection]
% \end{frame}
% }

\AtBeginSection[]
{
  \begin{frame}<beamer>
    \frametitle{Outline}
    \tableofcontents[currentsection]
  \end{frame}
}

% If you wish to uncover everything in a step-wise fashion, uncomment
% the following command:

% \beamerdefaultoverlayspecification{<+->}

\begin{document}
\lstset{language=C}
\normalem

\begin{frame}
  \titlepage
\end{frame}

\begin{frame}{The Roadmap}
  \begin{block}{Hardware trends}
    \begin{itemize}
    \item More cores (keep hearing $\bigO(1000)$ per node)
    \item Long vector registers (already 32 bytes for AVX and BG/Q)
    \item Must use SMT to hide memory latency
    \item Must use SMT for floating point performance (GPU, BG/Q)
    \item Large penalty for non-contiguous memory access
    \end{itemize}
  \end{block}
  \begin{block}{``Free flops'', but how can we use them?}
    \begin{itemize}
    \item High order methods good: better accuracy per storage
    \item High order methods bad: work unit gets larger
    \item GPU threads have very little memory, must keep work unit small
    \item Need library composability, keep user contribution embarrassingly parallel
    \end{itemize}
  \end{block}
\end{frame}


\begin{frame}{How to program this beast?}
  \begin{itemize}
  \item Decouple physics from discretization
    \begin{itemize*}
    \item Expose small, embarrassingly parallel operations to user
    \item Library schedules user threads for reuse between kernels
    \item User provides physics in kernels run at each quadrature point
    \item Continuous weak form: find $u \in \VV_D$
      \[ v^T F(u) \sim \int_\Omega v \cdot {\color{green!70!black} f_0(u,\nabla u)}
      + \nabla v \tcolon {\color{green!70!black} f_1(u,\nabla u)} = 0, \qquad \forall v \in \VV_0 \]
    \item Similar form at faces, but may involve Riemann solve
    \end{itemize*}
  \item Library manages reductions
    \begin{itemize*}
    \item Interpolation and differentiation on elements
    \item Interaction with neighbors (limiting, edge stabilization)
    \item Exploit tensor product structure to keep working set small
    \item Assembly into solution/residual vector (sum over elements)
    \end{itemize*}
  \end{itemize}
\end{frame}

\begin{frame}{Nodal $hp$-version finite element methods}
  \begin{columns}
    \begin{column}{0.4\textwidth}
      \includegraphics[width=\textwidth]{figures/lgl}
    \end{column}
    \begin{column}{0.6\textwidth}
      \begin{block}{1D reference element}
        \begin{itemize}
        \item Lagrange interpolants on Legendre-Gauss-Lobatto points
        \item Quadrature $\hat R$, weights $\hat W$
        \item Evaluation: $\hat B, \hat D$
        \end{itemize}
      \end{block}
      \vspace{-1em}
      \begin{block}{3D reference element}
      \vspace{-1em}
        \begin{align*}\label{eq:tprod}
          \begin{split}
            % \hat{\bm R} &= \hat R \otimes \hat R \otimes \hat R \\
            \hat{\bm W} &= \hat W \otimes \hat W \otimes \hat W \\
            \hat{\bm B} &= \alert<2>{\hat B \otimes \hat B \otimes \hat B} \\
          \end{split} &
          \begin{split}
            \hat{\bm D}_0 &= \alert<2>{\hat D \otimes \hat B \otimes \hat B} \\
            \hat{\bm D}_1 &= \alert<2>{\hat B \otimes \hat D \otimes \hat B} \\
            \hat{\bm D}_2 &= \alert<2>{\hat B \otimes \hat B \otimes \hat D} \\
          \end{split}
        \end{align*}
        \vspace{-1em}
        \begin{block}<2>{\alert{These tensor product operations \\ 
              are very efficient, 70\% of peak flop/s}}
        \end{block}
      \end{block}
    \end{column}
  \end{columns}
\end{frame}

\begin{frame}{Operations on physical elements}
  \begin{block}{Mapping to physical space}
    \vspace{-2em}
    \begin{gather*}
      x^e : \hat K \to K^e,\quad J^e_{ij} = \partial x_i^e/\partial \hat x_j, \quad (J^e)^{-1} = \partial \hat x/\partial x^e \\
    \end{gather*}
  \vspace{-2em}
  \end{block}
  \vspace{-2em}
  \begin{block}{Element operations in physical space}
  \vspace{-2em}
    \begin{align*}
      \bm B^e &= \hat{\bm B} \qquad \qquad \qquad \bm W^e = \hat{\bm W} \Lambda(\abs{J^e(\bm r)}) \\
      \bm D^e_i &= \Lambda\left(\frac{\partial \hat x_0}{\partial x_i}\right) \hat{\bm D}_0
      + \Lambda\left(\frac{\partial \hat x_1}{\partial x_i}\right) \hat{\bm D}_1
      + \Lambda\left(\frac{\partial \hat x_2}{\partial x_i}\right) \hat{\bm D}_2 \\
      (\bm D^e_i)^T &= \hat{\bm D}_0^T \Lambda\left(\frac{\partial \hat x_0}{\partial x_i}\right)
      + \hat{\bm D}_1^T \Lambda\left(\frac{\partial \hat x_1}{\partial x_i}\right)
      + \hat{\bm D}_2^T \Lambda\left(\frac{\partial \hat x_2}{\partial x_i}\right)
    \end{align*}
  \end{block}
  \vspace{-2em}
  \begin{block}{Global problem is defined by assembly}
  \vspace{-2em}
  \begin{equation*}
    F(u) =
    \sum_e \EE_e^T \Big[ (\bm B^e)^T \bm W^e \Lambda({\color{green!70!black} f_0(u^e,\nabla u^e)})
    + \sum_{i=0}^d(\bm D_i^e)^T \bm W^e \Lambda({\color{green!70!black} f_{1,i}(u^e,\nabla u^e)}) \Big] = \bm 0
  \end{equation*}
  where $u^e = \bm B^e \EE^e u$ and $\nabla u^e = \{\bm D_i^e \EE^e u\}_{i=0}^2$
  \end{block}
\end{frame}

\begin{frame}[shrink=5]{Representation of Jacobians, Automation}
  \begin{itemize}
  \item For unassembled representations, decomposition, and assembly
  \item Continuous weak form: find $u$
    \[ v^T F(u) \sim \int_\Omega v \cdot {\color{green!70!black} f_0(u,\nabla u)}
    + \nabla v \tcolon {\color{green!70!black} f_1(u,\nabla u)} = 0, \qquad \forall v \in \VV_0 \]
  \item Weak form of the Jacobian $J(u)$: find $w$
    \begin{gather*}
      v^T J(u) w \sim \int_\Omega \begin{bmatrix} v^T & \nabla v^T \end{bmatrix}
      {\color{blue} \begin{bmatrix} f_{0,0} & f_{0,1} \\ f_{1,0} & f_{1,1} \end{bmatrix}}
      \begin{bmatrix} w \\ \nabla w \end{bmatrix} \\
      {\color{blue} [f_{i,j}] = \begin{bmatrix} \dfrac{\partial f_0}{\partial u} & \dfrac{\partial f_0}{\partial \nabla u} \\[1em]
          \dfrac{\partial f_1}{\partial u} & \dfrac{\partial f_1}{\partial \nabla u} \end{bmatrix} (u,\nabla u) }
    \end{gather*}
  \item Terms in ${\color{blue} [f_{i,j}]}$ easy to compute symbolically, AD more scalable.
  \item Nonlinear terms ${\color{green!70!black}f_0,f_1}$ usually have the most expensive nonlinearities in the computation of scalars
    \begin{itemize}
    \item Equations of state, effective viscosity
    \item Compute gradient with reverse-mode, store at quadrature points.
    \item Perturb scalars, then use forward-mode to complete the Jacobian.
    \item Flip for action of the adjoint.
    \end{itemize}
  \end{itemize}
\end{frame}

\newcommand\smallterm[1]{{\color{gray} #1}}
\begin{frame}{Conservative (non-Boussinesq) two-phase ice flow}
  Find momentum density $\rho\uu$, pressure $p$, and total energy density $E$:
  \begin{gather*}
    (\rho\uu)_t + \div (\smallterm{\rho\uu\otimes\uu} - \eta D\uu_i + p\bm 1) - \rho \bm g = 0 \\
    \rho_t + \div \rho\uu = 0 \\
    E_t + \div \big((E+p)\uu - k_T\nabla T - k_\omega\nabla\omega \big) - \eta D\uu_i\tcolon D\uu_i - \smallterm{\rho\uu\cdot\bm g} = 0
  \end{gather*}
\begin{itemize}
\item Solve for density $\rho$, ice velocity $\uu_i$, temperature $T$, and melt fraction $\omega$ using constitutive relations.
  \begin{itemize}
  \item Simplified constitutive relations can be solved explicitly.
  \item Temperature, moisture, and strain-rate dependent rheology $\eta$.
  \item High order FEM, typically $Q_3$ momentum \& energy, SUPG (yuck).
  \end{itemize}
\item DAEs solved implicitly after semidiscretizing in space.
\item Preconditioning using nested fieldsplit
\end{itemize}
\end{frame}
\begin{frame}{How much nesting?}
  \begin{columns}
    \begin{column}{0.5\textwidth}
      \begin{equation*}
        P_1 =
        \begin{pmatrix}
          J_{uu} & J_{up} & J_{uE} \\
          0 & B_{pp} & 0 \\
          0 & 0 & J_{EE} \\
        \end{pmatrix}
      \end{equation*}
      \begin{itemize}
      \item $B_{pp}$ is a mass matrix in the pressure space weighted by inverse of kinematic viscosity.
      \item Elman, Mihajlovi\'c, Wathen, JCP 2011 for non-dimensional isoviscous Boussinesq.
      \item Works well for non-dimensional problems on the cube, not for realistic parameters.
      \end{itemize}
    \end{column}
    \begin{column}{0.5\textwidth}
      \begin{equation*}
        P =
        \begin{bmatrix}
          \begin{pmatrix}
            J_{uu} & J_{up} \\
            J_{pu} & 0
          \end{pmatrix} & \\
          \begin{pmatrix}
            J_{Eu} & J_{Ep}
          \end{pmatrix}
          & J_{EE}
        \end{bmatrix}
      \end{equation*}
      \begin{itemize}
      \item Inexact inner solve using upper-triangular with $B_{pp}$ for Schur.
      \item Another level of nesting.
      \item GCR tolerant of inexact inner solves.
      \item Outer converges in 1 or 2 iterations.
      \end{itemize}
    \end{column}
  \end{columns}
  \begin{itemize}
  \item Low-order preconditioning full-accuracy unassembled high order operator.
  \item Build these on command line with PETSc \cverb|PCFieldSplit|.
  \end{itemize}
\end{frame}


\begin{frame}
  \includegraphics[width=\textwidth]{figures/VHT/TopViewStreamline}
\end{frame}
\begin{frame}
  \includegraphics[width=\textwidth]{figures/VHT/VHTJakoContourStream}
\end{frame}

\begin{frame}[fragile,shrink=5]{CPU traversal code}
  \begin{itemize*}
  \item CPU traversal computes coefficients of test functions, \url{https://github.com/jedbrown/dohp/}
  \begin{ccode}
  while (IteratorHasPatch(iter)) {
    IteratorGetPatchApplied(iter,&Q,&jw,
        &x,&dx,NULL,NULL,
        &u,&du,&u_,&du_, &p,&dp,&p_,NULL, &e,&de,&e_,&de_);
    IteratorGetStash(iter,NULL,&stash);
    for (dInt i=0; i<Q; i++) {
      PointwiseFunction(context,x[i],dx[i],jw[i],
          u[i],du[i],p[i],dp[i],e[i],de[i],
          &stash[i], u_[i],du_[i],p_[i],e_[i],de_[i]);
    }
    IteratorCommitPatchApplied(iter,INSERT_VALUES, NULL,NULL,
                               u_,du_, p_,NULL, e_,de_);
    IteratorNextPatch(iter);
  }
  \end{ccode}
  \item GPU version calls \cfunc|PointwiseFunction| directly.
  \item Unassembled Jacobian application reuses \cverb|stash|
    \begin{ccode}
  PointwiseJacobian(context,&stash[i],dx[i],jw[i],
                    u[i],du[i],p[i],dp[i],e[i],de[i],
                    u_[i],du_[i],p_[i],e_[i],de_[i]);
    \end{ccode}
  \end{itemize*}
\end{frame}

\input{slides/GPU/FinerGrainedParallelismForGPUFEM.tex}
\begin{frame}{Finer grained parallelism for GPUs, low order}
%\begin{figure}
\centering
\tikzstyle{work group} = [draw,green,rounded corners]
\tikzstyle{thread}     = [fill,red]
\tikzstyle{operation}  = [dashed,rounded corners]
\tikzstyle{notation}   = [anchor=south,text width=4cm,text centered]
\begin{tikzpicture}[scale=0.4]
% Draw the basis function evaluation
\draw[operation] (-0.5,-0.5) rectangle +(9,12)
  ++(4.5,12) node[notation] {Map values at quadrature points to coefficients};
% Draw the basis function evaluation breakdown
\foreach \x in {0, 3, 6}
  \foreach \y/\j in {0/1, 6/0}
{
  % draw a cell work unit
  \path[work group] (\x,\y) rectangle +(2,5);
  % draw a 3 thread configuration
 \path[thread] (\x,\y)
             ++(0.5,0.5)
              +(0,0) rectangle +(1,1) +(0.5,0.5) node[anchor=mid,text=white] {\pgfmathtruncatemacro{\t}{\j*3+2}$t_{\t}$}
             ++(0.0,1.5)
              +(0,0) rectangle +(1,1) +(0.5,0.5) node[anchor=mid,text=white] {\pgfmathtruncatemacro{\t}{\j*3+1}$t_{\t}$}
             ++(0.0,1.5)
              +(0,0) rectangle +(1,1) +(0.5,0.5) node[anchor=mid,text=white] {\pgfmathtruncatemacro{\t}{\j*3+0}$t_{\t}$};
}
% Draw arrow for kernel continuation
\draw[<-,ultra thick] (8.5,5.5) -- node[anchor=south] {Continue with kernel} (17.5,5.5);
\draw[dashed] (8.5,11.5) -- (17.5,14.5);
\draw[dashed] (8.5,-0.5) -- (17.5,-3.5);
% Draw the quadrature point evaluation
\draw[operation] (17.5,-3.5) rectangle +(6,18)
  ++(3,18) node[notation] {Evaluate basis and process values at quadrature points};
% Draw the quadrature point evaluation breakdown
\foreach \x in {18, 21}
  \foreach \y/\j in {-3/2, 3/1, 9/0}
{
  % draw a cell work unit
  \path[work group] (\x,\y) rectangle +(2,5);
  % draw a 2 thread configuration
 \path[thread] (\x,\y)
             ++(0.5,1.0)
              +(0,0) rectangle +(1,1) +(0.5,0.5) node[anchor=mid,text=white] {\pgfmathtruncatemacro{\t}{\j*2+1}$t_{\t}$}
             ++(0.0,2.0)
              +(0,0) rectangle +(1,1) +(0.5,0.5) node[anchor=mid,text=white] {\pgfmathtruncatemacro{\t}{\j*2+0}$t_{\t}$};
}
\end{tikzpicture}
% \caption{Action of the residual evaluation kernel on a group of incoming cells. Each cell is displayed as a green,
%   rounded rectangle occupied by the threads which compute the cell information. Each thread computes its values in
%   series, so that thread $t_0$ first computes values at quadrature points for 2 cells, and then computes basis
%   coefficients for 3 cells.}
%\end{figure}
\end{frame}


\begin{frame}[fragile]{Avoiding copies}
  \begin{ccode}
    typedef enum {
      PETSC_CUSP_UNALLOCATED,
      PETSC_CUSP_GPU,
      PETSC_CUSP_CPU,
      PETSC_CUSP_BOTH
    } PetscCUSPFlag;
  \end{ccode}
  \begin{itemize}
  \item Flag used for matrices and vectors.
  \item Data stays on GPU until it is needed on CPU (e.g. for MPI).
  \item Control flow for matrix and vector operations resides on CPU
    \begin{itemize}
    \item almost all implementations run on GPU
    \item can mix and match CPU-only and GPU-accelerated algorithms \\
      (but would need to pay for more copies)
    \end{itemize}
  \item Currently always update the whole array
    \begin{itemize}
    \item could order for low-volume updates
    \end{itemize}
  \end{itemize}
\end{frame}

\begin{frame}[shrink=5]{Performance of assembled versus unassembled}
  \includegraphics[width=\textwidth]{figures/TensorVsAssembly} \\
  \begin{itemize}
  \item High order Jacobian stored unassembled using coefficients at quadrature points, can use local AD
  \item Choose approximation order at run-time, independent for each field
  \item Precondition high order using assembled lowest order method
  \item Implementation $> 70\%$ of FPU peak, SpMV bandwidth wall $< 4\%$
  \end{itemize}
\end{frame}

\begin{frame}{Hardware Arithmetic Intensity}
  \begin{tabular}{lc}
    \toprule
    Operation                         & Arithmetic Intensity (flops/B) \\
    \midrule
    Sparse matrix-vector product      & 1/6                  \\
    Dense matrix-vector product       & 1/4                  \\
    Unassembled matrix-vector product & $\approx 8$          \\
    High-order residual evaluation    & $> 5$                \\
    \bottomrule
  \end{tabular}

  \bigskip

  \begin{tabular}{lrrr}
    \toprule
    Processor & BW (GB/s) & Peak (GF/s) & Balanced AI (F/B) \\
    \midrule
    E5-2670 8-core      & 35   & 166  & 4.7 \\
    Magny Cours 16-core & 49   & 281  & 5.7 \\
    Blue Gene/Q node    & 43   & 205  & 4.8 \\
    Tesla M2090         & 120  & 665  & 5.5 \\
    Kepler K20Xm        & 160 & 1310 & 8.2 \\ % http://www.elekslabs.com/2012/11/nvidia-tesla-k20-benchmark-facts.html
    Xeon Phi            & 150 & 1248 & 8.3 \\
    \bottomrule
  \end{tabular}
\end{frame}


\begin{frame}{On preconditioning and multigrid}
  \begin{itemize}
  \item Currently using assembled matrices for preconditioning
  \item Want matrix-free preconditioners for high hardware utilization
  \item Geometric $h$- and $p$-multigrid, could be FAS
  \item Smoothers build/solve with small dense matrices
    \begin{itemize}
    \item ``point'' matrices: can use single threads
    \item ``element'' matrices: need to cooperate within thread blocks
    \item I want a dense linear algebra library to be called collectively within a thread block
    \end{itemize}
  \item Multiplicative (Gauss-Seidel) is algorithmically nice
  \item Spectral analysis for polynomial/multi-stage smoothers
  \item Coarser levels better to do on CPU
    \begin{itemize}
    \item Potential for additive correction to run concurrently
    \end{itemize}
  \end{itemize}
\end{frame}

\begin{frame}{Outlook}
  \begin{itemize}
  \item Sparse matrix assembly (for preconditioning)
    \begin{itemize}
    \item $> 100$ GF/s for lowest order Stokes (Matt Knepley)
    \item common ``pointwise'' physics code with CPU implementation
    \item Dohp CPU version faster than libMesh and Deal.II for $Q_1$
    \item $Q_1$ assembly embedded in higher order is 8\% slower than hand-rolled
    \end{itemize}
  \item Matrix-free tensor-product versions reliably get about 70\% of peak flops
  \item Finer grained parallelism in GPU tensor product kernels
  \item Can't wait for OpenCL to implement indirect function calls
  \item Symbolic differentiation too slow, tired of hand-differentiation
    \begin{itemize}
    \item I want source-transformation AD with indirect function calls
    \end{itemize}
  \item Find correct amount of reuse between face and cell integration
  \item Riemann solves harder to vectorize
  \item Hide dispatch to pointwise kernels inside library
    \begin{itemize}
    \item Easy, but scary. Library/framework becomes \alert{\bf F}ramework.
    \item Interoperbility of user-rolled, library-provided, and generated traversal code.
    \end{itemize}
  \end{itemize}
\end{frame}

\end{document}
